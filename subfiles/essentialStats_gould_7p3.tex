\documentclass[../mathNotesPreamble]{subfiles}

\providecommand{\relscalefact}{1.4}
\begin{document}
\relscale{\relscalefact}
  \section{7.3: The Central Limit Theorem for Sample Proportions}
  \begin{defn*}[Central Limit Theorem (CLT)]\mbox{}\\
    When estimating a population proportion, $p$, if
    \begin{enumerate}
      \item \emph{Random and Independent}: The sample is collected randomly from the population, and observations are independent of each other.
      \item \emph{Large Sample}: The sample size, $n$, is large
        enough that the sample can have at least 10 successes or
        failures.
        \begin{align*}
          n\hat{p}&\geq 10\\
          n\parens{1-\hat{p}}&\geq 10
        \end{align*}
      \item \emph{Big population}: If the sample is collected
        without replacement (e.g. SRS), then the population size must be at
        least 10 times bigger than the sample size.
        \begin{align*}
          N&\geq 10n
        \end{align*}
    \end{enumerate}
    then the sampling distribution for $\hat{p}$ is approximately Normal, with mean $p$ and standard deviation
      \[SE=\sqrt{\frac{p(1-p)}{n}}.\]
    This distribution is denoted as
      \[N\parens{p,\sqrt{\frac{p(1-p)}{n}}}.\]
  \end{defn*}
  \pagebreak

  \begin{ex*}
    Consider the groups from the previous section where $p=0.25$ of the group identified as female. Suppose that $\hat{p}=0.25$. If $N$ represents the population size, and $n$ the sample size, identify if the CLT can be applied.
  \end{ex*}
  \begin{extasks}[after-item-skip=\stretch{1}](1)
    \task $N=8, n=4$
    \task $N=1000, n=10$
    \task $N=1000, n=100$
  \end{extasks}
  \vspace*{\stretch{1}}
  \pagebreak

  \begin{ex*}
    Consider the group of 1000 people where $p=0.25$ identified as female. In a sample of $n=100$ people, what is the probability that $\hat{p}$ is at least 29\%?
  \end{ex*}
  \begin{flushright}
    \begin{tikzpicture}[scale=1.0, declare function={
      N=50;
      mu=0.25; sig=0.0433;
      xmin=mu-3.2*sig;
      xmax=mu+3.2*sig;
      ymin=-0.1*gauss(mu,mu,sig);
      h=0.08*gauss(mu,mu,sig);}]

      \begin{axis}[
        every axis plot post/.append style={
        mark=none, domain={xmin}:{xmax},samples=N,smooth},
        axis lines=center,
        ymin=ymin,
        axis line style={black,->},
        every axis x label/.style={at={(current axis.right of origin)},anchor=north},
        xlabel=$\hat{p}$, xlabel style={at={(axis description cs:0.5,-0.0625)},anchor=north},
        clip=false,
        ]

        % PLOTS
        \addplot[lander_blue,thick, name path=B] {gauss(x,mu,sig)};

        % FILL
        \path[name path=xaxis]
          (\pgfkeysvalueof{/pgfplots/xmin},0) -- (\pgfkeysvalueof{/pgfplots/xmax},0);
        \addplot[lander_blue!50] fill between[of=xaxis and B, soft clip={domain=0.29:{mu+3*sig}}];
        \addplot[draw=black] (0.29,0) -- (0.29,{gauss(0.29,mu,sig)});
      \end{axis}
    \end{tikzpicture}
  \end{flushright}
  \vspace*{\stretch{1}}

  \begin{ex*}
    Samuel Morse claimed that the true proportion of E's used in the English language is $0.12$. Suppose we take a sample of $876$ letters, and find a sample proportion of $0.1347$. If we took another sample, what is the probability that the new sample proportion would be greater than $0.1347$?
  \end{ex*}
  \begin{flushright}
    \begin{tikzpicture}[scale=1.0, declare function={
      N=50;
      mu=0.12; sig=0.010979;
      xmin=mu-3.2*sig;
      xmax=mu+3.2*sig;
      ymin=-0.1*gauss(mu,mu,sig);
      h=0.08*gauss(mu,mu,sig);}]

      \begin{axis}[
        scaled ticks=false,
        /pgf/number format/.cd, fixed, precision=2, %forces large ytick labels
        every axis plot post/.append style={
        mark=none, domain={xmin}:{xmax},samples=N,smooth},
        axis lines=center,
        ymin=ymin,
        axis line style={black,->},
        every axis x label/.style={at={(current axis.right of origin)},anchor=north},
        xlabel=$\hat{p}$, xlabel style={at={(axis description cs:0.5,-0.0625)},anchor=north},
        clip=false,
        ]

        % PLOTS
        \addplot[lander_blue,thick, name path=B] {gauss(x,mu,sig)};

        % FILL
        \path[name path=xaxis]
          (\pgfkeysvalueof{/pgfplots/xmin},0) -- (\pgfkeysvalueof{/pgfplots/xmax},0);
        \addplot[lander_blue!50] fill between[of=xaxis and B, soft clip={domain=0.1347:{mu+3*sig}}];
        \addplot[draw=black] (0.1347,0) -- (0.1347,{gauss(0.1347,mu,sig)});
      \end{axis}
    \end{tikzpicture}
  \end{flushright}
  \pagebreak

  \pagebreak
\end{document}
