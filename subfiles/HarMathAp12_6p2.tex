\documentclass[../mathNotesPreamble]{subfiles}

\providecommand{\relscalefact}{1.4}
\begin{document}
\relscale{\relscalefact}
  \section{6.2: Compound Interest}
    \begin{ex*}
      Suppose you invest \$1,000 at 5\% annual interest. With simple interest, you can take 2 approaches:
      \begin{enumerate}
        \item Gain interest on \emph{only} your initial investment
        \item Reinvest the interest gained
      \end{enumerate}
    \end{ex*}
    
    \noindent
    \begin{minipage}{0.4975\linewidth}
      \begin{center}
        \begin{tabular}{@{}l*{2}{M{35mm}}@{}}\toprule
          Year & Simple interest & Simple interest reinvested \\\midrule
          1 & \$1,050.00 & \$1,050.00 \\
          2 & \$1,100.00 & \$1,102.50 \\
          3 & \$1,150.00 & \$1,157.63 \\
          4 & \$1,200.00 & \$1,215.51 \\
          5 & \$1,250.00 & \$1,276.28 \\\bottomrule
        \end{tabular}
      \end{center}
      \vspace*{\baselineskip}
    \end{minipage}\hspace*{\stretch{1}}%
    \begin{minipage}{0.5\linewidth}
      \begin{center}
        \begin{tikzpicture}[scale=1.0, declare function={
          P=1000; r=0.05;
          si(\x) = 1000*((1+r*\x)-1);
          ci(\x) = 1000*((1+r)^\x-1);}]
          \begin{axis}[
            axis lines=center,
            axis line style={black,->},
            enlargelimits={value=0.025, auto},
            ticklabel style={font=\footnotesize,inner sep=0.5pt,fill=white,opacity=1.0, text opacity=1},
            xlabel=years, xlabel style={at={(axis description cs:0.5,-0.0625)},anchor=north},
            title=Interest earned,
            width=\linewidth,
            height=0.9*\axisdefaultheight,
            every axis plot/.append style={domain=xmin:xmax, line width=0.95pt, color=lander_blue, samples=255}]
            \addplot[color=lander_blue] expression[domain=0:5]{si(x)};
            \addplot[soldot, color=lander_blue] expression[domain=0:5, samples=6]{si(x)};

            \addplot[color=red] expression[domain=0:5]{ci(x)};
            \addplot[soldot, color=red] expression[domain=0:5, samples=6]{ci(x)};
          \end{axis}
        \end{tikzpicture}
      \end{center}
    \end{minipage}

    \begin{defn*}
      \textbf{Compound interest} is a method where the interest for each period is added to the principal before interest is calculated for the next period.
    \end{defn*}
    \vspace*{0.5\baselineskip}

    \begin{ex*}
      Using the example above, derive a formula for the future value of an investment compounded annually.
    \end{ex*}
    \pagebreak
    
    \begin{defn*}
      When interest is compounded multiple times a year (e.g. quarterly, monthly, etc.), the \textbf{nominal interest rate} is the interest rate \emph{per year}.\newline

      If $\$P$ is invested for $t$ years at a nominal interest rate $r$ compounded $m$ times per year, then the \textbf{total number of compounding periods} is
        \[n=mt\]
      the interest rate per compounding period (\textbf{periodic interest rate}) is
        \[i=\frac{r}{m}\]
      and the future value is
        \[S=P\parens{1+i}^n=P\parens{1+\frac{r}{m}}^{mt}\]
    \end{defn*}
    
    \begin{ex*}
      If $\$3,000$ is invested for $5$ years at $9\%$ compounded $4$ times a year, how
much interest is earned?
    \end{ex*}
    \pagebreak
    
    \noindent
    \begin{minipage}{0.74\linewidth}
      \begin{ex*}
        For the following, identify the annual interest rate, the length in years, the periodic interest rate, and the number of periods:
      \end{ex*}
      \begin{extasks}[after-item-skip=2\baselineskip](1)
        \task 12\% compounded monthly for 7 years
        \task 7.2\% compounded quarterly for 11 quarters
      \end{extasks}
      \vspace*{2\baselineskip}
    \end{minipage}\hspace*{\stretch{1}}%
    \begin{minipage}{0.225\linewidth}%\mbox{}
      \begin{tabular}{@{}lr@{}}\toprule
        Frequency& $m$\\\midrule
        Annually& 1\\
        Semi-annually& 2\\
        Quarterly& 4\\
        Monthly& 12\\
        Weekly& 52\\
        Daily& 365\\\bottomrule
      \end{tabular}
    \end{minipage}
  
  \begin{ex*}
    Ben and Taylor want to have $\$200,000$ in Arthur’s college fund on his
$18$th birthday, and they want to know the impact on this goal of
having $\$10,000$ invested at $9.8\%$, compounded quarterly, on his $1$st
birthday. To advise Ben and Taylor regarding this, find 
    \begin{extasks}[after-item-skip=\stretch{1}](1)
      \task the future value of the $\$10,000$ investment,
      \task the amount of compound interest that the investment earns,
      \task the impact this would have on their goal.
    \end{extasks}
  \end{ex*}
  \vspace*{\stretch{1}}
  \pagebreak
  
  \begin{ex*}
    What amount must be invested now to have $\$12,000$ after $3$ years with an interest rate of $6\%$, compounded semi-annually?
  \end{ex*}
  \vspace*{\stretch{1}}
  
  \begin{ex*}
    Three years after Google stock was first sold publicly, its share price had risen $500\%$. Google’s $500\%$ increase means that $\$10,000$ invested in Google stock at its initial public offering (I.P.O) was worth $\$60,000$ three years later. What interest rate compounded annually does this represent?
  \end{ex*}
  \vspace*{\stretch{1}}
  \pagebreak
  
  \begin{ex*}
    Suppose we invest $\$1$ at a $100\%$ interest rate for $1$ year:
      \[S=\parens{1+\frac{1.00}{m}}^m\]
    Compute the future value 
  \end{ex*}
  \begin{extasks}[after-item-skip=\stretch{1}](2)
    \task Annually
    \task Semi-annually
    \task Monthly
    \task Weekly
    \task Daily
    \task Each minute ($m=525,600$)
  \end{extasks}
  \vspace*{\stretch{1}}
  \pagebreak
  
  \begin{defn*}
    If $\$P$ is invested for $t$ years at a nominal rate $r$ compounded continuously, then the future value is given by the exponential function
      \[S=Pe^{rt}\]
  \end{defn*}
  
  \begin{ex*}
    Which investment strategy is worth more: $\$3,000$ for $8$ years at
      \begin{extasks}[after-item-skip=\stretch{1}](2)
        \task $9\%$, compounded annually
        \task $8\%$, compounded continuously
      \end{extasks}
  \end{ex*}
  \vspace*{\stretch{1}}
  
  \begin{ex*}
    Suppose you invest $\$900$ at $11.5\%$, compounded continuously. How long will it take to gain $\$700$ in interest?
  \end{ex*}
  \vspace*{\stretch{1}}
  \pagebreak
  \begin{defn*}
    Let $r$ represent the annual (nominal) interest rate for an investment. Then the \textbf{annual percentage yield (APY)} is:
    \begin{description}
      \item[Periodic compounding:] 
        \[\textnormal{APY }=\parens{1+\frac{r}{m}}^m-1\]
      \item[Continuous compounding:]
        \[\textnormal{APY }=e^r-1\]
    \end{description}
  \end{defn*}

  \pagebreak
\end{document}
