\documentclass[../mathNotesPreamble]{subfiles}

\providecommand{\relscalefact}{1.4}
\begin{document}
\relscale{\relscalefact}
  \section{8.1: The Essential Ingredients of Hypothesis Testing}
    \begin{defn*}
      \begin{itemize}
        \item A \textbf{hypothesis test} is a procedure that enables us to choose between two claims.
        \item The \textbf{null hypothesis}, $H_0$, represents the current belief, or status quo.
        \item The \textbf{alternative hypothesis}, $H_a$, is what we wish to test.
      \end{itemize}
      A hypothesis test has 4 steps:
      \begin{enumerate}
        \item Formulate your null and alternative hypotheses
        \item Examine or collect data
        \item Compare data to our expectations; is the result significant?
        \item Interpret the results
      \end{enumerate}
    \end{defn*}
    \begin{center}
      \begin{tabular}{@{}ccc@{}}\toprule
        Two-Sided& One-Sided (Left)& One-Sided (Right)\\\midrule
        $H_0: p=p_0$& $H_0: p=p_0$& $H_0: p=p_0$\\
        $H_a: p\neq p_0$& $H_a: p< p_0$& $H_a: p> p_0$\\\bottomrule
      \end{tabular}
    \end{center}

    \begin{ex*}
      When flipping a coin, it is considered fair if both sides of the coin have an equally likely chance of appearing face up. Suppose we have a coin that we believe might be unfair. Let $p$ be the proportion of times where heads appears face up. Formulate the null and alternative hypotheses.
    \end{ex*}
    \vspace*{\stretch{1}}
    \pagebreak

    \begin{ex*}
      Historically, about 70\% of all U.S. adults were married. A sociologist who asks whether marriage rates in the United States have declined will take a random sample of U.S. adults and record whether or not they are married.

      Write the null and alternative hypotheses.
    \end{ex*}
    \vspace*{\stretch{1}}

    \begin{ex*}
      An Internet retail business is trying to decide whether to pay a search engine company to upgrade its advertising. In the past 15\% of customers who visited the company's web page by clicking on the advertisement bought something. %If the business decides to purchase premium advertising, then the search engine company will make that company's ad more prominent.
      The search engine company offers to do an experiment: for one day a random sample of customers will see the retail business's ad in a more prominent position to try and increase the proportion of customers who make a purchase.

      Write the null and alternative hypotheses.
    \end{ex*}
    \vspace*{\stretch{1}}
    \pagebreak

    \begin{defn*}
      \begin{itemize}
        \item The \textbf{significance level}, denoted by $\alpha$, is the probability of rejecting the null hypothesis when it is actually true (false positive).
        \item A \textbf{test statistic} is similar to a $z$ score comparing the alternative hypothesis to the null hypothesis:
          \[z=\frac{\hat{p}-p_0}{SE},\ \textnormal{ where }\  SE=\sqrt{\frac{p_0(1-p_0)}{n}}\]
        \item The \textbf{$p$-value} is the probability that the null hypothesis is true. When the $p$-value is
          \begin{itemize}
            \item greater than $\alpha$, we fail to reject the null hypothesis
            \item less than or equal to $\alpha$, we reject the null hypothesis
          \end{itemize}
      \end{itemize}
      \emph{Note}: Hypothesis tests don't prove the null hypothesis!
    \end{defn*}
  \vspace*{\stretch{1}}

  \begin{center}
    \begin{tikzpicture}
      \begin{axis}[
        domain=-4:4,
        samples=255,
        axis x line=middle,
        axis y line=none,
        axis line style={black,-},
        xtick={-1.96,0,1.96},
        xticklabels={$-t^*$,$\mu_0$,$t^*$},
        ytick=\empty,
        clip=false,
        width=\linewidth, height=\axisdefaultheight
        ]
        \addplot[thick,lander_blue] {gauss(x,0,1)};
        % Shade rejection regions
        \addplot[domain=-4:-1.96, fill=red!30] {gauss(x,0,1)} \closedcycle;
        \addplot[domain=1.96:4, fill=red!30] {gauss(x,0,1)} \closedcycle;
        % Curly braces for regions
        \draw [decorate,decoration={brace,amplitude=10pt,mirror}] (axis cs:-4,-0.06) -- (axis cs:-1.98,-0.06)
          node[midway, yshift=-30pt, align=center] {$p$-value $<\alpha$ \\ Reject $H_0$};
        \draw [decorate,decoration={brace,amplitude=10pt,mirror}] (axis cs:-1.94,-0.06) -- (axis cs:1.94,-0.06)
          node[midway,yshift=-30pt, align=center] {$p$-value $>\alpha$ \\ Fail to reject $H_0$};
        \draw [decorate,decoration={brace,amplitude=10pt,mirror}] (axis cs:1.98,-0.06) -- (axis cs:4,-0.06)
          node[midway,yshift=-30pt, align=center] {$p$-value $<\alpha$ \\ Reject $H_0$};
      \end{axis}
    \end{tikzpicture}
  \end{center}

  \pagebreak
\end{document}
