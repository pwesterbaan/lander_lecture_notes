\documentclass[../mathNotesPreamble]{subfiles}

\begin{document}
%  \relscale{1.4} %TODO
  \section{8.2: Hypothesis Testing in Four Steps}
  \fbox{\parbox{0.9875\linewidth}{
    \begin{enumerate}
      \item \textbf{Hypothesize:} formulate your hypotheses
      \item \textbf{Check conditions:}
        \begin{itemize}
          \item \textbf{Random and Independent:} The sample must be randomly collected from the population, and observations are independent of each other
          \item \textbf{Large Sample:} The sample size must be large enough for at least 10 successes, $np_0\geq 10$, and 10 failures, $n(1-p_0)\geq 10$.
          \item \textbf{Large Population:} If the sample is collected without replacement, the population of size $N$ must be at least 10 times bigger than the sample: $N\geq 10 n$
%          \item \textbf{Independence:} Each observation or measurement must have no influence on any others.
        \end{itemize}
        If these conditions are met, we compute the test statistic for the One-Proportion $z$-Test which follows a $z$-distribution:
          \[z=\frac{\hat{p}-p_0}{SE}, \qquad \textnormal{where} \qquad SE=\sqrt{\frac{p_0(1-p_0)}{n}}\]
      \item \textbf{Compute:} Stating a significance level, compute the observed test statistic $z$ and/or $p$-value.
      \item \textbf{Interpret:} Decide whether to reject or fail to reject the null hypothesis.
    \end{enumerate}
  }}
  \pagebreak

  \begin{ex*}
    Unlike flipping a coin, spinning a coin leads to a biased outcome. Suppose we spun a coin 60 times, and saw a sample proportion of $\hat{p}=0.35$.
  \end{ex*}
  \begin{extasks}[after-item-skip=\stretch{1}](1)
    \task Formulate the null and alternative hypotheses
    \task Check the conditions required to perform a hypothesis test.
    \task Find the test statistic and $p$-value
    \task Using a significance level of $\alpha=0.05$, decide whether to reject or fail to reject the null hypothesis.
  \end{extasks}
  \vspace*{\stretch{1}}
  \pagebreak

  \begin{ex*}
    A group of medical researchers knew from pervious studies that in the past, about 39\% of all men between the ages of 45 and 59 were regularly active. Researchers were concerned that this percentage had declined over time. For this reason, they did selected a random sample, without replacement, of 1927 men in this age group and interviewed them. Out of this sample, 680 said they were regularly active.
  \end{ex*}
  \begin{extasks}[after-item-skip=\stretch{1}](1)
    \task Formulate the null and alternative hypotheses
    \task Check the conditions required to perform a hypothesis test.
    \task Find the test statistic and $p$-value
    \task Using a significance level of $\alpha=0.05$, decide whether to reject or fail to reject the null hypothesis.
  \end{extasks}
  \vspace*{\stretch{1}}
  \pagebreak

  \pagebreak
\end{document}
