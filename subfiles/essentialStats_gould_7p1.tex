\documentclass[../mathNotesPreamble]{subfiles}

\providecommand{\relscalefact}{1.4}
\begin{document}
\relscale{\relscalefact}
  \section{7.1: Learning about the World through Surveys}

    \begin{defn*}
      A \textbf{population} is a group of objects or people we wish to study.
      \begin{itemize}
        \item A \textbf{parameter} is a numerical value describing some aspect of the population\\ (e.g. means and proportions)
        \item A \textbf{census} is a survey of \emph{every member} of a population
      \end{itemize}
      A \textbf{sample} is a subset of the population of interest.
      \begin{itemize}
        \item A \textbf{statistic} is a numerical value describing some aspect of the sample
        \item Statistics are sometimes called \textbf{estimators}
        \item A \textbf{statistical inference} is the science of drawing conclusions about a population based on observing only a small subset of that population.
      \end{itemize}
    \end{defn*}

    \begin{ex*}
      In February 2014, the Pew Research Center surveyed 1428 cell phone users in the United States who were married or in a committed partnership. The survey found that 25\% of cell phone owners felt that their spouse or partner was distracted by their cell phone when they were together.
    \end{ex*}

    \begin{extasks}[after-item-skip=\stretch{1}](1)
      \task Identify the population and the sample
      \task Identify the parameter and the statistic
    \end{extasks}
    \vspace*{\stretch{1}}
    \pagebreak

    \noindent
    Sample statistics and population parameters are represented using different symbols\\ (English for statistics, Greek for population):
    \begin{center}
      \begin{tabular}{@{}l*{2}{c}@{}}\toprule
        & Statistics & Parameters \\\midrule
        mean & $\overline{x}$ & $\mu$ \\
        standard deviation & $s$ & $\sigma$\\
        proportion & $\hat{p}$ & $p$\\\bottomrule
      \end{tabular}
    \end{center}

  \begin{ex*}
    The City of Los Angeles provides an open data set of response times for emergency vehicles. Each row of the data set represents an emergency vehicle that has been sent to a particular emergency. A random sample of $1000$ of these rows shows that the mean response time was $8.25$ minutes. In addition, the proportion of vehicles that were ambulances was $0.328$.
  \end{ex*}
  \begin{extasks}[after-item-skip=\stretch{1}](1)
    \task Using correct notation, identify the data given above.
    \task What can we conclude about the overall population?
  \end{extasks}
  \vspace*{\stretch{1}}
  \pagebreak

  \begin{defn*}
    A method is \textbf{biased} if it tends to produce the wrong value.
    \begin{itemize}
      \item \textbf{Sampling bias} results from a sample that is not representative of the population.
      \item \textbf{Measurement bias} results from questions that do not produce a true answer.
    \end{itemize}
  \end{defn*}
  \begin{ex*}
    For the following scenarios, identify any bias:
  \end{ex*}
  \begin{extasks}[after-item-skip=\stretch{1}](1)
    \task Online reviews (Amazon, Yelp, etc.)
    \task Asking if people support a `fat tax' on non-diet sugary soft drinks
    \task Gallup poll calling landline phones
    \task Using a poorly written question (e.g. double negative)
  \end{extasks}
  \vspace*{\stretch{1}}
  \pagebreak

  \begin{defn*}
    A \textbf{simple random sample (SRS)} is where subjects from a population are drawn \emph{at random} and \emph{without replacement}. With an SRS, each member of the population has an equally likely chance of being selected.

    \textbf{Nonresponse bias} results from people refusing to respond to the survey.
  \end{defn*}
  \begin{ex*}
    Perform an SRS of $3$ people from the list below:
    \begin{center}
      \begin{tabular}{@{}ll@{}}\toprule
        1& Alberto\\ 2& Justin\\3& Michael\\4& Audrey\\5& Brandy\\6& Nicole\\\bottomrule
      \end{tabular}
    \end{center}
  \end{ex*}
  \pagebreak

  \pagebreak
\end{document}
