\documentclass[../mathNotesPreamble]{subfiles}

\providecommand{\relscalefact}{1.4}
\begin{document}
\relscale{\relscalefact}
  \section{6.3: Future Values of Annuities}
    \begin{defn*}
      \begin{itemize}
        \item An \textbf{annuity} is a financial plan characterized by regular payments (e.g. mortgages, student loans, etc.).
        \item The sum of all the payments and the interest earned is called the \textbf{future value of the annuity} or its \textbf{future value}.
        \item An \textbf{ordinary annuity} or (\textbf{annuity immediate}) is an annuity in which payments are made at the \emph{end of each of the equal payment intervals}.
      \end{itemize}

    \end{defn*}
    \begin{ex*}
      Suppose that we invest $\$100$ at the end of each year for $5$ years in an account that pays $10\%$ compounded annually. How much money will you have at the end of the $5$ years?
    \end{ex*}
    \vspace*{\stretch{1}}

    \begin{defn*}
      If $\$R$ is deposited at the \emph{end of each period} for $n$ periods in an annuity that earns interest at a rate of $i$ per period, the \textbf{future value of the annuity} will be
        \[S=R\cdot S_{\angln i}=R\sbrkt{\frac{(1+i)^\nmo}{i}}\]
      The notation $S_{\angln i}$ represents the future value of an ordinary annuity of $\$1$ per period for $n$ periods with an interest rate of $i$ per period.
    \end{defn*}
    \pagebreak

    \begin{ex*}
      %twins example
      Suppose a pair of twins take different steps to save for retirement. Both regularly make investments of $\$2,000$ into accounts that earn $10\%$, compounded annually. Starting at age 21:
    \end{ex*}
    \begin{extasks}[after-item-skip=\stretch{1}](1)
      \task Find the future value if twin A makes his payments for 8 years, and then lets his investment accrue compound interest every year for 36 years.
      \task Find the future value if twin B waits 8 years before making regular payments for the following 36 years.
    \end{extasks}
    \vspace*{\stretch{1}}
    \pagebreak

    \begin{ex*}
      Suppose that you wish to have $\$50,000$ saved up in $5$ years. To do this, you want to make regularly monthly payments. What is the amount of the monthly payments if the interest rate is $5\%$? What if the interest rate is $15\%$?
    \end{ex*}
    \vspace*{\stretch{1}}

    \begin{ex*}
      A small business invests $\$1,000$ at the end of each month in an account that earns $6\%$ compounded monthly. How long will it take until the business has $\$100,000$ toward the purchase of its own office building?
    \end{ex*}
    \vspace*{\stretch{1}}
    \pagebreak

    \begin{defn*}
      An \textbf{annuity due} differs from an ordinary annuity in that the payments are made at the \emph{beginning of each period}. \newline

      If $\$R$ is deposited at the \emph{beginning of each period} for $n$ periods in an annuity that earns interest at a rate of $i$ per period, the \textbf{future value of the annuity} will be
        \[S_{\textnormal{due}}=R\cdot S_{\angln i}(1+i)=R\sbrkt{\frac{(1+i)^\nmo}{i}}\parens{1+i}\]
    \end{defn*}
    \begin{ex*}
      Find the future value of an investment if $\$150$ is deposited at the beginning of each month for $9$ years at an interest rate of $7.2\%$ compounded monthly.
    \end{ex*}

  \pagebreak
\end{document}
