\documentclass[../mathNotesPreamble]{subfiles}

\begin{document}
 % \relscale{1.4} %TODO
 \section{2.4: Limits}
  \begin{ex*}
    Suppose that the position function of a maglev train (in feet) is given by
    \begin{align*}
      s(t)=4t^2,\qquad (0\leq t\leq 30)
    \end{align*}
    Using the position function, compute the \emph{average} velocity of the train
  \end{ex*}
  \vspace*{6\baselineskip}
  \begin{extasks}[after-item-skip=2\baselineskip](1)
    \task on the interval $[t,2]$
      \begin{center}
        \begin{tabularx}{0.9\linewidth}{@{}*{6}{X}@{}}\toprule
          $t$ & 1.5& 1.9& 1.99& 1.999& 1.9999\\\midrule
          \\[\baselineskip]\bottomrule
        \end{tabularx}
      \end{center}
    \task on the interval $[2,t]$
      \begin{center}
        \begin{tabularx}{0.9\linewidth}{@{}*{6}{X}@{}}\toprule
          $t$ & 2.5& 2.1& 2.01& 2.001& 2.0001\\\midrule
          \\[\baselineskip]\bottomrule
        \end{tabularx}
      \end{center}
  \end{extasks}
  What do the tables above suggest about \emph{instantaneous} velocity of the train at $t=2$?
  \pagebreak

  \begin{defn*}[Limit of a Function]
    The function $f$ has the \textbf{limit} $L$ as $x$ approaches $a$, written
    \begin{align*}
      \lim_{x\to a} f(x)=L
    \end{align*}
    if the value of $f(x)$ can be made as close to the number $L$ as we please by taking $x$ sufficiently close to (but not equal to) $a$.
  \end{defn*}

  \begin{ex*}
    Using the graph of $f$, determine the following values:
    \vspace*{-\baselineskip}
  \end{ex*}
  \begin{flushright}
    \smash{\raisebox{-\height}{
      \begin{tikzpicture}
        \begin{axis}[
          grid=both,
          grid style={line width=0.35pt, draw=gray!75},
          axis lines=center,
          axis line style={->},
          xmin=-1, xmax=6.5,
          ymin=-1, ymax=6.5,
          xtick={0,1,...,6},
          ytick={0,1,...,6},
          ticklabel style={font=\footnotesize,inner sep=0.5pt,fill=white,opacity=1.0, text opacity=1},
          xlabel=$x$, xlabel style={at={(ticklabel* cs:1)},anchor=north west},
          ylabel=$f(x)$, ylabel style={at={(ticklabel* cs:1)},anchor=south west},
          every axis plot/.append style={line width=0.95pt, color=lander_blue, samples=255}
          ]
          \addplot[<->] expression[domain=0:6.25] {(x-2)/abs(x-2)*abs(x-2)^(1/3)+3};
          \addplot[holdot] coordinates{(2,3)(3,4)};
          \addplot[soldot] coordinates{(2,5)};
        \end{axis}
      \end{tikzpicture}
    }}
  \end{flushright}

  \noindent
  \begin{extasks}[after-item-skip=\stretch{1}](1)
    \task $f(1)$ and $\ds\lim_{x\to 1}f(x)$
    \task $f(2)$ and $\ds\lim_{x\to 2}f(x)$
    \task $f(3)$ and $\ds\lim_{x\to 3}f(x)$
  \end{extasks}
  \vspace*{\stretch{1}}
  \pagebreak

  \begin{ex*}
    Find the limit of the following functions at the value specified: \hfill \href[pdfnewwindow]{https://www.desmos.com/calculator/tqrhikiclr}{\textcolor{blue}{\underline{Graphs}}}
  \end{ex*}
  \TabPositions{57mm}
  \begin{extasks}[after-item-skip=\stretch{1}](2)
    \task $f(x)=x^3 \quad$ at $x=2$
    \task $g(x)=\begin{cases}
      x+2, & x\neq 1\\
      1,& x=1
    \end{cases} \quad$  at $x=1$
    \task $h(x)=\begin{cases}
      -1,& x<0\\
      1,& x\geq 0
    \end{cases} \quad$  at $x=0$
    \task $j(x)=\dfrac{1}{(x-1)^2} \quad$ at $x=1$
    \task $k(x)=4 \quad$  at $x=0$
  \end{extasks}
  \vspace*{\stretch{1}}
  \pagebreak

  \begin{thmBox*}[Theorem 1: Properties of Limits]
    Suppose
    \begin{align*}
      \lim_{x\to a} f(x)=L\quad \textnormal{ and }\quad \lim_{x\to a} g(x)=M
    \end{align*}
    Then
    \TabPositions{80mm}
    \begin{enumerate}
      \item $\displaystyle\lim_{x\to a} \sbrkt{f(x)}^r=\sbrkt{\lim_{x\to a} f(x)}^r$ \tab where $r$ is a positive constant
      \item $\displaystyle \lim_{x\to a} cf(x)=c \lim_{x\to a} f(x)$ \tab where $c$ is a real number
      \item $\displaystyle \lim_{x\to a}\sbrkt{f(x)\pm g(x)}=\lim_{x\to a} f(x)\pm \lim_{x\to a} g(x)=L\pm M$
      \item $\displaystyle \lim_{x\to a} \sbrkt{f(x)g(x)}=\sbrkt{\lim_{x\to a} f(x)}\sbrkt{\lim_{x\to a} g(x)}=LM$
      \item $\displaystyle \lim_{x\to a} \frac{f(x)}{g(x)}=\frac{\displaystyle\lim_{x\to a}f(x)}{\displaystyle\lim_{x\to a}g(x)}=\frac{L}{M}$ \tab provided $M\neq 0$
    \end{enumerate}
  \end{thmBox*}
  \begin{ex*}
    Use the above theorem to evaluate the following limits:
  \end{ex*}
  \begin{extasks}[after-item-skip=\stretch{1}](1)
%    \task $\displaystyle\lim_{x\to 2} x^3$
%    \task $\displaystyle\lim_{x\to 4} 5x^{3/2}$
    \task $\displaystyle\lim_{x\to 1} \parens{5x^{3/2}-2}$
    \task $\displaystyle\lim_{x\to 3} \dfrac{2x^3 \sqrt{x^2+7}}{x+1}$
%    \task
  \end{extasks}
  \vspace*{\stretch{1}}
  \pagebreak

  \noindent
  Suppose that $\displaystyle \lim_{x\to a} f(x)=0$ and $\displaystyle \lim_{x\to a} g(x)=0$. Then
  \begin{align*}
    \lim_{x\to a} \dfrac{f(x)}{g(x)}
  \end{align*}
  has an \textbf{indeterminate form} of $\dfrac{0}{0}$. To evaluate such a limit, we replace the given function with a function that's equivalent everywhere except at $x=a$, and then evaluate the limit.
  \begin{ex*}
    Evaluate the following
  \end{ex*}
  \begin{extasks}[after-item-skip=\stretch{1}](1)
    \task $\displaystyle \lim_{t\to 2} \dfrac{4t^2-16}{t-2}$
    \task $\displaystyle \lim_{h\to 0} \dfrac{\sqrt{4+h}-2}{h}$
  \end{extasks}
  \vspace*{\stretch{1}}
  \pagebreak

  \noindent
  Suppose that $\displaystyle \lim_{x\to a} f(x)=L$ with $L\neq 0$ and $\displaystyle \lim_{x\to a} g(x)=0$. Then
  \begin{align*}
    \lim_{x\to a} \dfrac{f(x)}{g(x)}
  \end{align*}
  does not exist. We can further specify if this limit tends towards
  $-\infty$ or $\infty$.

  \begin{ex*}
    Evaluate the following \hfill \href[pdfnewwindow]{https://www.desmos.com/calculator/svhfldwwmo}{\textcolor{blue}{\underline{Graphs}}}
  \end{ex*}
  \begin{extasks}[after-item-skip=\stretch{1}](2)
    \task $\displaystyle \lim_{x\to 1} \dfrac{x}{x-1}$
    \task $\displaystyle \lim_{x\to 3} \dfrac{1}{(x-3)^2}$
    \task $\displaystyle \lim_{\mathllap{x}\to -2} \dfrac{x-2}{x^2-4}$
    \task $\displaystyle \lim_{x\to 2} \dfrac{x-2}{x^2-4}$
  \end{extasks}
  \vspace{\stretch{1}}
  \pagebreak

  \begin{thmBox*}[Limit of a Function at Infinity]
    The function $f$ has the limit $L$ as $x$ increases without bound, written
    \begin{align*}
      \lim_{x\to \infty} f(x)=L
    \end{align*}
    if $f(x)$ can be made arbitrarily close to $L$ by taking $x$ large enough.

    The function $f$ has the limit $M$ as $x$ decreases without bound, written
    \begin{align*}
      \lim_{x\to -\infty} f(x)=M
    \end{align*}
    if $f(x)$ can be made arbitrarily close to $M$ by taking $x$ to be negative and sufficiently large enough in absolute value.

    When the above limits exist, the equations $y=L$ and/or $y=M$ are called \textbf{horizontal asymptotes}.
  \end{thmBox*}
  \begin{ex*}
    Evaluate the following infinite limits
  \end{ex*}
  \begin{extasks}[after-item-skip=\stretch{1}](1)
    \task $\displaystyle \lim_{x\to \infty} \dfrac{2x^2+3x-4}{x^2-7x+1}$
    \task $\displaystyle \lim_{x\to -\infty} \dfrac{3x^2+4}{2x^3}$
    \task $\displaystyle \lim_{x\to \pm\infty} \dfrac{3x^5+2x^3-4}{x^4+4x^2-1}$
  \end{extasks}
  \vspace*{\stretch{1}}
  \pagebreak

  % \smash{\raisebox{-\height}{\href[pdfnewwindow]{https://www.desmos.com/calculator/izlmkakqnp}{\textcolor{blue}{\underline{Graph}}}}}
  \begin{center}
    \href[pdfnewwindow]{https://www.desmos.com/calculator/izlmkakqnp}{\textcolor{blue}{\underline{Graph}}}

    \begin{tikzpicture}[declare function={
      f(\x)=(7*\x^3-2)/(-x^3+sqrt(25*x^6+4));}]
      \begin{axis}[
        grid=both,
        grid style={line width=0.35pt, draw=gray!75},
        axis lines=center,
        axis line style={->},
        % xmin=-6, xmax=6,
        ymin=-2.5, ymax=2.5,
        enlargelimits={value=0.05, auto},
        ticklabel style={font=\footnotesize,inner sep=0.5pt,fill=white,opacity=1.0, text opacity=1},
        width=0.95\linewidth,
        height=\axisdefaultheight,
        % xlabel=$x$, xlabel style={at={(ticklabel* cs:1)},anchor=north west},
        % ylabel= $f(x)$, ylabel style={align=left, at={(ticklabel* cs:1)},anchor=south west},
        every axis plot/.append style={line width=0.95pt, color=lander_blue, samples=255}
        ]
        \addplot[<->] expression[domain=-6:6] {f(x)};
      \end{axis}
    \end{tikzpicture}
  \end{center}
  \begin{extasks}[after-item-skip=\stretch{1}](2)
    \task $\displaystyle \lim_{x\to -\infty} \frac{7x^3-2}{-x^3+\sqrt{25x^6-4}}$
    \task $\displaystyle \lim_{x\to \infty} \frac{7x^3-2}{-x^3+\sqrt{25x^6-4}}$
  \end{extasks}
  \vspace*{\stretch{1}}
  \pagebreak

  \begin{ex*}
    The company \emph{Custom Office} makes a line of executive desks. It is estimated that the total cost of making $x$ \emph{Senior Executive Model} desks is
    \begin{align*}
      C(x)=100x+200,000
    \end{align*}
    dollars per year. The average cost of making $x$ desks is given by
    \begin{align*}
      \overline{C}(x)=\frac{C(x)}{x}
    \end{align*}
    Compute $\displaystyle \lim_{x\to \infty} \overline{C}(x)$ and interpret the result.
  \end{ex*}
  \vspace*{\stretch{1}}
  \begin{thmBox*}[Theorem 2]
    For all $n>0$,
    \begin{align*}
      \lim_{x\to \pm\infty} \frac{1}{x^n}=0
    \end{align*}
    provided that $\dfrac{1}{x^n}$ is defined.
  \end{thmBox*}

  \pagebreak
\end{document}
