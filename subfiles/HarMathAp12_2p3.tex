\documentclass[../mathNotesPreamble]{subfiles}

\begin{document}
  % \relscale{1.4} %TODO
  \section{2.3: Business Applications Using Quadratics}
  Recall the following:
  \begin{defn*}
    \textbf{Profit} is the difference between the revenue and total cost:
      \[P(x)=R(x)-C(x)\]
    where
      \begin{align*}
        P(x)&= \textnormal{profit from sale of $x$ units,}\\
        R(x)&= \textnormal{total revenue from sale of $x$ units,}\\
        C(x)&= \textnormal{total cost from production and sale of $x$ units.}
      \end{align*}
%    \emph{Note}: In general, the symbols used in economics are $\pi$, $TR$ and $TC$ respectively.

    \vspace*{\baselineskip}
    In general, \textbf{total revenue} is
      \[\textnormal{Revenue}=\parens{\textnormal{price per unit}}\parens{\textnormal{number of units}}\]
    The \textbf{total cost} is composed of fixed cost and variable cost:
    \begin{itemize}
      \item \textbf{Fixed costs} $(FC)$ remain constant regardless of the number of units produced.
      \item \textbf{Variable costs} $(VC)$ are directly related to the number of units produced.
    \end{itemize}
    The total cost is given by
      \[\textnormal{Cost}=\textnormal{variable costs}+\textnormal{fixed costs}\]
  \end{defn*}
  \pagebreak

  \begin{ex*}
    Suppose that a company's cost include a fixed cost of $\$1,200$, and a variable cost per unit of $\frac{x}{4}+18$ dollars, where $x$ is the total number of units produced. If the selling price of their product is $(156-\frac{3x}{4})$ dollars per unit, then
  \end{ex*}
  \begin{extasks}[after-item-skip=\stretch{1}](1)
    \task How many units should be sold to maximize the revenue?
    \task Find the profit function.
    \task How many units should be sold to maximize the profit?
    \task Find the \textbf{break-even point} (e.g. where $R(x)=C(x)$ and $P(x)=0$).
  \end{extasks}
  \vspace*{\stretch{1}}

  \begin{center}
  % https://www.desmos.com/calculator/ig3ctztdys
    \begin{tikzpicture}[declare function={
      ro=10; rt=120; c=rt+ro; mp=c/2; s=0.2; t=0.25;
      R(\x)=((1+s)*c-(1-t)*\x)*\x;
      C(\x)=ro*rt+\x*(t*\x+s*c);
      P(\x)=R(\x)-C(\x);
      Rvx=((1+s)*c)/(2*(1-t)); Rvy=R(Rvx);},
      every node/.append style={black, align=left, font=\small}]
      \begin{groupplot}[
        group style={group size=2 by 1, horizontal sep=30mm},
        axis lines=center,
        axis line style={black,->},
        xmin=0,
        ymin=-500,
        width=0.425\linewidth,
        ticklabel style={font=\normalsize,inner sep=0.5pt,fill=white,opacity=0.0, text opacity=1},
        xlabel=$x$, xlabel style={at={(ticklabel* cs:1)},anchor=west, xshift=5pt},
        ylabel=$y$, ylabel style={at={(ticklabel* cs:1)},anchor=south east},
        every axis plot/.append style={line width=0.95pt, color=lander_blue, samples=255},
        ]
        \nextgroupplot
          \addplot[<-, name path=R] expression[domain=0:ro]{R(x)};
          \addplot[<-, name path=C] expression[domain=0:ro]{C(x)};
          \addplot[red!50,opacity=0.6] fill between[of=R and C];
          \addplot[-, name path=R] expression[domain=ro:rt]{R(x)}
            node[left, pos=0.65, xshift=-2pt] {$R(x)$};
          \addplot[-, name path=C] expression[domain=ro:rt]{C(x)}
            node[right, pos=0.5, xshift=2pt] {$C(x)$};
          \addplot[lander_blue!50,opacity=0.6] fill between[of=R and C];
          \node at (mp,{0.5*R(mp)+0.5*C(mp)}) {Profit\\ region};

          \addplot[->, name path=R] expression[domain=rt:c]{R(x)};
          \addplot[->, name path=C] expression[domain=rt:c]{C(x)};
          \addplot[red!50,opacity=0.6] fill between[of=R and C];

          \addplot[soldot, black] coordinates{(Rvx,Rvy)};
          \draw[dashed, black!75] (Rvx,0) -- (Rvx,Rvy);

        \nextgroupplot[ymax={1.05*P(mp)}]
          \addplot[draw=none, name path=Z] expression[domain=0:ro]{0};
          \addplot[<-, name path=P] expression[domain=0:ro]{P(x)};
          \addplot[red!50,opacity=0.6] fill between[of=P and Z];

          \addplot[draw=none, name path=Z] expression[domain=rt:c]{0};
          \addplot[->, name path=P] expression[domain=rt:c]{P(x)};
          \addplot[red!50,opacity=0.6] fill between[of=P and Z];

          \addplot[draw=none, name path=Z] expression[domain=ro:rt]{0};
          \addplot[-, name path=P] expression[domain=ro:rt]{P(x)}
            node[above left, pos=0.4] {$P(x)$};
          \addplot[lander_blue!50,opacity=0.6] fill between[of=P and Z];
          \node at (mp,{0.5*P(mp)}) {Profit\\ region};
      \end{groupplot}
    \end{tikzpicture}
  \end{center}
  \pagebreak

  \begin{ex*}
    Suppose that the demand function for a commodity is given by the equation
      \[p^2+7q=1900,\]
    and the supply function is given by the equation
      \[400-p^2+3q=0.\]
    Find the \textbf{market equilibrium}
  \end{ex*}
  \vspace*{\stretch{1}}
  \begin{flushright}
    \begin{tikzpicture}[declare function={
      po(\q)=sqrt(1900-7*\q);
      pt(\q)=sqrt(400+3*\q);}]
      \begin{axis}[
        axis lines=center,
        axis line style={black,->},
        width=0.425\linewidth,
        enlargelimits={value=0.05, auto},
        ticklabel style={font=\normalsize,inner sep=0.5pt,fill=white,opacity=0.0, text opacity=1},
        xlabel=$q$, xlabel style={at={(ticklabel* cs:1)},anchor=west, xshift=5pt},
        ylabel=$p$, ylabel style={at={(ticklabel* cs:1)},anchor=south east},
        every axis plot/.append style={line width=0.95pt, color=lander_blue, samples=255},
        clip=false,
        ]
        \addplot[<->] expression[domain=0:275]{po(x)}
          node[pos=0.1, right, black, yshift=7.5pt] {$p^2+7q=1900$};
        \addplot[<->] expression[domain=0:275]{pt(x)}
          node[pos=0.05, below right, black, xshift=-5pt] {$400-p^2+3q=0$};
        \addplot[soldot] coordinates{(150,29.15476)};
        \draw[dashed, black] (150,0) |- (0,29.15746);
      \end{axis}
    \end{tikzpicture}
  \end{flushright}
  \pagebreak

  \begin{ex*}
    If the supply and demand functions for a commodity are given by $p-q=10$ and $q(2p-10)=2100$, what is the equilibrium price and what is the corresponding number of units supplied and demanded?
  \end{ex*}
  \vspace*{\stretch{1}}
  \begin{flushright}
    \begin{tikzpicture}[declare function={
      qo(\p)=(\p-10);
      qt(\p)=2100/(2*\p-10);}]
      \begin{axis}[
        axis lines=center,
        axis line style={black,->},
        width=0.425\linewidth,
        enlargelimits={value=0.05, auto},
        ticklabel style={font=\normalsize,inner sep=0.5pt,fill=white,opacity=0.0, text opacity=1},
        xlabel=$x$, xlabel style={at={(ticklabel* cs:1)},anchor=west, xshift=5pt},
        ylabel=$y$, ylabel style={at={(ticklabel* cs:1)},anchor=south east},
        every axis plot/.append style={line width=0.95pt, color=lander_blue, samples=255},
        ]
        \addplot[<->] expression[domain=10:50]{qo(x)}
          node[pos=0.5, above left, black, yshift=-7.5pt] {$p-q=10$};
        \addplot[<->] expression[domain=20:50]{qt(x)}
          node[pos=0.1, right, black] {$q(2p-10)=2100$};
        \addplot[soldot] coordinates{(40,30)};
        \draw[dashed, black] (40,0) |- (0,30);
      \end{axis}
    \end{tikzpicture}
  \end{flushright}
  \pagebreak

  \pagebreak
\end{document}
