\documentclass[../mathNotesPreamble]{subfiles}

\providecommand{\relscalefact}{1.4}
\begin{document}
\relscale{\relscalefact}
  \section{7.4: Estimating the Population Proportion with Confidence Intervals}
    \begin{defn*}
      Suppose that we wish to estimate a population proportion $p$ based on a sample proportion $\hat{p}$.
      \begin{itemize}
        \item A \textbf{confidence interval} is an interval about the point estimate $\hat{p}$ that we can be confident contains the true population proportion $p$:
          \[\hat{p}\pm m\]
        \item The \textbf{margin of error (ME)} is half the width of the confidence interval. When estimating a population proportion, the margin of error is
          \[m=z^* SE\]
        \item The \textbf{confidence level} measures how often the estimation method is successful. A larger confidence level results in a larger margin of error.
      \end{itemize}
    \end{defn*}
    \vspace*{\stretch{1}}

    Recall the standard error (SE) for population proportions is
      \[SE=\sqrt{\frac{p(1-p)}{n}}\]
    \vspace*{\stretch{1}}

    \noindent
    Some common values for the margin of error:
    \begin{center}
      \begin{tabular}{@{}rr@{}}\toprule
        Confidence Level& Margin of Error\\\midrule
        99.7\%& $3.0\cdot SE$\\
        99\%& $2.58\cdot SE$\\
        95\%& $1.96\cdot SE$\\
        90\%& $1.645\cdot SE$\\
        80\%& $1.28\cdot SE$\\\bottomrule
      \end{tabular}
    \end{center}
    \vspace*{\stretch{1}}
    \pagebreak

    \begin{ex*}
      In 2018, Gallup took a poll of 497 randomly selected adults who teach K--12 students and 42\% of them said that digital devices (smartphones, tablets, computers) had ``mostly helpful'' effects on students' education.
    \end{ex*}
    \begin{extasks}[after-item-skip=\stretch{1}](1)
      \task Check that the conditions of the CLT apply.
      \task Estimate the standard error.
      \task Give the 95\% confidence interval and interpret the result.
    \end{extasks}
    \vspace*{\stretch{1}}
    \pagebreak

    \begin{ex*}
      After the Great Recession, the Pew Research Center noted there seemed to be a decline in households that rented their homes and were looking to purchase homes. However, Pew reported that in 2016 ``a solid 72\%'' of renters reported that they wished to buy their own home. Pew reports that the ``margin of error at 95\% confidence level is plus-or-minus 5.4 points.''

      State the confidence interval in interval form and interpret the result.
    \end{ex*}
    \pagebreak

  \pagebreak
\end{document}