\documentclass[../mathNotesPreamble]{subfiles}

\usepackage{slashed}
\DeclareMathOperator{\myR}{R}

\providecommand{\relscalefact}{1.4}
\begin{document}
\relscale{\relscalefact}
  \section{1.3: The Language of Relations and Functions}

  \begin{defn*}
    Let $A$ and $B$ be sets. A \textbf{relation $R$ from $A$ to $B$} is a subset of $A\times B$. Given an ordered pair $(x,y)$, \textbf{$x$ is related to $y$ by $R$}, written $\rel{x}{y}$, if, and only if, $(x,y)$ is in $R$. The set $A$ is called the \textbf{domain} of $R$ and the set $B$ is called its \textbf{co-domain}. 
    
    The notation for a relation $R$ may be written symbolically as follows:
      \[\rel{x}{y} \textnormal{ means that } (x,y)\in R.\]
    The notation $\notrel{x}{y}$ means that $x$ is not related to $y$ by $R$:
      \[\notrel{x}{y} \textnormal{ means that } (x,y)\not\in R.\]
  \end{defn*}
  \begin{ex*}
    Let $A=\set{1,2}$ and $B=\set{1,2,3}$ and define a relation $R$ from $A$ to $B$ as follows; Given any $(x,y)\in A\times B$,
      \[(x,y)\in R \textnormal{ means that } \dfrac{x-y}{2} \textnormal{ is an integer.}\]
  \end{ex*}
  \begin{extasks}[after-item-skip=\stretch{1}](3)
    \task* State explicitly which ordered pairs are in $A\times B$ and which are in $R$
    \task Is $\rel{1}{3}$? 
    \task Is $\rel{2}{3}$? 
    \task Is $\rel{2}{2}$?
    \task* What are the domain and co-domain of $R$?
  \end{extasks}
  \vspace*{\stretch{1}}
  \pagebreak

  \begin{ex*}
    Define a realtion $C$ from $\bbr$ to $\bbr$ as follows: For any $(x,y)\in\bbr\times\bbr$,
      \[(x,y)\in C \textnormal{ means that } x^2+y^2=1.\]
  \end{ex*}
  \begin{extasks}[after-item-skip=\stretch{1}](3)
    \task Is $(1,0)\in C$?
    \task Is $(0,0)\in C$?
    \task Is $\parens{-\frac{1}{2},\frac{\sqrt{3}}{2}}\in C$?
    \task Is $\rel[C]{-2}{0}?$
    \task Is $\rel[C]{0}{(-1)}?$
    \task Is $\rel[C]{1}{1}?$
    \task* What are the domain and co-domain of $C$?
    \task* Draw a graph for $C$ by plotting the points of $C$ in the Cartesian plane.
  \end{extasks}
  \vspace*{\stretch{1}}
  \pagebreak

  \begin{defn*}
    Suppose $R$ is a relation from set $A$ to a set $B$. The \textbf{arrow diagram for $R$} is obtained as follows:
    \begin{enumerate}
      \item Represent the elements of $A$ as points in one region and the elements of $B$ as points in another region.
      \item For each $x$ in $A$ and $y$ in $B$, draw an arrow from $x$ to $y$ if, and only if, $x$ is related to $y$ by $R$.
    \end{enumerate}
    \begin{center}
      \begin{tikzpicture}
        \draw (-2,0) circle [x radius=1, y radius=2];
        \node at (-2,2.5) {Domain};
        \draw (2,0) circle [x radius=1, y radius=1.75];
        \node at (2,2.25) {Range};
        \draw[fill=black] (xo) at (-2,1) circle[radius=0.05] node[left] {$x_1$};
        \draw[fill=black] (xt) at (-2,0) circle[radius=0.05] node[left] {$x_2$};
        \draw[fill=black] (xr) at (-2,-1) circle[radius=0.05] node[left] {$x_3$};
        \draw[fill=black] (fxo) at (2,0.5) circle[radius=0.05] node[right] {$y_1$};
        \draw[fill=black] (fxt) at (2,-0.5) circle[radius=0.05] node[right] {$y_2$};
%        \node[circle, fill, inner sep=1pt] (fxo) at (2,0.5) {$y_1$};
%        \node[circle, fill, inner sep=1pt] (xt) at (-2,0) {$x_2$};
%        \node[circle, fill, inner sep=1pt] (xr) at (-2,-1) {$y_1$};
%        \node[circle, fill, inner sep=1pt] (fxt) at (2,-0.5) {$x_3$};
%        \node at (0,1.6) {$f$};
        \draw[->, shorten >=5pt, shorten <=5pt] (xo) to [bend left=10] (fxo);
%        \draw[->, shorten >=5pt, shorten <=5pt] (xt) to [bend left=10] (fxt);
%        \draw[->, shorten >=5pt, shorten <=5pt] (xr) to [bend right=12.5] (fxt);
      \end{tikzpicture}
    \end{center}
  \end{defn*}

  \pagebreak
\end{document}
