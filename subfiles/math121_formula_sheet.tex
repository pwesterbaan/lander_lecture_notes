\documentclass{article}
\usepackage[margin=1in]{geometry}
\usepackage{amsmath, tabularx, xfrac, actuarialsymbol, mathtools}
\usepackage{xcolor}
\usepackage{booktabs}
\usepackage{colorPalette}
\newcommand*{\sbrkt}[1]{{\left[#1\right]}}
\newcommand*{\parens}[1]{{\left(#1\right)}}

\newcount\examUnit
\examUnit=-1 %TODO (based on chapter number)
\begin{document}
  \definecolor{greenish}{rgb}{0,0.5,0}
  \newcommand{\chOne}{%
    Slope of a Line: & $\mathllap{m}=\displaystyle\frac{y_2-y_1}{x_2-x_1}$\\
    Point-Slope form: & $\mathllap{y-y_1}=m(x-x_1)$\\
    Slope-Intercept form: & $\mathllap{y}=mx+b$\\
    Revenue: & $\mathllap{R(x)}=(\textnormal{Price per unit})(\textnormal{number of units})$\\
    Cost: & $\mathllap{C(x)}=(\textnormal{variable costs})+(\textnormal{fixed costs})$\\
    Profit: & $\mathllap{P(x)}=R(x)-C(x)$\\
  }
  \newcommand{\chTwo}{%
    Quadratic formula: & $\mathllap{x}=\dfrac{-b\pm\sqrt{b^2-4ac}}{2a}$\\[5pt]
    Vertex of a parabola: & $\parens{-\dfrac{b}{2a}, f\parens{-\dfrac{b}{2a}}}$\\
  }
  \newcommand{\chFour}{%
    Slope of a Line: & $\mathllap{m}=\displaystyle\frac{y_2-y_1}{x_2-x_1}$\\
    Point-Slope form: & $\mathllap{y-y_1}=m(x-x_1)$\\
    Slope-Intercept form: & $\mathllap{y}=mx+b$\\
  }
  \newcommand{\chFive}{%
    Negative exponents: & $\mathllap{x^{-m}}=\dfrac{1}{x^m}$\quad or\quad $\parens{\dfrac{a}{b}}^{-m}=\parens{\dfrac{b}{a}}^m$\\
    Fractional exponents: & $\mathllap{x^{\sfrac{1}{m}}}=\sqrt[m]{x}$\quad or \quad $x^{\sfrac{n}{m}}=\sqrt[m]{x^n}$\\
    Exponential and Logarithmic forms: &$\mathllap{\textcolor{lander_blue}{a}^{\textcolor{greenish}{y}}}=\textcolor{red}{x} \quad\longleftrightarrow\quad \log_{\textcolor{lander_blue}{a}}(\textcolor{red}{x})=\textcolor{greenish}{y}$\\
    % &$a^1&=a& \log_a(a)=1$\\
    % &$a^0&=1& \log_a(1)=0$\\
    % &$a^xa^y&=a^{x+y}& \log_a(xy)=\log_a(x)+\log_a(y)$\\
    % &$\dfrac{a^x}{a^y}=a^{x-y}& \log_a\parens{\frac{x}{y}}=\log_a(x)-\log_a(y)$\\
    % &$a^{xy}=\parens{a^x}^y \quad\longleftrightarrow\quad
    Change of base formula: & $\mathllap{\log_{\textcolor{lander_blue}{b}}(\textcolor{red}{x})}=\dfrac{\log_a(\textcolor{red}{x})}{\log_a(\textcolor{lander_blue}{b})}=\dfrac{\log(\textcolor{red}{x})}{\log(\textcolor{lander_blue}{b})}=\dfrac{\ln(\textcolor{red}{x})}{\ln(\textcolor{lander_blue}{b})}$\\
    Exponent within a Logarithm: &$\mathllap{\log_a(\textcolor{red}{x}^{\textcolor{greenish}{y}})}=\textcolor{greenish}{y}\log_a(\textcolor{red}{x})$\\
    % &$a^{\log_a(x)}&=x& \log_a(a^x)&=x$
  }
  \newcommand{\chSix}{%
    Simple Interest: & $\mathllap{I}=Prt$\\
    Future value (simple interest):& $\mathllap{S}=P+I=P+Prt=P(1+rt)$\\
    Future value (compound $m$ times/yr):& $\mathllap{S}=P\parens{1+\dfrac{r}{m}}^{mt}$\\
%    Future value (compound $m$ times):& $\mathllap{S}=P\parens{1+i}^n$, \hfill where $i=\dfrac{r}{m}$ and $n=mt$\\
    Future value (compound continuously):& $\mathllap{S}=Pe^{rt}$\\[10pt]
%    & $\mathllap{S_{\angln i}}=\sbrkt{\dfrac{(1+i)^n-1}{i}}$, \hfill where $i=\dfrac{r}{m}$ and $n=mt$\\
    Future value of Ordinary Annuity:& $\mathllap{S}=R\cdot \sbrkt{\dfrac{(1+i)^n-1}{i}}$ \hfill $i=\dfrac{r}{m}$,\ \ $n=mt$\\[10pt]
    Future value of Annuity Due:& $\mathllap{S_{\textnormal{due}}}=R\cdot \sbrkt{\dfrac{(1+i)^n-1}{i}}(1+i)$\\[10pt]
%    & $\mathllap{a_{\angln i}}=\sbrkt{\dfrac{1-\parens{1+i}^{-n}}{i}}$, \hfill where $i=\dfrac{r}{m}$ and $n=mt$\\
%    Present value of Ordinary Annuity:& $\mathllap{A_n}=R\cdot \sbrkt{\dfrac{1-\parens{1+i}^{-n}}{i}}$ \hfill $i=\dfrac{r}{m}$, $n=mt$\\[10pt]
%    Present value of Annuity Due:& $\mathllap{A_{(n,\textnormal{due})}}=R\cdot \sbrkt{\dfrac{1-\parens{1+i}^{-n}}{i}}(1+i)$\\
    Amortization:& $\mathllap{R}=A_n\sbrkt{\dfrac{i}{1-(1+i)^{-n}}}$
  }
  \begin{center}
    \section*{Math 121 Formula Sheet}

    %TODO Why does the following affect section name printing?
    \renewcommand{\arraystretch}{2.05}
    \begin{tabularx}{0.95\linewidth}{@{}>{\bfseries\hsize=1.075\hsize}X>{\hsize=0.925\hsize}X@{}}
      \ifnum\examUnit=1 \chOne  \fi
      \ifnum\examUnit=4 \chFour \fi
      \ifnum\examUnit=2 \chTwo  \fi
      \ifnum\examUnit=5 \chFive \fi
      \ifnum\examUnit=6 \chSix  \fi
      \ifnum\examUnit=-1 %%Final Exam
        \multicolumn{2}{c}{\hrulefill\textbf{ Linear equations and inequalities }\hrulefill}\\ \chOne
%        \multicolumn{2}{c}{\hrulefill\textbf{ Quadratics }\hrulefill}\\ \chTwo
        \multicolumn{2}{c}{\hrulefill\textbf{ Logs and exponentials }\hrulefill}\\ \chFive
        \multicolumn{2}{c}{\hrulefill\textbf{ Finance }\hrulefill}\\ \chSix
      \fi
    \end{tabularx}
  \end{center}
  % \vspace*{\stretch{1}}

  \pagebreak
\end{document}