\documentclass[../mathNotesPreamble]{subfiles}

\begin{document}
%  \relscale{1.4} %TODO
  \section{2.2: The Algebra of Functions}
  
    \begin{defn*}
      Let $f$ and $g$ be functions with domains $A$ and $B$, respectively. Then the \textbf{sum} $f+g$, \textbf{difference} $f-g$, and \textbf{product} $fg$ of $f$ and $g$ are functions with domain $A\cap B$.
      \begin{align*}
        (f+g)(x)&=f(x)+g(x)\\
        (f-g)(x)&=f(x)-g(x)\\
        (fg)(x)&=f(x)g(x)
        \intertext{The \textbf{quotient} $f/g$ of $f$ and $g$ has domain $A\cap B$ excluding all numbers $x$ such that $g(x)=0$ and rule given by}
        \parens{\frac{f}{g}}(x)&=\frac{f(x)}{g(x)}
      \end{align*}
    \end{defn*}
    \begin{ex*}
      Let $f(x)=\sqrt{x+1}$ and $g(x)=4-x$. Find the domain of the following:
    \end{ex*}
    \begin{extasks}[after-item-skip=\stretch{1}](2)
      \task $f(x)+g(x)=$
      \task $f(x)-g(x)=$
      \task $f(x)g(x)=$
      \task $\dfrac{f(x)}{g(x)}=$
    \end{extasks}
    \vspace*{\stretch{1}}
    \pagebreak
    
    \begin{defn*}[The Composition of Two Functions]
      Let $f$ and $g$ be functions. Then the composition of $g$ and $f$ is the function $g\circ f$ defined by
      \begin{align*}
        \parens{g\circ f}(x)=g\parens{f(x)}
      \end{align*}
      The domain of $g\circ f$ is the set of all $x$ is the domain of $f$ such that $f(x)$ lies in the domain of $g$.
    \end{defn*}
    
    \begin{ex*}
      Let $f(x)=\sqrt{x+1}$ and $g(x)=4-x$. Find the domain of the following:
    \end{ex*}    
    \begin{flushright}
      \smash{\raisebox{-\height}{
        \newcommand{\funcMachine}{--++(0,-vert) --++(dx,-dy) |- ++(-a,-vert) |- ++(c,-h)
           -- ++(0,-vert) --++(-dx, -dy)[sharp corners] --++(0,-vert-vgap) %bottom left portion
          --++(b+2*dx,0)[rounded corners] --++(0,vert+vgap) --++(-dx,dy) |- ++(a,vert)
          -- ++(0,h) -| ++(-c,vert) --++(dx,dy) --++(0,vert); %top right portion
          }
        %bottom left corner is (0,0)
        %a,b,c are widths representing left wall to chute, width of chute, chute to right wall
        %h is height of rectangle
        %dx, dy control slope of "funnel"
        %vert is the height of straight parts of "funnel"
        %vgap is space between function diagrams
        %paths start at top left (on "funnel")
        \begin{tikzpicture}[scale=0.7,
        declare function={
          a=1.25; b=1.25; c=5;
          dx=0.9*a; dy=0.5*a;
          h=1.75; vert=0.6; vgap=0.2;}]
          \begin{scope}
            \clip (-0.1+0,h+2*vert+dy) rectangle ++(0.2+a+b+c,-4*vert-2*dy-h);
            \draw[line width=1pt, fill=lander_blue!25, rounded corners=6pt] 
              (a-dx,h+2*vert+dy) \funcMachine %f(x)
          \end{scope}
          \node (Input_f) [align=center] at ({a+b/2},{h+vert+dy}) {$x$};
          \node[align=center] at ({(a+b+c)/2},{h/2}) {$f$};
          \node (Output_f) [align=center] at ({c+b/2},{-vert-dy}) {$f(x)$};
          \draw[->] (Input_f) -- ++(0,-vert-dy);
          \draw[<-] (Output_f) -- ++(0,vert+dy);
%
          \begin{scope}
            \clip (-0.1+c-a,-2*vert-dy-vgap) rectangle ++(0.2+a+b+c,-4*vert-2*dy-h);
            \draw[line width=1pt, fill=lander_gold!75, rounded corners=6pt] 
              (c-dx,-2*vert-dy-vgap) \funcMachine %g(x)
          \end{scope}
          \node (Input_g) [align=center] at ({c+b/2},{-2*vert-dy-vgap}) {};
          \node[align=center] at ({c-a+(a+b+c)/2},{-4*vert-2*dy-vgap-h/2}) {$g$};
          \node (Output_g) [align=center] at ({2*c-a+b/2},{-h-5.25*vert-3*dy-vgap}) {$g\parens{f(x)}$};
          \draw[->] (Input_g) -- ++(0,-vert-dy);
          \draw[<-] (Output_g) -- ++(0,vert+dy);
%
      \end{tikzpicture}
      }}
    \end{flushright}
    \begin{extasks}[after-item-skip=\stretch{1}](1)
      \task $g\parens{f(x)}=$
      \task $f\parens{g(x)}=$
      \task $f\parens{f(x)}=$
%      \task $g\parens{g(x)}=$
    \end{extasks}
    \vspace*{\stretch{1}}

  \pagebreak
\end{document}
