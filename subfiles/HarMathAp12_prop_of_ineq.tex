\documentclass[../mathNotesPreamble]{subfiles}

\providecommand{\relscalefact}{1.4}
\begin{document}
  \noindent\textbf{Properties of Inequalities}
  \begin{center}
    \fbox{\parbox{0.9875\linewidth}{
      \begin{description}
        \item[Substitution Property:] The inequality formed by substituting one expression for an equal expression is equivalent to the original inequality:
          \begin{align*}
            5x-4x+2&<6\\
            x&<4\quad
            \Rightarrow \textnormal{The solution set is } \set{x: x<6}
          \end{align*}
        \item[Addition Property:] The inequality formed by adding the same quantity to both sides of an inequality is equivalent to the original inequality:
          \begin{align*}
            x-4&<6 & x+5&\geq12\\
            x-4\textcolor{red}{+4}&<6\textcolor{red}{+4}& x+5\textcolor{red}{+(-5)}&\geq 12\textcolor{red}{+(-5)}\\
            x&<10& x&\geq 7
          \end{align*}
        \item[Multiplication Property] The inequality formed by multiplying both sides of an inequality by the same \emph{positive} quantity is equivalent to the original inequality. The direction of the inequality is flipped when multiplying by a \emph{negative} quantity:
          \begin{align*}
            \frac{1}{3}x&>6 & 5x-5\textcolor{red}{+5}&\leq6x+20\textcolor{red}{+5}\\
            \textcolor{red}{3}\parens{\frac{1}{3}x}&>\textcolor{red}{3}(6) & \dfrac{-x}{\textcolor{red}{-1}}&\leq \dfrac{25}{\textcolor{red}{-1}}\\
            x&>18& x&\geq-25
          \end{align*}
      \end{description}
    }}
  \end{center}
\end{document}