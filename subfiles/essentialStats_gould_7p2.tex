\documentclass[../mathNotesPreamble]{subfiles}

\providecommand{\relscalefact}{1.4}
\begin{document}
\relscale{\relscalefact}
  \section{7.2: Measuring the Quality of a Survey}
    The true population proportion can be estimated by the sample proportion. How accurate can we expect our estimate to be?
    \begin{itemize}
      \item The \emph{accuracy} of an estimation method is measured in terms of \emph{bias}
      \item The \emph{precision} of an estimation method is measured in terms of \emph{standard error}
    \end{itemize}
    \vspace*{2\baselineskip}

    \begin{ex*}
      Consider a group of 8 people, where 2 identify as female, and 6 identify as male. What is the true population proportion of females? When using a sample size of $n=4$, what are possible sample proportions?
    \end{ex*}
    \vspace*{\stretch{1}}
    \begin{flushright}
      \begin{tikzpicture}
        \begin{axis}[
          axis x line=center,
          axis y line=left,
          axis line style={black,->},
          xmin=-0.05,
          ymin=0,
          enlargelimits={value=0.075, auto},
          xlabel=Sample Proportion $\hat{p}$, xlabel style={at={(axis description cs:0.5,-0.05)},anchor=north},
          ylabel=Probability, ylabel style={at={(axis description cs:-0.1, 0.5)}, anchor=south},
          ticklabel style={font=\normalsize,inner sep=0.5pt,fill=white,opacity=1.0, text opacity=1}]
          \foreach \star/\freq in {
            0/{15/70},
            0.25/{40/70},
            0.5/{15/70}
            }{
              \addplot[draw=lander_blue, line width=0.95pt] coordinates{(\star,0) (\star,\freq)};
              \addplot[soldot, mark size=2pt, lander_blue] coordinates{(\star,\freq)};
            }
        \end{axis}
      \end{tikzpicture}
    \end{flushright}
    \pagebreak

    \begin{ex*}
      Now, consider a group of 1000 people where 25\% identify as female ($p=0.25$). When using a sample size of $n=10$, what are possible sample proportions?
    \end{ex*}
    \vspace*{\stretch{1}}
    \begin{flushright}
      \begin{tikzpicture}
        \begin{axis}[
          axis x line=center,
          axis y line=left,
          axis line style={black,->},
          xmin=-0.1,
          ymin=0,
          enlargelimits={value=0.075, auto},
          xlabel=Sample Proportion $\hat{p}$, xlabel style={at={(axis description cs:0.5,-0.05)},anchor=north},
          ylabel=Probability, ylabel style={at={(axis description cs:-0.1, 0.5)}, anchor=south},
          ticklabel style={font=\normalsize,inner sep=0.5pt,fill=white,opacity=1.0, text opacity=1}]
          \foreach \star/\freq in {
            0/0.0554689157775412,
            0.1/0.187142091017345,
            0.2/0.282604734481045,
            0.3/0.251542302920769,
            0.4/0.146141344674062,
            0.5/0.0579072818091049,
            0.6/0.0158481725237162,
            0.7/0.00295808307193478,
            0.8/0.000360368074772804,
            0.9/0.0000258742246239485,
            1/0.00000083142508458288
            }{
              \addplot[draw=lander_blue, line width=0.95pt] coordinates{(\star,0) (\star,\freq)};
              \addplot[soldot, mark size=2pt, lander_blue] coordinates{(\star,\freq)};
            }
        \end{axis}
      \end{tikzpicture}
    \end{flushright}
    \vspace*{\stretch{1}}

    \noindent
    Finally, consider a group of 1000 people where 25\% identify as female ($p=0.25$). When using a sample size of $n=100$, what are possible sample proportions?
    \vspace*{\stretch{1}}
    \begin{flushright}
      \begin{tikzpicture}
        \begin{axis}[
          scaled ticks=false,
          /pgf/number format/.cd, fixed, precision=6, %forces large ytick labels
          axis x line=center,
          axis y line=left,
          axis line style={black,->},
          ymin=0,
          enlargelimits={value=0.075, auto},
          xlabel=Sample Proportion $\hat{p}$, xlabel style={at={(axis description cs:0.5,-0.05)},anchor=north},
          ylabel=Probability, ylabel style={at={(axis description cs:-0.1, 0.5)}, anchor=south},
          ticklabel style={font=\normalsize,inner sep=0.5pt,fill=white,opacity=1.0, text opacity=1}]
          %the following was generated using python:
          % from scipy.special import comb
          % for i in range(101):
          %     print(f"""{i/100}/{comb(250,i)*comb(750,100-i)/comb(1000,100)},""")
          \foreach \star/\freq in {
0.0/5.354464293688157e-14,
0.01/2.0562458885112173e-12,
0.02/3.887156242156614e-11,
0.03/4.82253421691469e-10,
0.04/4.416785065100699e-09,
0.05/3.184940338852979e-08,
0.06/1.8833710972266445e-07,
0.07/9.392693952156758e-07,
0.08/4.0323991380068734e-06,
0.09/1.5136964155900303e-05,
0.1/5.029829710653921e-05,
0.11/0.00014942142999597245,
0.12/0.00040009355959788725,
0.13/0.0009722194584309997,
0.14/0.0021564321227820074,
0.15/0.004387663795888279,
0.16/0.008224810841113003,
0.17/0.014257596074870909,
0.18/0.022931441324986427,
0.19/0.034320489321603066,
0.2/0.04792318475343966,
0.21/0.06257800010387081,
0.22/0.076575933433404,
0.23/0.08797896506102307,
0.24/0.09506579687811512,
0.25/0.09676148901801435,
0.26/0.09290226030839627,
0.27/0.08424683335364443,
0.28/0.07224263591257024,
0.29/0.05864233875511893,
0.3/0.04510573222577011,
0.31/0.032903617842652395,
0.32/0.022782559073526696,
0.33/0.014984189880029827,
0.34/0.00936769593506021,
0.35/0.005570213691360771,
0.36/0.003152078730834223,
0.37/0.0016983701285461182,
0.38/0.0008717269260618604,
0.39/0.0004264068592966938,
0.4/0.00019885082195658839,
0.41/8.84374133160243e-05,
0.42/3.7521392224014344e-05,
0.43/1.519037805935954e-05,
0.44/5.8695071627719196e-06,
0.45/2.1650083015099866e-06,
0.46/7.624459207742268e-07,
0.47/2.563907403537759e-07,
0.48/8.233369076727608e-08,
0.49/2.5249863881939202e-08,
0.5/7.395324418671383e-09,
0.51/2.068564353073617e-09,
0.52/5.525578294659768e-10,
0.53/1.4094606970781903e-10,
0.54/3.432816865069962e-11,
0.55/7.982018531387024e-12,
0.56/1.7716059442766619e-12,
0.57/3.752556723580026e-13,
0.58/7.583885702551988e-14,
0.59/1.461988819425456e-14,
0.6/2.687519822753178e-15,
0.61/4.709402746750783e-16,
0.62/7.8635845519779e-17,
0.63/1.250638884200339e-17,
0.64/1.893638489097993e-18,
0.65/2.7283062556218837e-19,
0.66/3.7383153472891014e-20,
0.67/4.8683154962925716e-21,
0.68/6.021578994363819e-22,
0.69/7.068931499704923e-23,
0.7/7.869796556512385e-24,
0.71/8.301635327541362e-25,
0.72/8.289817825724309e-26,
0.73/7.828199102562291e-27,
0.74/6.982783991182631e-28,
0.75/5.876454168947451e-29,
0.76/4.659541646575609e-30,
0.77/3.4759902670792874e-31,
0.78/2.4357152142699488e-32,
0.79/1.600379637583501e-33,
0.8/9.840690545478456e-35,
0.81/5.6506979876388725e-36,
0.82/3.02285606395955e-37,
0.83/1.5025093669214723e-38,
0.84/6.9184249589496916e-40,
0.85/2.9412303627005353e-41,
0.86/1.150079807204868e-42,
0.87/4.1182539299461755e-44,
0.88/1.3437084376625606e-45,
0.89/3.9716124170476665e-47,
0.9/1.0561149535421369e-48,
0.91/2.505945226479389e-50,
0.92/5.253146049297284e-52,
0.93/9.609367149034548e-54,
0.94/1.5100512606941688e-55,
0.95/1.9970441257641486e-57,
0.96/2.161122092084223e-59,
0.97/1.837247558928331e-61,
0.98/1.1504101134293209e-63,
0.99/4.716385139545823e-66,
1.0/9.495655414286151e-69
            }{
              \addplot[draw=lander_blue, line width=0.95pt] coordinates{(\star,0) (\star,\freq)};
              \addplot[soldot, mark size=2pt, lander_blue] coordinates{(\star,\freq)};
            }
        \end{axis}
      \end{tikzpicture}
    \end{flushright}
    \pagebreak

    \begin{defn*}
      \begin{itemize}
        \item The \textbf{sampling distribution} is the probability distribution of $\hat{p}$.
        \item The \textbf{standard error} for $\hat{p}$ is given by
          \[SE=\sqrt{\frac{p(1-p)}{n}}\]
          provided that
          \begin{itemize}
            \item The sample is randomly selected from the population of interest.
            \item If sampling without replacement, the population needs to be much larger than the sample size (e.g. at least 10 times bigger)
          \end{itemize}
          Since the true population proportion is typically unknown, we can estimate the standard error:
          \[SE_{est}=\sqrt{\frac{\hat{p}(1-\hat{p})}{n}}\]
      \end{itemize}
    \end{defn*}

    \noindent
    \emph{Note}: Larger sample sizes have smaller standard error!

  \pagebreak
\end{document}
