\documentclass[../mathNotesPreamble]{subfiles}

\begin{document}
%  \relscale{1.4} %TODO
  \section{2.5: One-Sided Limits and Continuity}
  Consider the function

  \noindent
  \begin{minipage}{0.5\linewidth}
    \begin{align*}
      f(x)=\begin{cases}
        x-1,& x<0\\
        x+1,& x\geq 0
      \end{cases}
    \end{align*}
  \end{minipage}\hfill%
  \begin{minipage}{0.45\linewidth}
    \begin{flushright}
      \smash{\raisebox{-0.65\height}{
      \begin{tikzpicture}[scale=1.0, declare function={
        f(\x)=\x<0 ? \x-1 : (\x>0 ? \x+1: inf);}]
        \begin{axis}[
          unbounded coords=jump,
          axis lines=center,
          axis line style={black,->},
          xmin=-2, xmax=2,
          ymin=-4, ymax=4,
          enlargelimits={abs=0.75},
          ticklabel style={font=\footnotesize,inner sep=0.5pt,fill=white,opacity=1.0, text opacity=1},
          xlabel=$x$, xlabel style={at={(ticklabel* cs:1)},anchor=north west},
          ylabel=$y$, ylabel style={at={(ticklabel* cs:1)},anchor=south west},
          every axis plot/.append style={line width=0.95pt, color=lander_blue, samples=255}
          ]
          \addplot[->,samples at={-2.5,-10e-4,0,10e-4,2.5}] {f(x)};
          \addplot[holdot] coordinates{(0,-1)};
          \addplot[soldot] coordinates{(0,1)};
        \end{axis}
      \end{tikzpicture}
      }}
    \end{flushright}
  \end{minipage}
  \vspace*{\baselineskip}

  \noindent What is $\displaystyle\lim_{x\to 0} f(x)$?
  \vspace*{\stretch{1}}
  \begin{defn*}[One-Sided Limits]
    The function $f$ has a \textbf{right-hand limit} $L$ as $x$ approaches $a$ from the right, written
    \begin{align*}
      \lim_{x\to a^+} f(x)=L
    \end{align*}
    if the values of $f(x)$ can be made as close to $L$ as we please by taking $x$ sufficiently close to (but not equal to) $a$ and to the right of $a$.

    The function $f$ has a \textbf{left-hand limit} $L$ as $x$ approaches $a$ from the left, written
    \begin{align*}
      \lim_{x\to a^-} f(x)=M
    \end{align*}
    if the values of $f(x)$ can be made as close to $L$ as we please by taking $x$ sufficiently close to (but not equal to) $a$ and to the left of $a$.
  \end{defn*}
  \pagebreak

  \begin{thmBox*}[Theorem 3]
    Let $f$ be a function that is defined for all values of $x$ close to $x=a$ with the possible exception of $a$ itself. Then
    \begin{align*}
      \lim_{x\to a} f(x)=L \quad \textnormal{ if and only if } \quad \lim_{x\to a^-} f(x)=\lim_{x\to a^+} f(x)=L
    \end{align*}
  \end{thmBox*}
  % \vspace*{-1.5\baselineskip}

  \begin{ex*}
    Using the graph below, evaluate the following limits:
  \end{ex*}
  \begin{center}
    % \smash{\raisebox{-\height}{
    \begin{tikzpicture}[declare function={
      g(\x)=\x<-2 ? 1/(2*(\x+2))+3 :
        (\x==-2 ? inf :
        (\x< -1 ? 1/(\x+2)-2:
        (\x==-1 ? inf :
        (\x<  1 ? -\x^2+1:
        (\x<  3 ? 2*(\x-2)^2-2: -exp(-\x+3)+1))));}]
      \begin{axis}[
        unbounded coords=jump,
        grid=both,
        grid style={line width=0.35pt, draw=gray!75},
        axis lines=center,
        axis line style={black,-},
        xmin=-5, xmax=7,
        ymin=-5, ymax=5,
        xtick={-6,-5,...,7},
        ytick={-6,-5,...,6},
        enlargelimits={abs=0.75},
        width=0.8*\linewidth, height=\axisdefaultheight,
        ticklabel style={font=\footnotesize,inner sep=0.5pt,fill=white,opacity=1.0, text opacity=1},
        every axis plot/.append style={line width=0.95pt, color=lander_blue},
        ]
          %% Using 'samples at' instead of domain to control plotting discontinuities
          \addplot[->,samples at={-6,-5.975,...,-2.005,-2,-1.95,...,-1.05,-1.005,-1,-0.95,...,7.5}] {g(x)};
          \addplot[holdot] coordinates{(-1,0)(2,-2)};
          \addplot[soldot] coordinates{(-1,-1)(2,-1)};
      \end{axis}
    \end{tikzpicture}
    % }}
  \end{center}

  \noindent
  % \begin{minipage}{0.5\linewidth}
  %   \begin{ex*}
  %     Using the graph to the right, evaluate the following limits:
  %   \end{ex*}
  % \end{minipage}
  \begin{extasks}[after-item-skip=\stretch{1}](3)
    \task $\displaystyle \lim_{x\to -2^-} f(x)$
    \task $\displaystyle \lim_{x\to -2^+} f(x)$
    \task $\displaystyle \lim_{x\to -2} f(x)$
    \task $\displaystyle \lim_{x\to -1^-} f(x)$
    \task $\displaystyle \lim_{x\to -1^+} f(x)$
    \task $\displaystyle \lim_{x\to -1} f(x)$
    \task $\displaystyle \lim_{\phantom{-}x\to 1\phantom{^-}} f(x)$
    \task $\displaystyle \lim_{\phantom{-}x\to 2\phantom{^-}} f(x)$
    \task $\displaystyle \lim_{x\to \infty} f(x)$
  \end{extasks}
  \vspace*{\stretch{1}}
  \pagebreak

  \begin{defn*}[Continuity of a Function at a Number]
    A function $f$ is \textbf{continuous} at $a$ if $\ds\lim_{x \to a}f(x)=f(a)$.
  \end{defn*}

  \begin{center}
    \fbox{\parbox{0.9\linewidth}{\textbf{Continuity Checklist:}

    In order for $f$ to be continuous at $a$, the following three conditions must hold:
    \begin{enumerate}
      \item $f(a)$ is defined ($a$ is in the domain of $f$),
      \item $\ds\lim_{x \to a} f(x)$ exists,
      \item $\ds\lim_{x \to a} f(x)=f(a)$ (the value of $f$ equals the limit of $f$ at $a$).
    \end{enumerate}
    }}
  \end{center}
  \begin{ex*}
    Determine the values of $x$ for which the following functions are continuous:
  \end{ex*}
  \begin{extasks}[after-item-skip=\stretch{1}](1)
    \task $f(x)=3x^3+2x^2-x+10$
    \task $g(x)=\dfrac{8x^{10}-4x+1}{x^2+1}$
    \task $h(x)=\dfrac{4x^3-3x^2+1}{x^2-3x+1}$
  \end{extasks}
  \vspace*{\stretch{1}}
  \pagebreak

  \begin{ex*}
    Determine whether the following are continuous at $a$:
  \end{ex*}
  \begin{extasks}[after-item-skip=\stretch{1}](2)
    \task $f(x)=x^2+\sqrt{7-x},\ a=4$
    \task $g(x)=\dfrac{1}{x-3},\ a=3$
    \task
      $h(x)=\begin{cases}
        \dfrac{x^2+x}{x+1}, &x\neq -1\\
        0,& x=-1
      \end{cases},\ a=-1$
    \task
      $j(x)=\abs{x}=\begin{cases}
        x,& x\geq 0\\
        -x& x<0
      \end{cases},\ a=0$
    \task
      $k(x)=\begin{cases}
        \dfrac{x^2+x-6}{x^2-x},& x\neq 2\\
        -1,& x=2
      \end{cases},\ a=2$
    \end{extasks}
  \vspace*{\stretch{1}}
  \pagebreak
  \begin{thmBox*}[Properties of Continuous Functions]
    \begin{enumerate}
      \item The constant function $f(x)=c$ is continuous everywhere.
      \item The identify function $f(x)=x$ is continuous everywhere.
    \end{enumerate}
    If $f$ and $g$ are continuous at $x=a$, then
    \begin{enumerate}[resume*]
      \item $\sbrkt{f(x)}^n$, where $n$ is a real number, is continuous at $x=a$ whenever it is defined at that number
      \item $f\pm g$ is continuous at $x=a$
      \item $fg$ is continuous at $x=a$
      \item $f/g$ is continuous at $x=a$ provided that $g(a)\neq 0$
    \end{enumerate}
  \end{thmBox*}

  \begin{thmBox*}[Polynomial and Rational Functions]
    \begin{enumerate}
      \item A polynomial function is continuous for all $x$.
      \item A rational function (a function of the form $\frac{p}{q}$, where $p$ and $q$ are polynomials) is continuous for all $x$ for which $q(x)\neq 0$.
    \end{enumerate}
  \end{thmBox*}
  \pagebreak

  \begin{defn*}\
    \begin{enumerate}[label=,itemsep=\stretch{1}]
      \item A \textbf{removable discontinuity} at $x=a$ is one that disappears when the function becomes continuous after defining $f(a)=\ds\lim_{x \to a} f(x)$.
      \item A \textbf{jump discontinuity} is one that occurs whenever $\ds\lim_{x \to a^-} f(x)$ and $\ds\lim_{x \to a^+} f(x)$ both exist, but $\ds\lim_{x \to a^-} f(x)\neq\lim_{x \to a^+} f(x)$.
      \item A \textbf{vertical discontinuity} occurs whenever $f(x)$ has a vertical asymptote.
    \end{enumerate}
  \end{defn*}
  \vspace*{\stretch{1}}

  \uplevel{\centering
  \begin{tikzpicture}
    \begin{groupplot}[
      group style={group size=2 by 2, horizontal sep=2cm, vertical sep=1cm},
      axis lines=center,
      axis line style={->},
      xmin=-0.5, xmax=5.5,
      ymin=-0.5, ymax=2.5,
      xmajorticks=false,
      ymajorticks=false,
%      xlabel=$x$, xlabel style={at={(ticklabel* cs:1)},anchor=north west},
%      ylabel=$y$, ylabel style={at={(ticklabel* cs:1)},anchor=south west},
      every axis plot/.append style={line width=0.95pt, color=lander_blue},
      ]
    \nextgroupplot
      \addplot[<->] expression[domain=-0.45:4.85, samples=101] {-(x-0.5)^2/8+2};
      \addplot[holdot] coordinates{(2.5,1.5)} node[above right, black, yshift=-5pt] {$\parens{\frac{5}{2},\frac{3}{2}}$};
      \node[align=left, anchor=south west] at (axis cs: 0.25,0.25) [draw, rectangle, rounded corners, font=\large] {Removable\\ discontinuity};
    \nextgroupplot
      \addplot[<->] expression[domain=-0.45:4.85, samples=101] {-(x-0.5)^2/8+2};
      \addplot[holdot] coordinates{(2.5,1.5)} node[above right, black, yshift=-5pt] {$\parens{\frac{5}{2},\frac{3}{2}}$};
      \addplot[soldot] coordinates{(2.5,2)} node[above right, black] {$\parens{\frac{5}{2},2}$};
      \node[align=left, anchor=south west] at (axis cs: 0.25,0.25) [draw, rectangle, rounded corners, font=\large] {Removable\\ discontinuity};
    \nextgroupplot
      \addplot[<-] expression[domain=-0.45:2.5, samples=101] {-(x-0.5)^2/8+2};
      \addplot[->] expression[domain=2.5:4.85, samples=101] {-(x-0.5)^2/8+2.5};
      \addplot[holdot] coordinates{(2.5,1.5)} node[below left, black, yshift=5pt] {$\parens{\frac{5}{2},\frac{3}{2}}$};
      \addplot[soldot] coordinates{(2.5,2)} node[above right, black] {$\parens{\frac{5}{2},2}$};
      \node[align=left, anchor=south west] at (axis cs: 0.25,0.25) [draw, rectangle, rounded corners, font=\large] {Jump\\ discontinuity};
    \nextgroupplot[
      xmin=-0.5, xmax=3.5,
      ymin=-7, ymax=35,
      ]
      \addplot[<->] expression[unbounded coords=jump, domain=-0.5:3.5, samples=99] {(x-2.5)^-2};
      \node[align=left, anchor=south west] at (axis cs: 3.5/22,3.5) [draw, rectangle, rounded corners, font=\large] {Infinite\\ discontinuity};
    \end{groupplot}
  \end{tikzpicture}}
  \vspace*{\stretch{1}}
  \pagebreak

  \begin{thmBox*}[Theorem 4: Intermediate Value Theorem]
    Suppose $f$ is continuous on the interval $\sbrkt{a,b}$ and $L$ is a number strictly between $f(a)$ and $f(b)$. Then there exists at least one number $c$ in $(a,b)$ satisfying $f(c)=L$.
  \end{thmBox*}

  \begin{center}
    \begin{tikzpicture}
      \begin{groupplot}[
        group style={group size=2 by 1, horizontal sep=2.75cm},
        axis lines=center,
        axis line style={->},
        xmin=-0.0625, xmax=3,
        ymin=-0.0625, ymax=2,
        enlargelimits={abs=0.75},
        width=\axisdefaultwidth, height=0.9*\axisdefaultheight,
        xlabel=$x$, xlabel style={at={(ticklabel* cs:1)},anchor=north west},
        ylabel=$y$, ylabel style={at={(ticklabel* cs:1)},anchor=south west},
        every axis plot/.append style={line width=0.95pt, color=lander_blue},
        ]
      \nextgroupplot[
        xtick={1,2.414,3},
        xticklabels={$a$,$c$,$b$},
        ytick={0.5,1.5,2.5},
        yticklabels={$f(a)$,$L$,$f(b)$},
        ]
          \addplot[-] expression[domain=1:3] {(x-1)^2/2+0.5};
          \addplot[soldot] coordinates{(1,0.5)};
          \addplot[soldot] coordinates{(3,2.5)};
          \draw[dashed] (axis cs:0,1.5) to (axis cs:4,1.5);
          \draw[dashed] (axis cs:2.414,0) to (axis cs:2.414, 1.5);
      \nextgroupplot[
        xtick={0.5,1,2,3,3.5},
        xticklabels={$a$,$c_1$,$c_2$,$c_3$,$b$},
        ytick={0.5625,1.5,2.4375},
        yticklabels={$f(b)$,$L$,$f(a)$},
        ]
          \addplot[-] expression[domain=0.5:3.5, samples=50] {-0.5*(x-1)*(x-2)*(x-3)+1.5};
          \addplot[soldot] coordinates{(0.5,2.4375)};
          \addplot[soldot] coordinates{(3.5,0.5625)};
          \draw[dashed] (axis cs:0,1.5) to (axis cs:4,1.5);
          \draw[dashed] (axis cs:1,0) to (axis cs:1, 1.5);
          \draw[dashed] (axis cs:2,0) to (axis cs:2, 1.5);
          \draw[dashed] (axis cs:3,0) to (axis cs:3, 1.5);
      \end{groupplot}
    \end{tikzpicture}
  \end{center}

%  \vspace*{\stretch{1}}
  \textit{Note:} It is important that the function be continuous on the interval $[a,b]$:
  \begin{center}
    \begin{tikzpicture}
      \begin{axis}[
        axis lines=center,
        axis line style={->},
        xmin=-0.5, xmax=3.5,
        ymin=-0.5, ymax=3.5,
        xtick={0.5,4},
        xticklabels={$a$,$b$},
        ytick={0.53125,1.5,3.5},
        yticklabels={$f(a)$,$L$,$f(b)$},
        enlargelimits={abs=0.75},
        width=\axisdefaultwidth, height=0.9*\axisdefaultheight,
        xlabel=$x$, xlabel style={at={(ticklabel* cs:1)},anchor=north west},
        ylabel=$y$, ylabel style={at={(ticklabel* cs:1)},anchor=south west},
        every axis plot/.append style={line width=0.95pt, color=lander_blue},
        ]
        \addplot[-] expression[domain=0.5:2] {x^2/8+0.5};
        \addplot[-] expression[domain=2:4] {x^2/8+1.5};
        \addplot[soldot] coordinates{(0.5,0.53125)(2,2)(4,3.5)};
        \addplot[holdot] coordinates{(2,1)};
        \draw[dashed] (axis cs:0,1.5) to (axis cs: 5,1.5);
      \end{axis}
    \end{tikzpicture}
  \end{center}
  \begin{thmBox*}[Theorem 5: Existence of Zeros of a Continuous Function]
    If $f$ is a continuous function on a closed interval $\sbrkt{a,b}$, and if $f(a)$ and $f(b)$ have opposite signs, then there is at least one solution of the equation $f(x)=0$ in the interval $(a,b)$.
  \end{thmBox*}
  \pagebreak


  \begin{ex*}
    Check the conditions of the Intermediate Value Theorem to see if there exists a value $c$ on the interval $(a,b)$ such that the following equations hold: \hfill\href{https://www.desmos.com/calculator/lqbkapgcee}{\textcolor{blue}{\underline{Graph}}}
  \end{ex*}
  %TODO IVT examples. Include at least one root problem
  \begin{extasks}[after-item-skip=\stretch{1}](2)
    \task $x^x-x^2=\frac{1}{2}$ \hfill on $[0,2]$\hspace*{15mm}
    \task $\sqrt{x^4+25x^3+10}=5$ \hfill on $[0,1]$\hspace*{15mm}
    \task $x+\sqrt{1-x^2}=0$ \hfill on $[-1,0]$\hspace*{15mm}
    \task $\dfrac{x^2}{x^2+1}=0$ \hfill on $[-1,1]$\hspace*{15mm}
  \end{extasks}
  \vspace*{\stretch{1}}
  \pagebreak

  \begin{ex*}
    Consider the function
    \begin{align*}
      f(x)=\dfrac{x+1}{x-1}
    \end{align*}
    on the interval $\sbrkt{0,2}$. Does there exist a $c$ on the interval $\sbrkt{0,2}$ such that $f(c)=1$?
  \end{ex*}
  \begin{flushright}
    \begin{tikzpicture}[declare function={
      f(\x)=2/(\x-1)+1;}]
      \begin{axis}[
        unbounded coords=jump,
        axis lines=center,
        axis line style={->},
        xmin=-2, xmax=3,
        ymin=-5, ymax=6,
        enlargelimits={abs=0.75},
        xlabel=$x$, xlabel style={at={(ticklabel* cs:1)},anchor=north west},
        ylabel=$y$, ylabel style={at={(ticklabel* cs:1)},anchor=south west},
        every axis plot/.append style={line width=0.95pt, color=lander_blue, samples=255},
        ]
        \addplot[<->] expression[domain=-2.5:0.68]{f(x)};
        \addplot[<->] expression[domain=1.35:3.5]{f(x)};
        \draw[dotted] (axis cs:1,-6) to (axis cs:1, 7);
        \draw[dotted] (axis cs:-3,1) to (axis cs:4, 1);
      \end{axis}
    \end{tikzpicture}
  \end{flushright}

  \pagebreak
\end{document}
