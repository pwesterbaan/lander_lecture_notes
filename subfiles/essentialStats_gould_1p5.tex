\documentclass[../mathNotesPreamble]{subfiles}

\begin{document}
%\relscale{1.4} %TODO
  \section{Section 1.5: Collecting Data to Understand Causality}
    \begin{defn*}
      \begin{itemize}
        \item In an \textbf{observational study}, we observe individuals and measure variables of interest but do not attempt to influence the responses. (Observe but do not disturb)
        \item In a \textbf{controlled experiment}, we deliberately impose some treatment on (that is, do something to) individuals in order to observe their responses. Researchers assign subjects to a treatment group or control group.
        \item \textbf{Anecdotal evidence} is a story based on someone’s experience.
      \end{itemize}
    \end{defn*}
    
    \vspace*{\stretch{1}}
    \begin{itemize}
      \item In an \textbf{observational study}, the researcher observes values of the response variable for the sampled subjects, without anything being done to the subjects (such as imposing a treatment).
      \item In short, an \emph{observational study} merely observes rather than experiments with the study subjects. 
    \end{itemize}
    \begin{center}
      \fbox{\parbox{0.75\linewidth}{
        \emph{Note:} Anecdotal evidence and observational studies:
        \begin{itemize}
          \item NEVER point to causality (cause-and effect).
          \item Only point to an association between variables!  
        \end{itemize}
        To establish cause-and-effect:	Use a controlled experiment!
        }}
    \end{center}

    \vspace*{\stretch{1}}
    \begin{defn*}
      Differences between two groups that could explain different experiences/outcomes are called \textbf{confounding variables} or \textbf{confounding factors}.
    \end{defn*}
  \pagebreak
  
  How to design a good experiment (``Gold standard'' in experiments):
  \begin{itemize}
    \item Random allocation -- participants randomly allocated to treatment and control group
    \item Use of a placebo if appropriate 
      \begin{itemize}
        \item A \textbf{placebo} is a fake treatment (e.g. sugar pill).
        \item The \textbf{Placebo-Effect} is reacting to a treatment you haven't received.
      \end{itemize}
    \item Blinding the study  -- used to avoid bias
      \begin{itemize}
        \item Single blind -- Researcher is unaware of treatment group
        \item Double blind -- Researcher and subjects are both unaware of treatment group
      \end{itemize}
    \item Large sample size -- accounts for variability
  \end{itemize}
  \pagebreak
\end{document}
