\documentclass[../mathNotesPreamble]{subfiles}

\begin{document}
%  \relscale{1.4} %TODO
  \section{1.6: Applications of Functions in Business and Economics}
  \begin{defn*}
    \textbf{Profit} is the difference between the revenue and total cost:
      \[P(x)=R(x)-C(x)\]
    where
      \begin{align*}
        P(x)&= \textnormal{profit from sale of $x$ units,}\\
        R(x)&= \textnormal{total revenue from sale of $x$ units,}\\
        C(x)&= \textnormal{total cost from production and sale of $x$ units.}
      \end{align*}
    \emph{Note}: In general, the symbols used in economics are $\pi$, $TR$ and $TC$ respectively.

    \vspace*{\baselineskip}
    In general, \textbf{total revenue} is
      \[\textnormal{Revenue}=\parens{\textnormal{price per unit}}\parens{\textnormal{number of units}}\]
    The \textbf{total cost} is composed of fixed cost and variable cost:
    \begin{itemize}
      \item \textbf{Fixed costs} $(FC)$ remain constant regardless of the number of units produced.
      \item \textbf{Variable costs} $(VC)$ are directly related to the number of units produced.
    \end{itemize}
    The total cost is given by
      \[\textnormal{Cost}=\textnormal{variable costs}+\textnormal{fixed costs}\]
  \end{defn*}
  \pagebreak

  \begin{ex*}
    Suppose a firm manufactures MP3 players and sells them for \$50 each. The costs incurred in the production and sale of the MP3 players are \$200,000 plus \$10 for each player produced and sold. Write the profit function for the production and sale of $x$ players.
  \end{ex*}
  \vspace*{\stretch{1}}

  \begin{center}
    \begin{tikzpicture}[scale=0.8, declare function={
      R(\x)=50*\x;
      C(\x)=10*\x+200000;
      P(\x)=R(\x)-C(\x);},
      every node/.append style={black, align=left, font=\large}]
      \begin{groupplot}[
        group style={group size=3 by 1, horizontal sep=20mm},
        axis lines=center,
        axis line style={black,->},
        xmin=0, xmax=15000,
        ymin=0, ymax=350000,
        width=0.3\linewidth/0.8,
        ticklabel style={font=\normalsize,inner sep=0.5pt,fill=white,opacity=1.0, text opacity=1},
        xlabel=$x$, xlabel style={at={(ticklabel* cs:1)},anchor=west, xshift=5pt},
        every axis plot/.append style={line width=0.95pt, color=lander_blue, samples=255},
        yticklabel style={/pgf/number format/fixed},
        ]
        \nextgroupplot[ylabel=$R(x)$, ylabel style={at={(ticklabel* cs:1)},anchor=south east}]
          \addplot[<->] expression[domain=0:7000]{R(x)}
          node[below right, pos=0.6, color=black] {$R(x)=50x$};
        \nextgroupplot[ylabel=$C(x)$, ylabel style={at={(ticklabel* cs:1)},anchor=south east}]
          \addplot[<->] expression[domain=0:12000]{C(x)}
          node[below right, pos=0.1, color=black] {$C(x)=10x+200,000$};
        \nextgroupplot[ylabel=$P(x)$, ylabel style={at={(ticklabel* cs:1)},anchor=south east}, ymin=-200000]
          \addplot[<->] expression[domain=0:12000]{P(x)}
          node[below right, pos=0.175, xshift=-6pt, color=black] {$P(x)=40x-200,000$};
      \end{groupplot}
    \end{tikzpicture}
  \end{center}
  \pagebreak

  \begin{ex*}
    The ABC company produces widgets which sell at \$25 each. ABC can produce 30 widgets at a total cost of \$2,050, and 180 widgets at a total cost of \$4,300. Find the revenue, cost, and profit functions.
  \end{ex*}
  \vspace*{\stretch{1}}

  \begin{defn*}[Marginals]
    The
    \begin{itemize}
      \item \textbf{marginal profit} $(\overline{MP})$ is the rate of change in profit\dots
      \item \textbf{marginal cost} $(\overline{MC})$ is the rate of change in costs\dots
      \item \textbf{marginal revenue} $(\overline{MR})$ is the rate of change in revenue\dots
    \end{itemize}
    with respect to the number of units produced and sold. When these functions are linear, the marginals are given by the slope of their respective function.
  \end{defn*}
  \pagebreak

  \begin{ex*}
    A manufacturer sells widgets for \$10 per unit. The manufacturer's variable costs are \$2.50 per unit, and the total cost of 100 units is \$1,450.
    \begin{tasks}[after-item-skip=\stretch{1}, label=\textbullet](1)
      \task Find the profit function. What are the marginal revenue, cost and profit?
      \task Find the break-even point (where $R(x)=C(x)$). What happens if we sell more or less than the break-even point?
    \end{tasks}
  \end{ex*}
  \vspace{\stretch{2}}

  \begin{center}
    \begin{tikzpicture}[scale=1.0, declare function={
      R(\x)=10*\x;
      C(\x)=2.5*\x+1200;
      P(\x)=R(\x)-C(\x);},
      every node/.append style={black, align=left, font=\small}]
      \begin{groupplot}[
        group style={group size=2 by 1, horizontal sep=30mm},
        axis lines=center,
        axis line style={black,->},
        xmax=290,
        ymin=0, ymax=2900,
        width=0.425\linewidth,
        ticklabel style={font=\normalsize,inner sep=0.5pt,fill=white,opacity=0.0, text opacity=1},
        xlabel=$x$, xlabel style={at={(ticklabel* cs:1)},anchor=west, xshift=5pt},
        ylabel=$y$, ylabel style={at={(ticklabel* cs:1)},anchor=south east},
        every axis plot/.append style={line width=0.95pt, color=lander_blue, samples=255},
        ]
        \nextgroupplot
          \addplot[<-, name path=R] expression[domain=0:160]{R(x)};
          \addplot[<-, name path=C] expression[domain=0:160]{C(x)};
          \addplot[red!50,opacity=0.6] fill between[of=R and C];
          \node at (40,950) {Loss\\ region};

          \addplot[->, name path=R] expression[domain=160:290]{R(x)} node[above left, pos=0.75] {$R(x)$};
          \addplot[->, name path=C] expression[domain=160:290]{C(x)} node[below right, pos=0.5] {$C(x)$};
          \addplot[lander_blue!50,opacity=0.6] fill between[of=R and C];
          \node at (260,2150) {Profit\\ region};
        \nextgroupplot[ymin=-1300, ymax=1300]
          \addplot[draw=none, name path=Z] expression[domain=0:160]{0};
          \addplot[<-, name path=P] expression[domain=0:160]{P(x)};
          \addplot[red!50,opacity=0.6] fill between[of=P and Z];
          \node at (40,-450) {Loss\\ region};

          \addplot[draw=none, name path=Z] expression[domain=160:290]{0};
          \addplot[->, name path=P] expression[domain=160:290]{P(x)} node[above left, pos=0.75] {$P(x)$};
          \addplot[lander_blue!50,opacity=0.6] fill between[of=P and Z];
          \node at (250,300) {Profit\\ region};
      \end{groupplot}
    \end{tikzpicture}
  \end{center}
  \pagebreak

  \begin{defn*}
    \begin{itemize}
      \item \textbf{Market equilibrium} occurs when the quantity of a commodity demanded is equal to the quantity supplied.
      \item The \textbf{law of demand} states that the quantity demanded will decrease as the price increases
      \item The \textbf{law of supply} states that the quantity supplied will increase as the price increases
    \end{itemize}
  \end{defn*}
  \begin{ex*}
    Below is a graph containing a supply and demand curve. Find the market equilibrium.
  \end{ex*}
  \begin{center}
    \begin{tikzpicture}[scale=1.0, declare function={
      S(\q)=\q/10+5;
      D(\q)=-\q/10+35;},
      every node/.append style={black, align=center, font=\large}]
      \begin{axis}[
        axis lines=center,
        axis line style={black,->},
        xmax=290,
        ymin=0, ymax=45,
        width=0.7\linewidth, height=0.35\linewidth,
        xtick={0,60,...,240},
        ytick={10,30,...,50},
        minor x tick num=1,
        minor y tick num=1,
        ticklabel style={font=\normalsize,inner sep=0.5pt,fill=white,opacity=0.0, text opacity=1},
        xlabel=$q$, xlabel style={at={(ticklabel* cs:1)},anchor=north west, xshift=5pt},
        ylabel=$p$, ylabel style={at={(ticklabel* cs:1)},anchor=south},
        every axis plot/.append style={line width=0.95pt, color=lander_blue, samples=255},
        clip=false
        ]
        \addplot[<->] expression[domain=0:290]{S(x)} node[right,pos=1] {Supply: $10p-q=50$};
        \addplot[<->] expression[domain=0:290]{D(x)} node[right,pos=1] {Demand: $10p+q=350$};
        \draw[dashed] (150,0) |- (0,20);
        \node at (axis cs: -35,20) {Equilibrium\\ price};
        \node at (axis cs: 150,-10) {Equilibrium\\ quantity};
      \end{axis}
    \end{tikzpicture}
  \end{center}
  \pagebreak

  \begin{ex*}
    Find the market equilibrium point for the following demand and supply functions:
      \begin{align*}
        \textnormal{Supply: \quad} p&=2q+170\\
        \textnormal{Demand: \quad} p&=-5q+450
      \end{align*}
  \end{ex*}
  \vspace*{\stretch{1.5}}
  
  \begin{ex*}
    Using the supply and demand functions above, modify the supply function to include a \$14 tax per unit sold, then find the new market equilibrium point.
  \end{ex*}
  \vspace*{\stretch{1}}
  \pagebreak

  \begin{ex*}
    Retailers will buy 45 Wi-Fi routers from a wholesaler if the price is \$10 each but only 20 if the price is \$60. The wholesaler will supply 56 outers at \$42 each and 70 at \$50 each. Assuming that the supply and demand functions are linear, find the market equilibrium point.
  \end{ex*}
  \pagebreak

  \pagebreak
\end{document}
