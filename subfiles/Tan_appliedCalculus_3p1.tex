\documentclass[../mathNotesPreamble]{subfiles}

\begin{document}
%  \relscale{1.4} %TODO
  \section{3.1: Basic Rules of Differentiation}
  \begin{thmBox*}[Rule 1: Derivative of a Constant]
    \begin{align*}
      \ddx\sbrkt{c}&=0
      \intertext{\textbf{Rule 2: The Power Rule}}\\[-1.75\baselineskip]
      \intertext{If $n$ is any real number, then}
      \ddx\sbrkt{x^n}&=nx^{\nmo}
    \end{align*}
  \end{thmBox*}
  \begin{ex*}
    Find the derivative of the following functions
  \end{ex*}
  \begin{extasks}[after-item-skip=\stretch{1}](2)
    \task $f(x)=x$
    \task $g(x)=x^8$
    \task $h(x)=x^{\frac{5}{2}}$
    \task $j(x)=\sqrt{x}$
    \task $k(x)=\dfrac{1}{\sqrt[3]{x}}$
    \task $\ell(x)=\pi^4$
  \end{extasks}
  \vspace*{\stretch{1}}
  \pagebreak
  
  \begin{thmBox*}[Rule 3: Derivative of a Constant Multiple of a Function]
    \begin{align*}
      \ddx\sbrkt{cf(x)}&=c\ddx\sbrkt{f(x)}
      \intertext{\textbf{Rule 4: The Sum Rule}}
      \ddx\sbrkt{f(x)\pm g(x)}&=\ddx\sbrkt{f(x)}\pm \ddx\sbrkt{g(x)}
    \end{align*}
  \end{thmBox*}
  \begin{ex*}
    Find the derivative of the following functions
  \end{ex*}
  \begin{extasks}[after-item-skip=\stretch{1}](2)
    \task $f(x)=5x^3$
    \task $g(x)=\dfrac{3}{\sqrt{x}}$
    \task $h(x)=4x^5+3x^4-8x^2+x+3$
    \task $j(t)=\dfrac{t^2}{5}+\dfrac{5}{t^2}+\pi$
  \end{extasks}
  \vspace*{\stretch{1}}
  \pagebreak
  
  \begin{ex*}
    Find the line tangent to the curve 
    \begin{align*}
      f(x)&=2x+\dfrac{1}{\sqrt{x}}
    \end{align*}
    at the point $(1,3)$
    \hfill \href[pdfnewwindow]{https://www.desmos.com/calculator/cwectqrate}{\textcolor{blue}{\underline{Graph}}}
  \end{ex*}
  \pagebreak
  
  \begin{ex*}
    An experimental rocket lifts off vertically. Its altitude (in feet) $t$ seconds into flight is given by 
    \begin{align*}
      s=f(t)=-t^3+96t^2+5, \qquad (t\geq 0)
    \end{align*}
  \end{ex*}
  \begin{extasks}[after-item-skip=\stretch{1}](1)
    \task Find an expression $v$ for the rocket's velocity at any time $t$.
    \task Compute the rocket's velocity when $t=0, 30, 50, 64$, and $70$. Interpret your results. 
    \task Using the results from above and the observation that at the highest point in its trajectory the rocket's velocity is zero, find the maximum altitude attained by the rocket.
  \end{extasks}
  \vspace*{\stretch{1}}
  \pagebreak

  \pagebreak
\end{document}
