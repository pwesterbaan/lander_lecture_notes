\documentclass[../mathNotesPreamble]{subfiles}

\providecommand{\relscalefact}{1.4}
\begin{document}
\relscale{\relscalefact}
  \section{5.6: Exponential Functions As Mathematical Models}

  \begin{ex*}
    Consider the exponential function
      \[Q(t)=Q_0e^{kt}\]
  \end{ex*}
  \begin{extasks}[after-item-skip=\stretch{1}](1)
    \task What does $Q_0$ represent?
    \task What does $k$ represent?
    \task Show that the rate of increase of $Q(t)$ is proportional to the quantity $Q(t)$.
  \end{extasks}
  \vspace*{\stretch{1}}
  \begin{defn*}
    $Q(t)$ is said to exhibit \textbf{Exponential Growth}.
  \end{defn*}
  \pagebreak
    
  \begin{ex*}
    Under ideal laboratory conditions, the number of bacteria in a culture grows in accordance with the law $Q(t)=Q_0e^{kt}$, where $Q_0$ denotes the number of bacteria initially present in the culter, $k$ is a constant determined by the strain of bacteria under consideration and other factors, and $t$ is the elapsed time measured in hours. Suppose $10,000$ bacteria are present initially in the culture and $60,000$ are present $2$ hours later.
  \end{ex*}
  \begin{extasks}[after-item-skip=\stretch{1}](1)
    \task How many bacteria will there be in the culture at the end of $4$ hours?
    \task What is the rate of growth of the population after $4$ hours?
  \end{extasks}
  \vspace*{\stretch{1}}
  \pagebreak
  
  \begin{ex*}
    Radioactive substances decay exponentially. For example, the amount of radium present at any time $t$ obeys the law $Q(t)=Q_0e^{-kt}$, where $Q_0$ is the initial amount present and $k$ is a specific positive constant. The \textbf{half-life of a radioactive substance} is the time required for a given amount to be reduced by one-half. It is known that the half-life of radium is approximately 1600 years. Suppose initially there are 200 milligrams of pure radium.
  \end{ex*}
  \begin{extasks}[after-item-skip=\stretch{1}](1)
    \task What is the amount left after $t$ years? What about $800$ years?
    \task How fast is the amount of radium present after $t$ years? What about $800$ years?
  \end{extasks}
  \vspace*{\stretch{1}}
  \pagebreak

  \begin{ex*}
    Carbon 14, a radioactive isotope of carbon, has a half-life of $5730$ years. What is its decay constant?
  \end{ex*}
  \vspace*{\stretch{1}}
  
  \begin{ex*}
    The Camera Division of Eastman Optical produces a $35$-mm single-lens reflex camera. Eastman's training department determines that after completing the basic training program, a new, previously inexperienced employee will be able to assemble
      \[Q(t)=50-30e^{-0.5t}\]
    model F cameras per day $t$ months after the employee starts work on the assembly line.
  \end{ex*}
  \begin{extasks}[after-item-skip=\stretch{1}](1)
    \task How many model F cameras can a new employee assemble per day after basic training?
    \task How many model F cameras can an employee with 1 month of experience assemble per day? What about 2 months? 6 months?
    \task How many model F cameras can the average experienced employee ultimately be expected to assemble per day?
  \end{extasks}
  \vspace*{\stretch{1}}
  \pagebreak

  \begin{ex*}
    The number of soldiers at Fort MacArthur who contracted influenza after $t$ days during a flu epidemic is approximated by the \emph{logistic model}
      \[Q(t)=\dfrac{5000}{1+1249e^{-kt}}\]
  \end{ex*}
  \begin{extasks}[after-item-skip=\stretch{1}](1)
    \task If $40$ soldiers contracted the flu by day $7$, find how many soldiers contracted the flu by day $15$.
    \task At what rate is the number of soldiers contracting the flue changing on day $15$?
  \end{extasks}
  \vspace*{\stretch{1}}

  \pagebreak
\end{document}
