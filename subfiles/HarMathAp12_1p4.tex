\documentclass[../mathNotesPreamble]{subfiles}

\providecommand{\relscalefact}{1.4}
\begin{document}
\relscale{\relscalefact}
  \section{1.4: Graphs and Graphing Utilities}
  As graphing calculators are \emph{not} required for this course, we will use Desmos:
  \begin{center}
    \href[pdfnewwindow]{https://www.desmos.com/calculator}{\textcolor{blue}{\underline{desmos.com/calculator}}}
  \end{center}

  \hrule\vspace{\baselineskip}
  \begin{ex*}
    For a certain city, the cost $C$ of obtaining drinking water with $p$ percent impurities (by volume) is given by
      \[C=\frac{120,000}{p}-1200\]
    The equation for $C$ requires that $p\neq 0$, and because $p$ is the percent impurities, we know $0<p\leq 100$. Use the restriction on $p$ and a graphing calculator to obtain an accurate graph of the equation.
  \end{ex*}
  \vspace*{\stretch{1}}
  \begin{center}
    \begin{tikzpicture}[scale=1.0, declare function={
      C(\p)=120000/\p-1200;}]
      \begin{groupplot}[
        group style={group size=2 by 1, horizontal sep=2cm},
        axis lines=center,
        axis line style={black,->},
        xmin=-2,
%        enlargelimits={abs=0.75},
        ticklabel style={font=\footnotesize,inner sep=0.5pt,fill=white,opacity=1.0, text opacity=1},
        xlabel=$p$, xlabel style={at={(ticklabel* cs:1)},anchor=north west},
%        ylabel=$y$, ylabel style={at={(ticklabel* cs:1)},anchor=south},
        every axis plot/.append style={line width=0.95pt, color=lander_blue, samples=255}
        ]
        \nextgroupplot[xmax=105,, ymin=-18000]
          \addplot[<->] expression[domain=1:100]{C(x)} node[pos=0.9, above right, black] {$C(p)$};
        \nextgroupplot[xmax=135, ymin=-500, ymax=3500]
          \addplot[<->] expression[domain=27:130]{C(x)} node[pos=0.8, above right, black] {$C(p)$};
      \end{groupplot}
    \end{tikzpicture}
  \end{center}
  \vspace*{\stretch{1}}

  \pagebreak
\end{document}