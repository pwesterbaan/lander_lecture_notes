\documentclass[../mathNotesPreamble]{subfiles}

\providecommand{\relscalefact}{1.4}
\begin{document}
\relscale{\relscalefact}
  \section{4.1: Applications of the First Derivative}
  \begin{defn*}
    Consider the function $f(x)$ on the interval $(a,b)$. Given \emph{any} two numbers $x_1$ and $x_2$ in $(a,b)$ where $x_1<x_2$, we say $f$ is
    \begin{center}
      \begin{tikzpicture}[declare function={
        a=1; b=2; c=3.4; d=1; ell=0.25; t=0.725; eps=0.05;
        f(\x)=(\x-a)*(\x-b)*(\x-c);
        lc(\x,\y,\ell)=(1-\ell)*\x+\ell*\y;
        quad_form(\a,\b,\c,\s)=(-\b+\s*sqrt(\b^2-4*\a*\c))/(2*\a);
        ap=quad_form(3,-2*(a+b+c), a*b+b*c+a*c,-1);
        bp=quad_form(3,-2*(a+b+c), a*b+b*c+a*c,1);
        xo=lc(a,b,t);     xt=lc(c,b,t);
        fo=f(xo);         ft=f(xt);
        xmin=lc(a,b,ell); xmax=lc(c,b,ell);
        ymin=d+f(bp)-eps; ymax=d-f(bp)+eps;
        }]
        \begin{groupplot}[
          group style={group size=2 by 1, horizontal sep=20mm, vertical sep=0cm},
          axis lines=center,
          axis line style={black,->},
          ymin=ymin, ymax=ymax,
          xtick={ap,xo,xt,bp}, xticklabels={$\vphantom{b}a$,$\vphantom{b}x_1$, $\vphantom{b}x_2$,$b$},
          yticklabels={$f(x_1)$, $f(x_2)$},
          enlargelimits={value=0.05, auto},
          ticklabel style={font=\normalsize,inner sep=0.5pt,fill=white,opacity=1.0, text opacity=1},
          every axis plot/.append style={domain=xmin:xmax,line width=0.95pt, color=lander_blue, samples=255},
          ]
            \nextgroupplot[ytick={d-fo,d-ft}, title=increasing if $f(x_1)<f(x_2)$]
              \addplot[<->] {d-f(x)};
              \addplot[soldot] coordinates{(xo,{d-f(xo)})(xt,{d-f(xt)})};
              \draw[dashed] (axis cs: xo,0) |- (axis cs: 0,d-fo)
                            (axis cs: xt,0) |- (axis cs: 0,d-ft);
            \nextgroupplot[ytick={d+fo,d+ft}, title=decreasing if $f(x_1)>f(x_2)$]
              \addplot[<->] {d+f(x)};
              \addplot[soldot] coordinates{(xo,{d+f(xo)})(xt,{d+f(xt)})};
              \draw[dashed] (axis cs: xo,0) |- (axis cs: 0,d+fo)
                            (axis cs: xt,0) |- (axis cs: 0,d+ft);
        \end{groupplot}
      \end{tikzpicture}
    \end{center}
    Thus, for every value of $x$ on the interval $(a,b)$, if
    \begin{enumerate}[--]
      \item $f'(x)>0$, then $f$ is increasing on $(a,b)$.
      \item $f'(x)<0$, then $f$ is decreasing on $(a,b)$.
      \item $f'(x)=0$, then $f$ is constant on $(a,b)$.
    \end{enumerate}
  \end{defn*}
  \begin{ex*}
    Find the intervals where $f(x)=x^2$ is increasing and decreasing.
  \end{ex*}
  \pagebreak

  \begin{thmBox*}[Determining intervals where a function is increasing or decreasing.]
    \begin{enumerate}
      \item Find all values of $x$ such that $f'(x)=0$ or $f'(x)$ is undefined.
      \item Determine the sign of $f'(x)$ on each open interval.
    \end{enumerate}
  \end{thmBox*}
  \begin{ex*}
    Suppose that $f$ is continuous everywhere and
      \[f'(x)=\dfrac{(x-1)(x+2)}{(x-4)^2(x+5)}.\]
    We see that $f'(-2)=f(1)=0$ and $f(-5)$ and $f(4)$ are undefined. Complete a sign chart to show where $f(x)$ is increasing and decreasing.
  \end{ex*}
  \pagebreak
  
  \begin{ex*}
    Find the intervals where the following functions are increasing and decreasing:
  \end{ex*}
  \begin{extasks}[after-item-skip=\stretch{1}](1)
    \task $f(x)=x^3-3x^2-24x+32$ \hfill \href[pdfnewwindow]{https://www.desmos.com/calculator/re1imdb5c4}{\textcolor{blue}{\underline{Graph}}}
    \task $g(x)=(x+1)^{2/3}$
  \end{extasks}
  \vspace*{\stretch{1}}
  \pagebreak

  \begin{extasks}[after-item-skip=\stretch{1}](1)
    \task $h(x)=x+\dfrac{1}{x}$
    \task $j(x)=\dfrac{x^2}{1-x^2}$
  \end{extasks}
  \vspace*{\stretch{1}}
  \pagebreak

  \begin{defn*}[Relative Extrema]
    A function $f$ has a
    \begin{itemize}
%      \item
%        \textbf{relative maximum} at $x=c$ if there exists an open interval $(a,b)$ containing $c$ such that $f(x)\leq f(c)$ for all $x$ in $(a,b)$.
%      \item
%        \textbf{relative minimum} at $x=c$ if there exists an open interval $(a,b)$ containing $c$ such that $f(x)\geq f(c)$ for all $x$ in $(a,b)$.
      \item
        \textbf{relative maximum} at $x=c$ if $f(c)\geq f(x)$ for every $x$ in $(a,b)$
      \item
%        $f(c)\leq f(x)$ for every $x$ in $(a,b)$, then $f(c)$ is a \textbf{relative minimum} value of $f$.
        \textbf{relative minimum} at $x=c$ if $f(c)\leq f(x)$ for every $x$ in $(a,b)$
    \end{itemize}
  \end{defn*}

  \begin{center}
    \begin{tikzpicture}[
      declare function={
        cubeRoot(\x)=\x/abs(\x)*abs(\x)^(1/3);
        f(\x)=\x<=7.81901? -(\x-6.81901)^7+1: 1.5*cubeRoot(\x-7.81901);},
      custNode/.style={
        align=center,
        color=black,
        inner sep=0.25pt,
        fill=white,
        opacity=0.7,
        text opacity=1}]
      \begin{axis}[
        axis lines=center,
        axis line style={black,->},
        xmajorticks=false,
        ymajorticks=false,
        height=\axisdefaultheight, width=0.9\linewidth,
        enlargelimits={value=0.025, auto},
        ticklabel style={font=\footnotesize,inner sep=0.5pt,fill=white,opacity=1.0, text opacity=1},
        every axis plot/.append style={line width=0.95pt, color=lander_blue, samples=255},
        clip=false,
        ]
        \addplot[<->, smooth, tension=0.85] coordinates{(-0.25,-1)(0.25,0.75)(1,4)(2.5,1)(4,2)(5,-1)(5.5,5)};
        \addplot[<->] expression[domain=5.6:9] {f(x};
        \draw[black, dashed] (axis cs: 5.55,-1.4) -- (axis cs: 5.55,5);
        \node[custNode, above, yshift= 5pt] at (1,4)       {relative\\max};
        \node[custNode, below, yshift=-5pt] at (2.7,1)     {relative\\min};
        \node[custNode, above, yshift= 5pt] at (3.8,2)     {relative\\max};
        \node[custNode, below, yshift=-5pt] at (4.9,-1)    {relative\\min};
        \node[custNode, below, yshift=-5pt] at (7.81901,0) {relative\\min};
      \end{axis}
    \end{tikzpicture}
  \end{center}
  \begin{defn*}
    A \textbf{critical point} of a function $f$ is any number $x$ in the domain of $f$ such that $f'(x)=0$ or $f'(x)$ does not exist.
  \end{defn*}
  \pagebreak

  \begin{thmBox*}[Procedure for Finding the Relative Extrema of a Continuous Function $f$]
    \textbf{The First Derivative Test:}
    \begin{enumerate}
      \item Determine the critical points of $f$.
      \item Determine the sign change of $f'(x)$ to the left and right of each critical point:\\
        If, at $x=c$, $f'(x)$ \dots
        \begin{enumerate}
          \item changes sign from \emph{positive} to \emph{negative}, then $f$ has a \emph{relative maximum} \hfill
            \tikz{\begin{axis}[axis lines=none, width=27.5mm, height=20mm]
              \addplot[<->, lander_blue] expression[domain=-1:1]{-x^2};
            \end{axis}}
          \item changes sign from \emph{negative} to \emph{positive}, then $f$ has a \emph{relative minimum} \hfill
            \tikz{\begin{axis}[axis lines=none, width=27.5mm, height=20mm]
              \addplot[<->, lander_blue] expression[domain=-1:1]{x^2};
            \end{axis}}
          \item does not change sign, then $f$ does \emph{not} have a relative extremum
        \end{enumerate}
        at $x=c$.
    \end{enumerate}
  \end{thmBox*}

  \begin{ex*}
    Consider the function $f(x)=6x-x^3$. \hfill \textcolor{blue}{\underline{\href[pdfnewwindow]{https://www.desmos.com/calculator/ayiubdaov9}{Graph}}}
  \end{ex*}
  \begin{extasks}[after-item-skip=\stretch{1}](1)
    \task Use $f'(x)$ to find the intervals on which the function is increasing and decreasing.
    \task Identify the function's local extreme values \hfill
      (e.g. ``local max of \underline{\hspace*{15pt}} at $x=\underline{\hspace*{15pt}}$'')
  \end{extasks}
  \vspace*{\stretch{1}}
  \pagebreak

  \begin{ex*}
    Find the relative maximums/relative minimums of the following:
  \end{ex*}
  \begin{extasks}[after-item-skip=\stretch{1}](1)
    \task $f(x)=x^3-3x^2-24x+32$ \hfill \href[pdfnewwindow]{https://www.desmos.com/calculator/re1imdb5c4}{\textcolor{blue}{\underline{Graph}}}
    \task $g(x)=(x+1)^{2/3}$
  \end{extasks}
  \vspace*{\stretch{1}}
  \pagebreak

  \begin{extasks}[after-item-skip=\stretch{1}](1)
    \task $h(x)=x+\dfrac{1}{x}$
    \task $j(x)=\dfrac{x^2}{1-x^2}$
  \end{extasks}
  \vspace*{\stretch{1}}

  \pagebreak
\end{document}
