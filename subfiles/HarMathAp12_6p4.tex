\documentclass[../mathNotesPreamble]{subfiles}

\providecommand{\relscalefact}{1.4}
\begin{document}
\relscale{\relscalefact}
  \section{6.4: Present Values of Annuities}
%    \begin{ex*}
%      Suppose that we invest $\$100$ at the end of each year for $5$ years in an account that pays $10\%$ compounded annually. How much money will you have at the end of the $5$ years?
%    \end{ex*}
%
    \begin{ex*}
      Suppose we wish to invest a lump sum of money, $A_n$, into an annuity that earns interest at a rate of $10\%$ per year, so that we may receive payments of $\$100$ for $5$ years. What is the amount of the lump sum?
    \end{ex*}
    \vspace*{\stretch{1}}
    \begin{defn*}
      If a payment of $\$R$ is to be withdrawn at the \emph{end of each period} for $n$ periods from an account that earns interest at a rate of $i$ per period, then the account is an \textbf{ordinary annuity} and the \textbf{present value} is
        \[A_n=R\cdot a_{\angln i}=R\sbrkt{\frac{1-\parens{1+i}^{-n}}{i}}\]
      The notation $a_{\angln i}$ represents the present value of an ordinary annuity of $\$1$ per period for $n$ periods with an interest rate of $i$ per period.
    \end{defn*}
    \pagebreak

    \begin{ex*}
      Find the lump sum that one must invest in an annuity to receive $\$1000$ at the end of each month for the next 16 years, if the annuity pays $9\%$ compounded monthly.
    \end{ex*}
    \vspace*{\stretch{1}}
    
    \begin{ex*}
      Suppose that a couple plans to set up an ordinary annuity with a $\$100,000$ inheritance they received. What is the size of the quarterly payments they will receive for the next $6$ years if the account pays $7\%$ compounded quarterly?
    \end{ex*}
    \vspace*{\stretch{1}}
    \pagebreak
    
    \begin{ex*}
      An inheritance of $\$250,000$ is invested at $9\%$ compounded monthly. If $\$2500$ is withdrawn at the end of each month, how long will it be until the account balance is $\$0$?
    \end{ex*}
    \vspace*{\stretch{1}}
    \pagebreak

    \begin{defn*}
      If a payment of $\$R$ is to be withdrawn at the \emph{beginning of each period} for $n$ periods from an account that earns interest at a rate of $i$ per period, then the account is an \textbf{annuity due} and the \textbf{present value} is
        \[A_{(n,\textnormal{due})}=R\cdot a_{\angln i}(1+i)=R\sbrkt{\frac{1-\parens{1+i}^{-n}}{i}}(1+i)\]
      The notation $a_{\angln i}$ represents the present value of an ordinary annuity of $\$1$ per period for $n$ periods with an interest rate of $i$ per period.
    \end{defn*}
    \begin{ex*}
      Suppose that a court settlement results in a $\$750,000$ award. If this is invested at $9\%$ compounded semiannually, how much will it provide at the \emph{beginning} of each half-year for a period of $7$ years?
    \end{ex*}
    \vspace*{\stretch{1}}
    \pagebreak
    
    \noindent
    \textbf{Comparing annuity calculations:}
    \begin{center}
      \begin{tabularx}{0.75\linewidth}{@{}
        >{\hsize=\hsize}X
        >{\hsize=0.75\hsize}Y
        >{\hsize=0.75\hsize}Y
        >{\hsize=1.5\hsize}Z
        @{}}
        & Ordinary annuities& Annuity due\\\midrule
        Future Value& $S$& $S_{\textnormal{due}}$& \lnret[r@{}]{Regular payments\\ Total \emph{after} $n$ periods}\\[20pt]
        Present Value& $A_n$& $A_{(n,\textnormal{due})}$& \lnret[r@{}]{Regular withdrawals\\ Total \emph{before} $n$ periods}\\\midrule
        & end of each period& beginning of each period
      \end{tabularx}
    \end{center}

  \pagebreak
\end{document}
