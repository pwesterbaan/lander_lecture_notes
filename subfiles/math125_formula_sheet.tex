\documentclass{article}
\usepackage[margin=1in]{geometry}
\usepackage{amsmath, tabularx, xfrac, actuarialsymbol, mathtools}
\usepackage{xcolor}
\usepackage{booktabs}
\usepackage{colorPalette}
\newcommand*{\sbrkt}[1]{{\left[#1\right]}}
\newcommand*{\parens}[2][]{{\left(\vphantom{#1[}#2\right)}}

\newcount\examUnit
\examUnit=-1 %TODO (based on chapter number)
\begin{document}
  \providecommand{\formulaSheetTitle}[1]{{\Large \textbf{#1}}}
  \definecolor{greenish}{rgb}{0,0.5,0}
  \newcommand{\examOne}{%
    stuff& things\\
  }
  \newcommand{\examTwo}{%
  }
  \newcommand{\examThree}{%
  }
  \begin{center}
    \formulaSheetTitle{Math 125 Formula Sheet}

    \renewcommand{\arraystretch}{1.755}
    \begin{tabularx}{0.95\linewidth}{@{}>{\bfseries\hsize=1.075\hsize}X>{\hsize=0.925\hsize}X@{}}
      \ifnum\examUnit=1 \examOne   \fi%
      \ifnum\examUnit=2 \examTwo   \fi%
      \ifnum\examUnit=3 \examThree \fi%
      \ifnum\examUnit=-1 %%Final Exam
%        \multicolumn{2}{c}{\hrulefill\textbf{ Linear equations and inequalities }\hrulefill}\\ \examOne
%        \multicolumn{2}{c}{\hrulefill\textbf{ Logs and exponentials }\hrulefill}\\ \examTwo
%        \multicolumn{2}{c}{\hrulefill\textbf{ Finance }\hrulefill}\\ \examThree
      \fi%
    \end{tabularx}
  \end{center}

  \pagebreak
\end{document}