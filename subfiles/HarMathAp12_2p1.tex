\documentclass[../mathNotesPreamble]{subfiles}

\providecommand{\relscalefact}{1.4}
\begin{document}
\relscale{\relscalefact}
  \section{2.1: Quadratic Equations}
    \begin{defn*}
      A \textbf{quadratic equation} in one variable is an equation of second degree that can be written in the \emph{general form} as
        \[ax^2+bx+c=0 \quad (a\neq 0)\]
      where $a$, $b$, and $c$ represent constants.
      \vspace*{\baselineskip}

      The \textbf{zero product property} states that for real numbers $a$ and $b$, $ab=0$ if and only if $a=0$ or $b=0$ or both.
    \end{defn*}

    \begin{ex*}
      Solve the following for $x$:
    \end{ex*}
    \begin{extasks}[after-item-skip=\stretch{1}](2)
      \task $x(x+3)=0$
      \task $(x-4)(3x+1)=0$
    \end{extasks}
    \vspace*{\stretch{1}}

    \pagebreak
    \begin{center}
      \fbox{\parbox{0.9875\linewidth}{
        \noindent\textbf{Solving quadratic equations via factoring:}
        \begin{ex*}
          Solve $2x^2+x=3x+12$
          \begin{align*}
            &\textnormal{\parbox{0.5\linewidth}{1. Rewrite the equation in the general form:}}&
              2x^2-2x-12&=0\\[5mm]
            &\textnormal{\parbox{0.5\linewidth}{2. Rewrite $bx$ using factors of $ac$:}}&
              2x^2\textcolor{red}{-6x+4x}-12&=0 \\[5mm]
            &\textnormal{\parbox{0.5\linewidth}{3. Factor out like terms:}}&
              2x(x-3)+4(x-3)&=0 \\[5mm]
            &\textnormal{\parbox{0.5\linewidth}{4. Factor by grouping:}}&
              (x-3)(2x+4)&=0 \\[5mm]
            &\textnormal{\parbox{0.5\linewidth}{5. Solve for the roots:}}&
              x=3 \textnormal{ and } x=-2
          \end{align*}
        \end{ex*}
      }}
    \end{center}

    \begin{ex*}
      Solve the following for $x$ via factoring:
    \end{ex*}
    \begin{extasks}[after-item-skip=\stretch{1}](2)
      \task $(x+3)(x-1)=5$
      \task $-4x^2+8x-3=0$
    \end{extasks}
    \vspace*{\stretch{1}}
    \pagebreak

    \begin{center}
      \fbox{\parbox{0.5\linewidth}{ \centering
        Solutions to $x^2=C$ are $x=\pm \sqrt{C}$
      }}
    \end{center}

    \begin{ex*}
      Solve the following:
    \end{ex*}
    \begin{extasks}[after-item-skip=\stretch{1}](2)
      \task $(x-1)^2=9$
      \task $4x^2-1=0$
    \end{extasks}
    \vspace*{\stretch{1}}

    \begin{defn*}
      The \textbf{quadratic formula}
        \[x=\dfrac{-b\pm\sqrt{b^2-4ac}}{2a}\]
      gives the solutions to $ax^2+bx+c=0$.\\[\baselineskip]
      Quadratic equations can have one, two, or no solutions. The \textbf{discriminant} is $b^2-4ac$:
      \begin{itemize}
        \item $b^2-4ac>0$: The equation has \emph{exactly} two distinct \emph{real} solutions.
        \item $b^2-4ac=0$: The equation has \emph{exactly} one \emph{real} solution.
        \item $b^2-4ac<0$: The equation has no \emph{real} solutions.
      \end{itemize}
    \end{defn*}
    \pagebreak

    \begin{ex*}
      Suppose some hooligans kick a ball up in the air off the roof of the library. Assuming the height, in $ft$, of the ball $t$ seconds after kicking it is given by
        \[h(t)=-32t^2+64t+40\]
      Solve for $t$ when
    \end{ex*}
    \begin{extasks}[after-item-skip=\stretch{1}](1)
      \task the ball is $80$ feet off of the ground
      \task the ball is $72$ feet off of the ground
      \task the ball is $40$ feet off of the ground
      \task the ball hits the ground
    \end{extasks}
    \vspace{\stretch{1}}
    \pagebreak

    \begin{ex*}
      The Social Security Trust Fund balance $B$, in billions of dollars, can be described by the function
        $B=-7.97t^2+312t-356$
      where $t$ is the number of years past the year 1995. For planning purposes, it is important to know when the trust fund balance will be $0$. Solve
        \[0=-7.97t^2+312t-356.\]
    \end{ex*}

  \pagebreak
\end{document}
