\documentclass[../mathNotesPreamble]{subfiles}

\begin{document}
%  \relscale{1.4} %TODO
  \section{4.2: Linear Programming: Graphical Methods}
    \begin{defn*}
      Linear programming is an optimization technique that can be used to solve linearly constrained problems:
      \begin{alignat*}{4}
%        r&l&r&l&r&l&r&l\\
        \max F&=&\,3x&+&y\\
        \textnormal{subject to }&& x&+&5y&\leq &\,50\\
          &&2x&+&y&\leq& 28\\
          &&&&\mathllap{x\geq 0}, y&\geq& 0
      \end{alignat*}
      
      The \textbf{constraints} of a linear program (LP) may be limitations or requirements of the variables. The \textbf{objective function} is the function that we wish to optimize (e.g. maximize profit \emph{or} minimize cost).
    \end{defn*}
    \vspace*{\stretch{1}}
    
    \begin{center}
      \begin{tikzpicture}[declare function={
        f(\x)=-\x/5+10; g(\x)=-2*\x+28;
        h(\x)=f(\x)<=g(\x) ? f(\x) : g(\x);}]
        \begin{axis}[
          grid=both, %major,minor
          grid style={line width=0.3pt, draw=gray!35},
          major grid style={line width=0.375pt, draw=gray!75},
          axis lines=center,
          axis line style={black,->},
          xmin=-1.5, xmax=22.5,
          ymin=-1.5, ymax=11.5,
          width=0.625\linewidth,
          ticklabel style={font=\footnotesize,inner sep=0.5pt,fill=white,opacity=0.5, text opacity=1},
          xlabel=$x$, xlabel style={at={(ticklabel* cs:1)},anchor=north west},
          ylabel=$y$, ylabel style={at={(ticklabel* cs:1)},anchor=south west},
          every axis plot/.append style={line width=0.95pt, color=lander_blue, samples=255}]
            \draw[fill=lander_blue!50, opacity=0.5] plot[smooth, samples=100, domain=0:14] (\x,{h(\x)}) -| (0,0) -- cycle;
            \addplot[<->] expression[domain=-1.25:16]{f(x)}
              node[pos=0.95, above right, black, fill=white, opacity=0.75, text opacity=1.0, font=\large, inner sep=2pt] {$x+5y=50$};
            \addplot[<->] expression[domain=8.5:14.5]{g(x)}
              node[pos=0.025, right, xshift=5pt, black, fill=white, opacity=0.75, text opacity=1.0, font=\large, inner sep=2pt] {$2x+y=28$};
            \addplot[soldot] coordinates{(10,8)} node[above right, black, fill=white, opacity=0.75, text opacity=1.0, inner sep=2pt, font=\normalsize] {$(10,8)$};
            \addplot[soldot] coordinates{(14,0)} node[above right, black, fill=white, opacity=0.75, text opacity=1.0, inner sep=2pt, font=\normalsize] {$(14,0)$};
            \addplot[soldot] coordinates{(0,10)} node[above right, black, fill=white, opacity=0.75, text opacity=1.0, inner sep=2pt, font=\normalsize] {$(0,10)$};
        \end{axis}
      \end{tikzpicture}
    \end{center}
  \pagebreak
  
  \noindent \textbf{Linear programming (graphical method)}
  
  \begin{center}
    \fbox{\parbox{0.95\linewidth}{
      \begin{enumerate}
        \item Write the objective function and constraint inequalities from the problem.
        \item Graph the solution of the constraint system.
        \item Find the corners of the resulting feasible region.
        \item Evaluate the objective function at each corner.
        \item If two corners give the optimal value, then the entire boundary joining these two points optimizes the function.
      \end{enumerate}
    }}
  \end{center}
  \begin{ex*}
    A farm co-op has 6000 acres available on which to plant corn and soybeans. The following table summarizes each crop's requirement for fertilizer/herbicide, harvesting labor hours, and the available amounts of these resources.
    \begin{center}
      \begin{tabular}{@{}lccc@{}}\toprule
        & \textbf{Corn}& \textbf{Soybeans}& \textbf{Available}\\\midrule
        Fertilizer/herbicide& 9 gal/acre& 3 gal/acre& 40,500 gal\\
        Harvesting labor& 3/4 hr/acre& 1 hr/acre& 5,250 hr\\\bottomrule
      \end{tabular}
    \end{center}
    Setup the system of inequalities that represents the constraints.
  \end{ex*}
  
  \vspace*{\stretch{1}}
  \begin{flushright}
    \begin{tikzpicture}[declare function={
        f(\x)=-3*\x+13500; g(\x)=-0.75*\x+5250; h(\x)=-\x+6000;
        j(\x)=f(\x)<=h(\x) ? f(\x) : k(\x);
        k(\x)=h(\x)<=g(\x) ? h(\x) : g(\x);}]
        \begin{axis}[
          grid=both, %major,minor
          grid style={line width=0.3pt, draw=gray!35},
          major grid style={line width=0.375pt, draw=gray!75},
          axis lines=center,
          axis line style={black,->},
          xmin=-15, xmax=7400,
          ymin=-15, ymax=5500,
          ticklabel style={font=\footnotesize,inner sep=0.5pt,fill=white,opacity=0.5, text opacity=1},
          xlabel=$x$, xlabel style={at={(ticklabel* cs:1)},anchor=north west},
          ylabel=$y$, ylabel style={at={(ticklabel* cs:1)},anchor=south west},
          every axis plot/.append style={line width=0.95pt, color=lander_blue, samples=255}]
            \draw[fill=lander_blue!50, opacity=0.5] plot[smooth, samples=255, domain=0:7000] (\x,{j(\x)}) -| (0,0) -- cycle;
            \addplot[<->] expression[domain=900:6000]{h(x)}
              node[pos=0.2, sloped, above, black, fill=white, opacity=0.75, text opacity=1.0, font=\large, inner sep=2pt] {$x+y=6,000$};
            \addplot[<->] expression[domain=2750:4500]{f(x)}
              node[pos=0.225, above right, black, fill=white, opacity=0.75, text opacity=1.0, font=\large, inner sep=2pt] {$9x+3y=40,500$};
            \addplot[<->] expression[domain=0:7000]{g(x)}
              node[pos=0.77, sloped, above, black, fill=white, opacity=0.75, text opacity=1.0, font=\large, inner sep=2pt] {$\frac{3}{4}x+y=5,250$};
        \end{axis}
      \end{tikzpicture}
  \end{flushright}
  \pagebreak
  
  \begin{ex*}
    Using the linear constraints from above, suppose the co-ops profits per acre are \$240 for corn and \$160 for soybeans. This gives us the following linear program:
      \begin{alignat*}{4}
%        r&l&r&l&r&l&r&l\\
        \max P&=&\,240x&+&160y\\
        \textnormal{subject to }&& x&+&y&\leq& 6,000\\
          &&9x&+&3y&\leq &\,40,500\\
          &&\frac{3}{4}x&+&y&\leq& 5,250\\
          &&&&\mathllap{x\geq 0}, y&\geq\mathrlap{\,0}& 
      \end{alignat*}
    \begin{enumerate}
      \item Find the ``corners'' of the feasible region
      \item Evaluate the profit function at the corners
    \end{enumerate}
  \end{ex*}
  
  \vspace*{\stretch{1}}
  \begin{minipage}{0.45\linewidth}
    \begin{tabular}{@{}lr@{}}
      $(x,y)$& $P=240x+160y$\\\midrule
      $(0,0)$& \$0\\
      $(0,5250)$& \$840,000\\
      $(3000,3000)$& \$1,200,000\\
      $(3750,2250)$& \$1,260,000\\
      $(4500,0)$& \$1,080,000\\\bottomrule
    \end{tabular}
  \end{minipage}\hspace*{\stretch{1}}%
  \begin{minipage}{0.45\linewidth}
    \begin{flushright}
     \begin{tikzpicture}[declare function={
        f(\x)=-3*\x+13500; g(\x)=-0.75*\x+5250; h(\x)=-\x+6000;
        j(\x)=f(\x)<=h(\x) ? f(\x) : k(\x);
        k(\x)=h(\x)<=g(\x) ? h(\x) : g(\x);}]
          \begin{axis}[
            grid=both, %major,minor
            grid style={line width=0.3pt, draw=gray!35},
            major grid style={line width=0.375pt, draw=gray!75},
            axis lines=center,
            axis line style={black,->},
            xmin=-15, xmax=7400,
            ymin=-15, ymax=5500,
            ticklabel style={font=\footnotesize,inner sep=0.5pt,fill=white,opacity=0.5, text opacity=1},
            xlabel=$x$, xlabel style={at={(ticklabel* cs:1)},anchor=north west},
            ylabel=$y$, ylabel style={at={(ticklabel* cs:1)},anchor=south west},
            every axis plot/.append style={line width=0.95pt, color=lander_blue, samples=255}]
              \draw[fill=lander_blue!50, opacity=0.5] plot[smooth, samples=100, domain=0:4500] (\x,{j(\x)}) -| (0,0) -- cycle;
              \addplot[<->] expression[domain=900:6000]{h(x)}
                node[pos=0.2, sloped, above, black, fill=white, opacity=0.75, text opacity=1.0, font=\large, inner sep=2pt] {$x+y=6,000$};
              \addplot[<->] expression[domain=2750:4500]{f(x)}
                node[pos=0.225, above right, black, fill=white, opacity=0.75, text opacity=1.0, font=\large, inner sep=2pt] {$9x+3y=40,500$};
              \addplot[<->] expression[domain=0:7000]{g(x)}
                node[pos=0.77, sloped, above, black, fill=white, opacity=0.75, text opacity=1.0, font=\large, inner sep=2pt] {$\frac{3}{4}x+y=5,250$};
          \end{axis}
        \end{tikzpicture}
    \end{flushright}
  \end{minipage}
  \pagebreak
  
  \begin{ex*}
    Suppose the profits per acre are instead \$300 for corn, and \$100 for soybeans. This gives us the following linear program:
      \begin{alignat*}{4}
%        r&l&r&l&r&l&r&l\\
        \max P&=&\,300x&+&100y\\
        \textnormal{subject to }&& x&+&y&\leq& 6,000\\
          &&9x&+&3y&\leq &\,40,500\\
          &&\frac{3}{4}x&+&y&\leq& 5,250\\
          &&&&\mathllap{x\geq 0}, y&\geq\mathrlap{\,0}& 
      \end{alignat*}
    Evaluate the profit function at the corners. What combination of corn and soy-beans maximizes the profit?
  \end{ex*}
  
  \vspace*{\stretch{1}}
  \begin{minipage}{0.45\linewidth}
    \begin{tabular}{@{}lr@{}}
      $(x,y)$& $P=300x+100y$\\\midrule
      $(0,0)$& \$0\\
      $(0,5250)$& \$525,000\\
      $(3000,3000)$& \$1,200,000\\
      $(3750,2250)$& \$1,350,000\\
      $(4500,0)$& \$1,350,000\\\bottomrule
    \end{tabular}
  \end{minipage}\hspace*{\stretch{1}}%
  \begin{minipage}{0.45\linewidth}
    \begin{flushright}
     \begin{tikzpicture}[declare function={
        f(\x)=-3*\x+13500; g(\x)=-0.75*\x+5250; h(\x)=-\x+6000;
        j(\x)=f(\x)<=h(\x) ? f(\x) : (h(\x)<=g(\x) ? h(\x) : g(\x));}]
          \begin{axis}[
            grid=both, %major,minor
            grid style={line width=0.3pt, draw=gray!35},
            major grid style={line width=0.375pt, draw=gray!75},
            axis lines=center,
            axis line style={black,->},
            xmin=-15, xmax=7400,
            ymin=-15, ymax=5500,
            ticklabel style={font=\footnotesize,inner sep=0.5pt,fill=white,opacity=0.5, text opacity=1},
            xlabel=$x$, xlabel style={at={(ticklabel* cs:1)},anchor=north west},
            ylabel=$y$, ylabel style={at={(ticklabel* cs:1)},anchor=south west},
            every axis plot/.append style={line width=0.95pt, color=lander_blue, samples=255}]
              \draw[fill=lander_blue!50, opacity=0.5] plot[smooth, samples=100, domain=0:4500] (\x,{j(\x)}) -| (0,0) -- cycle;
              \addplot[<->] expression[domain=900:6000]{h(x)}
                node[pos=0.2, sloped, above, black, fill=white, opacity=0.75, text opacity=1.0, font=\large, inner sep=2pt] {$x+y=6,000$};
              \addplot[<->] expression[domain=2750:4500]{f(x)}
                node[pos=0.225, above right, black, fill=white, opacity=0.75, text opacity=1.0, font=\large, inner sep=2pt] {$9x+3y=40,500$};
              \addplot[<->] expression[domain=0:7000]{g(x)}
                node[pos=0.77, sloped, above, black, fill=white, opacity=0.75, text opacity=1.0, font=\large, inner sep=2pt] {$\frac{3}{4}x+y=5,250$};
          \end{axis}
        \end{tikzpicture}
    \end{flushright}
  \end{minipage}
  \pagebreak
  
  \begin{ex*}
    Two chemical plants, one at Macon and one at Jonesboro, produce three types of fertilizer: low phosphorus (LP), medium phosphorus (MP), and high phosphorus (HP). At each plant, the fertilizer is produced in a single production run, so the three types are produced in fixed proportions. The Macon plant produces 1 ton of LP, 2 tons of MP, and 3 tons of HP in a single operation and charges \$600 for what is produced in one operation. On the other hand, one operation of the Jonesboro plant produces 1 ton of LP, 5 tons of MP, and 1 ton of HP, and it charges \$1,000 for what it produces in one operation. If a customer needs 100 tons of LP, 260 tons of MP, and 180 tons of HP, how many production runs should be ordered from each plant to minimize costs?
  \end{ex*}
  \begin{extasks}[after-item-skip=3\baselineskip](1)
    \task Organize the information from the problem in the following table:

      \begin{tabularx}{0.8\linewidth}{@{}*{4}{X}@{}}\toprule
        & Macon& Jonesboro& Requirements\\\midrule
        Units of LP \\
        Units of MP \\
        Units of HP \\\bottomrule
      \end{tabularx}
    \task What is the objective function?
    \task Write the linear program that we aim to solve below:
  \end{extasks}
  \pagebreak
  
  \begin{ex*}
    From above, we get the following linear program
    \begin{alignat*}{4}
%      r&l&r&l&r&l&r&l\\
      \min C&=&\,600x&+&1000y\\
      \textnormal{subject to }&& x&+&y&\geq& 100\\
        &&2x&+&5y&\leq &\,260\\
        &&3x&+&y&\leq& 180\\
        &&&&\mathllap{x\geq 0}, y&\geq\mathrlap{\,0}& 
    \end{alignat*}
    Graph the solution region of the constraints and evaluate the objective function at the corners of the feasible region above.
  \end{ex*}

  \vspace*{\stretch{1}}
  \begin{minipage}{0.45\linewidth}
    \begin{tabular}{@{}lr@{}}
      $(x,y)$& $P=600x+1,000y$\\\midrule
      $(0,180)$& $\$180,000$\\
      $(40,60)$& $\$84,000$\\
      $(80,20)$& $\$68,000$\\
      $(130,0)$& $\$78,000$\\\bottomrule
    \end{tabular}
  \end{minipage}\hspace*{\stretch{1}}%
  \begin{minipage}{0.45\linewidth}
    \begin{tikzpicture}[declare function={
      X=200;
      f(\x)=-3*\x+180;  
      g(\x)=-\x+100;
      h(\x)=-0.4*\x+52;
      j(\x)=f(\x)>=g(\x) ? f(\x) : (h(\x)>=g(\x) ? h(\x) : g(\x));}]
        \begin{axis}[
          grid=both, %major,minor
          grid style={line width=0.3pt, draw=gray!35},
          major grid style={line width=0.375pt, draw=gray!75},
          axis lines=center,
          axis line style={black,->},
          enlargelimits={value=0.025, auto},
          ticklabel style={font=\footnotesize,inner sep=0.5pt,fill=white,opacity=0.5, text opacity=1},
          xlabel=$x$, xlabel style={at={(ticklabel* cs:1)},anchor=north west},
          ylabel=$y$, ylabel style={at={(ticklabel* cs:1)},anchor=south west},
          every axis plot/.append style={line width=0.95pt, color=lander_blue, samples=255}]
            \draw[fill=lander_blue!50, opacity=0.5] plot[smooth, samples=100, domain=0:130] (\x,{j(\x)}) -- (X,0) -- (X,X) -| cycle;
            \addplot[<->] expression[domain=0:60]{f(x)}
              node[pos=0.2, above right, black, fill=white, opacity=0.75, text opacity=1.0, font=\large, inner sep=2pt, rounded corners=3pt] {$x+y=100$};
            \addplot[<->] expression[domain=0:100]{g(x)}
              node[pos=0.475, above right, black, fill=white, opacity=0.75, text opacity=1.0, font=\large, inner sep=2pt, rounded corners=3pt] {$2x+5y=260$};
            \addplot[<->] expression[domain=0:130]{h(x)}
              node[pos=0.64, above right, black, fill=white, opacity=0.75, text opacity=1.0, font=\large, inner sep=2pt, rounded corners=3pt] {$3x+y=180$};
      \end{axis}
    \end{tikzpicture}
  \end{minipage}
  \pagebreak
\end{document}
