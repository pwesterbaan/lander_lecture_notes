\documentclass[../mathNotesPreamble]{subfiles}

\providecommand{\relscalefact}{1.4}
\begin{document}
\relscale{\relscalefact}
  \section{6.4: Boolean Algebras, Russell's Paradox, and the Halting Problem}

  \begin{defn*}
    A \textbf{Boolean algebra} is a set $B$ together with two operations, generally denoted $+$ and $\cdot$\,, such that for all $a$ and $b$ in $B$ both $a+b$ and $a\cdot b$ are in $B$ and the following axioms are assumed to hold: 
    \begin{enumerate}
      \item \textit{Commutative Laws:} For all $a$ and $b$ in $B$,
        \[a+b=b+a \textnormal{ and } a\cdot b=b\cdot a\]
      \item \textit{Associative Laws:} For all $a$ and $b$ in $B$,
        \[(a+b)+c=a+(b+c) \textnormal{ and } (a\cdot b)\cdot c=a\cdot(b\cdot c)\]
      \item \textit{Distributive Laws:} For all $a$ and $b$ in $B$,
        \[a+(b\cdot c)=(a+b)\cdot(a+c) \textnormal{ and } a\cdot(b+c)=(a\cdot b)+(a\cdot c)\]
      \item \textit{Identity Laws:} There exist distinct elements $0$ and $1$ in $B$ such that for each $a$ in $B$,
        \[a+0=a \textnormal{ and } a\cdot 1=a\]
      \item \textit{Complement Laws:} For each $a$ in $B$, there exists an element in $B$, denoted $\overline{a}$ and called the \textbf{complement} or \textbf{negation} of $a$, such that
        \[a+\overline{a}=1 \textnormal{ and } a\cdot\overline{a}=0\]
    \end{enumerate}
  \end{defn*}
  \pagebreak

  \begin{thmBox*}[Properties of a Boolean Algebra]
    Let $B$ be any Boolean algebra.
    \begin{enumerate}
      \item \textit{Uniqueness of the Complement Laws:} For all $a$ and $x$ in $B$, if $a+x=1$ and $a\cdot x=0$, then $x=\overline{a}$.
      \item \textit{Uniqueness of $0$ and $1$:} If there exists $x$ in $B$ such that $a+x=a$ for every $a$ in $B$, then $x=0$, and if there exists $y$ in $B$ such that $a\cdot y=a$ for every $a$ in $B$, then $y=1$.
      \item \textit{Double Complement Law:} For every $a\in B$, $\overline{(\overline{a})}=a$.
      \item \textit{Idempotent Laws:} For every $a\in B$,
        \[a+a=a \textnormal{ and } a\cdot a=a.\]
      \item \textit{Universal Bound Laws:} For every $a\in B$,
        \[a+1=1 \textnormal{ and } a\cdot 0=0.\]
      \item \textit{De Morgan's Laws:} For all $a$ and $b\in B$,
        \[\overline{a+b}=\overline{a}\cdot\overline{b} \textnormal{ and }\overline{a\cdot b}=\overline{a}+\overline{b}.\]
      \item \textit{Absorption Laws:} For all $a$ and $b\in B$,
        \[(a+b)\cdot a=a \textnormal{ and } (a\cdot b)+a=a.\]
      \item \textit{Complements of $0$ and $1$:}
        \[\overline{0}=1 \textnormal{ and } \overline{1}=0.\]
    \end{enumerate}
  \end{thmBox*}
  \pagebreak

  \begin{ex*}
    Prove that for all elements $a$ in a Boolean algebra $B$:
    \begin{extasks}[after-item-skip=\stretch{1}](1)
      \task $\overline{(\overline{a})}=a$.
      \task $a+a=a$.
    \end{extasks}
    \vspace*{\stretch{1}}
  \end{ex*}
  \pagebreak

  \begin{ex*}
    Prove that for all elements $a$ in a Boolean algebra $B$:
    \begin{extasks}[after-item-skip=\stretch{1}](1)
      \task $a\cdot a=a$.
      \task $(a+b)\cdot a=a$.
    \end{extasks}
    \vspace*{\stretch{1}}
  \end{ex*}
  \pagebreak

  \begin{thmBox*}[Russell's Paradox]
    Define the following set $S$:
      \[S=\set{A\,\middle|\, A \textnormal{ is a set and } A\notin A}.\]
    Is $S$ an element of itself?
  \end{thmBox*}

  \noindent
  \textbf{The Barber Puzzle:} In a certain town, there is a male barber who shaves all those men, and only those men, who do not shave themselves.
  
  \noindent
  Does the barber shave himself? 

  \vspace*{\stretch{1}}\noindent
  Is the sentence ``The barber shaves himself'' a statement?
  \vspace*{\stretch{1}}
  \begin{ex*}
    Determine whether each sentence is a statement:
    \begin{extasks}[after-item-skip=\stretch{1}](1)
      \task If $1+1=3$, then $1=0$.
      \task This sentence is false and $1+1=3$.
    \end{extasks}
    \vspace*{\stretch{1}}
  \end{ex*}
  \pagebreak

  \begin{thmBox*}[The Halting Problem (Alan M. Turing)]
    There is no computer algorithm that will accept any algorithm $X$ and data set $D$ as input and then will output ``halts'' or ``loops forever'' to indicate whether or not $X$ terminates in a finite number of steps when $X$ is run with data set $D$.
  \end{thmBox*}

  \pagebreak
\end{document}
