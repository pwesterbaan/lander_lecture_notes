\documentclass[../mathNotesPreamble]{subfiles}

\providecommand{\relscalefact}{1.4}
\begin{document}
\relscale{\relscalefact}
  \section{1.1: Variables}
 
  \begin{defn*}
    A \textbf{variable} is a placeholder for something which may or may not be unknown.
  \end{defn*}
  
  \begin{ex*}
    Is there a number with the following property: doubling it and adding $3$ gives the same result as squaring it?
    \begin{flushright}
      $\rightarrow$ Is there a number $x$ with the property that $2x+3=x^2$?\\
      $\rightarrow$ Is there a number $\square$ with the property that $2\cdot\square+3=\square^2$?
    \end{flushright}
  \end{ex*}
  \vspace*{\stretch{1}}

  \begin{ex*}
    No matter what number might be chosen, if it is greater than $2$, then its square is greater than $4$.
    \begin{flushright}
      \parbox{0.8\linewidth}{
        \raggedleft
        $\rightarrow$ No matter what number $n$ might be chosen, if $n$ is greater than $2$, then $n^2$ is greater than $4$.
      }
    \end{flushright}
  \end{ex*}
  \vspace*{\stretch{1}}  
  \pagebreak

  \begin{ex*}
    Use variables to rewrite the following sentences:
  \end{ex*}
  \begin{extasks}[after-item-skip=\stretch{1}](1)
    \task Are there numbers with the property that the sum of their squares equals the square of their sum?
    \task Given any real number, its square is nonnegative.
  \end{extasks}
  \vspace*{\stretch{1}}

  \begin{defn*}
    \begin{itemize}
      \item A \textbf{universal statement} says that a certain property is true for all elements in a set.
      \item A \textbf{conditional statement} says that if one thing is true, then some other thing also has to be true.
      \item Given a property that may or may not be true, an \textbf{existential statement} says that there is at least one thing for which the property is true.
    \end{itemize}
  \end{defn*}
  \pagebreak

  \begin{defn*}
    A \textbf{universal conditional statement} is both universal and conditional:
    \begin{quote}
      For every animal $a$, if $a$ is a dog, then $a$ is a mammal.
    \end{quote}
    Conditional statements can be rewritten in ways that make them appear more to be purely universal or purely conditional:
    \begin{quote}
      If $a$ is a dog, then $a$ is a mammal.\\
      All dogs are mammals
    \end{quote}
  \end{defn*}
  \vspace*{\stretch{0.5}}
  \begin{ex*}
    Rewrite the following universal condition statement:
    \begin{quote}
      For every real number $x$, if $x$ is nonzero then $x^2$ is positive.
    \end{quote}
  \end{ex*}
  \vspace*{\stretch{0.5}}
  \begin{extasks}[after-item-skip=\stretch{1}](1)
    \task If a real number is nonzero, then its square \underline{\hspace*{2in}}.
    \task For every nonzero real number $x$, \underline{\hspace*{2in}}.
    \task If $x$ \underline{\hspace*{2in}}, then \underline{\hspace*{2in}}.
    \task The square of any nonzero real number is \underline{\hspace*{2in}}.
    \task All nonzero real numbers have \underline{\hspace*{2in}}.
  \end{extasks}
  \vspace*{\stretch{0.5}}
  \pagebreak

  \begin{defn*}
    A \textbf{universal existence statement} is a statement that is universal because its first part says that a certain property is true for all objects of a given type, and it is existential because its second part asserts the existence of something:
    \begin{quote}
      Every real number has an additive inverse.
    \end{quote}
    In the above example, note that the particular additive inverse depends on the given real number:
    \begin{quote}
      For every real number $r$, there is an additive inverse for $r$.
    \end{quote}
  \end{defn*}
  \vspace*{\stretch{1}}
  \begin{ex*}
    Rewrite the following universal existence statement:
    \begin{quote}
      Every pot has a lid
    \end{quote}
  \end{ex*}
  \vspace*{\stretch{0.5}}
  \begin{extasks}[after-item-skip=\stretch{1}](1)
    \task All pots \underline{\hspace{2in}}.
    \task For ever pot $P$, there is \underline{\hspace{2in}}.
    \task For every pot $P$, there is a lid $L$ such that \underline{\hspace{2in}}.
  \end{extasks}
  \vspace*{\stretch{1}}
  \pagebreak

  \begin{defn*}
    An \textbf{existential universal statement} is a statement that is existential because its first part asserts that a certain object exists and is universal because its second part says that the object satisfies a certain property for all things of a certain kind:
    \begin{quote}
      There is a positive integer that is less than or equal to every positive integer.
    \end{quote}
    The number one satisfies the above statement, which can also be rewritten:
    \begin{quote}
      There is a positive integer $m$ that is less than or equal to every positive integer.
    \end{quote}
  \end{defn*}
  \vspace*{\stretch{1}}
  \begin{ex*}
    Rewrite the following existence universal statement:
    \begin{quote}
      There is a person in my class who is at least as old as every person in my class.
    \end{quote}
  \end{ex*}
  \vspace*{\stretch{0.5}}
  \begin{extasks}[after-item-skip=\stretch{1}](1)
    \task Some \underline{\hspace*{2in}} is at least as old as \underline{\hspace*{2in}}.
    \task There is a person $p$ in my class such that $p$ is \underline{\hspace*{2in}}.
    \task There is a person $p$ in my class with the property that for every person $q$ in my class, $p$ is \underline{\hspace*{2in}}.
  \end{extasks}
  \vspace*{\stretch{0.5}}

  \pagebreak
\end{document}
