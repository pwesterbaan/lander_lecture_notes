\documentclass[../mathNotesPreamble]{subfiles}

\providecommand{\subfiles}{.}
\begin{document}
%  \relscale{1.4} %TODO
  \section{4.3: Modeling Linear Trends}
    \begin{defn*}
      The \textbf{regression line} is a model used for making predictions about \emph{future} observed values. The equation of the regression line is
        \[y=a+bx\]
      where $a$ is the \textbf{$y$-intercept} and $b$ is the \textbf{slope}.
      \begin{center}
        \begin{tikzpicture}[declare function={
          f(\x)=4*x+1.5;}]
          \begin{axis}[
            axis lines=center,
            axis line style={black,->},
            ymin=0,
            xmajorticks=false,
            ytick={1.5}, yticklabels={$a$},
            width=0.75\linewidth, height=\axisdefaultheight,
            enlargelimits={value=0.025, auto},
            xlabel=$x$, xlabel style={at={(ticklabel* cs:1)},anchor=north west},
            ylabel=$y$, ylabel style={at={(ticklabel* cs:1)},anchor=south west},
            every axis plot/.append style={line width=0.95pt, color=lander_blue, samples=255},
            nodes={black, fill=white, opacity=0.8, text opacity=1.0, inner sep=1pt, rounded corners=2.5pt}
            ]
              \addplot[->] expression[domain=0:1] {f(x)} coordinate [pos=0.35] (A) coordinate [pos=0.6] (B);
              \coordinate (MP) at (A-|B);
              \draw[dashed] (A)-|(B);
              \node[below, yshift=-5pt] at ($(A)!0.5!(MP)$) {$1$};
              \node[right, xshift=5pt] at ($(B)!0.5!(MP)$) {$b$};
          \end{axis}
        \end{tikzpicture}
      \end{center}
      
      \noindent
      \begin{minipage}{0.49\linewidth}
        The input variable $x$ is known as the
          \begin{itemize}
            \item Independent variable
            \item Predictor variable
            \item Explanatory variable
          \end{itemize}
      \end{minipage}\hfill%
      \begin{minipage}{0.49\linewidth}
        The output variable $y$ is known as the
          \begin{itemize}
            \item Dependent variable
            \item Predicted variable
            \item Response variable
          \end{itemize}
      \end{minipage}
    \end{defn*}
    \pagebreak

    \begin{ex*}
      Below is a scatterplot comparing number of pages a book has against the width of the book. Interpret the intercept and the slope of the regression line.

      \vspace*{0.5\baselineskip}
      \begin{flushright}
        \begin{tikzpicture}[nodes=black]
          \begin{axis}[
            axis lines=center,
            axis line style={black,->},
            enlargelimits={value=0.025, auto},
            height=0.35\linewidth,
            ticklabel style={font=\footnotesize,inner sep=0.5pt,fill=white,opacity=1.0, text opacity=1},
            title={Predicted Width=\pgfmathprintnumber{\pgfplotstableregressionb}\pgfmathprintnumber[print sign, precision=4]{\pgfplotstableregressiona}\,Pages},
            xlabel=Pages, xlabel style={at={(ticklabel* cs:0.5)},anchor=north, yshift=-10pt},
            ylabel=Width (mm), ylabel style={at={(ticklabel* cs:0.5)},anchor=south, xshift=-15.5pt, rotate=90},
            ]
              \addplot[only marks, blue] table [x=pages, y=width, col sep=comma] {\subfiles/scat_plot_data_4p3.csv};
              \addplot[-, black, line width=1pt] table [x=pages, y={create col/linear regression={y=width}}, col sep=comma] {\subfiles/scat_plot_data_4p3.csv};
          \end{axis}
        \end{tikzpicture}
      \end{flushright}
    \end{ex*}
    \vspace*{\stretch{1}}

    \begin{ex*}
      Below is a scatterplot comparing the number of home runs and RBIs in the 2016 season. Interpret the intercept and slope of the regression line.
      \vspace*{0.5\baselineskip}
      \begin{flushright}
        \begin{tikzpicture}[nodes=black]
          \begin{axis}[
            axis lines=center,
            axis line style={black,->},
            enlargelimits={value=0.025, auto},
            height=0.35\linewidth,
            ticklabel style={font=\footnotesize,inner sep=0.5pt,fill=white,opacity=1.0, text opacity=1},
            title={Predicted RBI=\pgfmathprintnumber{\pgfplotstableregressionb}\pgfmathprintnumber[print sign]{\pgfplotstableregressiona}\,HR},
            xlabel=Home Runs, xlabel style={at={(ticklabel* cs:0.5)},anchor=north, yshift=-10pt},
            ylabel=RBIs, ylabel style={at={(ticklabel* cs:0.5)},anchor=south, xshift=-15.5pt, rotate=90},
            every axis plot/.append style={color=black, line width=1pt},
            ]
              \addplot[-] table [x=HR, y={create col/linear regression={y=RBI}}, col sep=comma] {\subfiles/scat_plot_data_4p3.csv};
              \addplot[-, domain=8:11]{\pgfplotstableregressiona*x+\pgfplotstableregressionb};
              \addplot[only marks, blue] table [x=HR, y=RBI, col sep=comma] {\subfiles/scat_plot_data_4p3.csv};
          \end{axis}
        \end{tikzpicture}
        \hspace*{4.75mm}
      \end{flushright}
    \end{ex*}
    \vspace*{\stretch{1}}
  \pagebreak

    \begin{defn*}\vspace{0.5\baselineskip}

      \begin{minipage}{0.575\linewidth}
        Now we define the formula of the regression line:
          \[y=a+bx\]
        Where
          \[b=r\dfrac{s_y}{s_x} \quad\textnormal{ and }\quad a=\overline{y}-b\overline{x}.\]
      \end{minipage}\hspace*{\stretch{1}}%
      \begin{minipage}{0.375\linewidth}
        \begin{center}
          \begin{tikzpicture}
            \begin{axis}[
              axis lines=center,
              axis line style={black,->},
              enlargelimits={value=0.025, auto},
              width=\linewidth,
              ticklabel style={font=\footnotesize,inner sep=0.5pt,fill=white,opacity=1.0, text opacity=1},
              every axis plot/.append style={color=black, line width=1pt},
              ]
                \addplot[-] table [x=women_height, y={create col/linear regression={y=women_weight}}, col sep=comma] {\subfiles/scat_plot_data_4p2.csv};
                \addplot[only marks, blue] table [x=women_height, y=women_weight, col sep=comma] {\subfiles/scat_plot_data_4p2.csv};
                \addplot[
                  no markers, draw=none,
                  error bars/y dir=plus,
                  error bars/y explicit,
                  error bars/error mark=none,
                  error bars/error bar style=red, very thick
                ] table [
                  col sep=comma,
                  x=women_height,
                  y=women_weight,
                  y error expr=-\thisrow{women_weight}+\pgfplotstableregressiona*\thisrow{women_height}+\pgfplotstableregressionb,
                ] {\subfiles/scat_plot_data_4p2.csv};
            \end{axis}
          \end{tikzpicture}
        \end{center}
      \end{minipage}%

      These formulae minimize the residual error: \href{https://www.desmos.com/calculator/n1fvvayq5q}{\textcolor{blue}{\underline{Try this!}}}
    \end{defn*}

    \begin{ex*}
      Below are the heights and weights of six women:
        \begin{center}
          \begin{tabular}{@{}l*{6}{c}@{}}\toprule
            Heights& 61& 62& 63& 64& 66& 68\\
            Weights& 104& 110& 141& 125& 170& 160\\\bottomrule
          \end{tabular}
        \end{center}
      From this we get
        \begin{align*}
          \overline{x}&=64 & s_x&=2.608 \\
          \overline{y}&=135 & s_y&=26.728 \\
          r&=0.881
        \end{align*}
    \end{ex*}
    \vspace*{\stretch{1}}

    \noindent
    \begin{minipage}[t]{0.45\linewidth}\vspace*{0pt}
      Find the equation of the regression line.
    \end{minipage}\hspace*{\stretch{1}}%
    \begin{minipage}[t]{0.525\linewidth}\vspace*{0pt}
      \begin{flushright}
        \begin{tikzpicture}
          \begin{axis}[
            axis lines=center,
            axis line style={black,->},
            enlargelimits={value=0.025, auto},
            ticklabel style={font=\footnotesize,inner sep=0.5pt,fill=white,opacity=1.0, text opacity=1},
            every axis plot/.append style={color=black, line width=1pt},
            xlabel=Height (in), xlabel style={at={(ticklabel* cs:0.5)},anchor=north, yshift=-10pt},
            ylabel=Weight (lbs), ylabel style={at={(ticklabel* cs:0.5)},anchor=south, xshift=-20pt, rotate=90},
            ]
              \addplot[-] table [x=women_height, y={create col/linear regression={y=women_weight}}, col sep=comma] {\subfiles/scat_plot_data_4p2.csv};
              \addplot[only marks, blue] table [x=women_height, y=women_weight, col sep=comma] {\subfiles/scat_plot_data_4p2.csv};
          \end{axis}
        \end{tikzpicture}
      \end{flushright}
    \end{minipage}
  \pagebreak
  \begin{ex*}
    Open the \texttt{popdensity\_and\_crime} dataset in StatCrunch, use the ``Simple Linear'' tool under the \texttt{Stat\textrangle Regression} menu to find the regression line for the following columns. Interpret the slope and intercept where appropriate.
  \end{ex*}
  \begin{extasks}[after-item-skip=\stretch{1}](1)
    \task the \texttt{pop1990} and \texttt{pop2000} columns,
    \task the \texttt{pop2000} and \texttt{totcrimerate} columns, and
    \task the \texttt{pop2000} and \texttt{Rank Pop} columns.
  \end{extasks}
  \vspace*{\stretch{1}}
  \pagebreak



  \pagebreak
\end{document}
