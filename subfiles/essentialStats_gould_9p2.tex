\documentclass[../mathNotesPreamble]{subfiles}

\begin{document}
%  \relscale{1.4} %TODO
  \section{9.2: The Central Limit Theorem for Sample Means}
  
  \begin{defn*}[Central Limit Theorem (CLT)]\mbox{}\\
    When estimating a population mean, $\mu$, if
    \begin{enumerate}
      \item \emph{Random and Independent}: Each observation is collected randomly from the population, and observations are independent of each other.
      \item \emph{Large Sample}: Either the population distribution is Normal, or the sample size is large ($n\geq 25$).
      \item \emph{Big population}: If the sample is collected without replacement (e.g. SRS), then the population size must be at least 10 times bigger than the sample size.
        \begin{align*}
          N&\geq 10n
        \end{align*}
    \end{enumerate}
    then the sampling distribution for $\overline{x}$ is approximately Normal, with mean $\mu$ and standard deviation
      \[SE=\frac{\sigma}{\sqrt{n}}.\]
    This distribution is denoted as
      \[N\parens{\mu,\frac{\sigma}{\sqrt{n}}}.\]
  \end{defn*}
  \pagebreak

  \begin{ex*}
    The population distribution of \emph{all} emergency response times from the LA Fire Department is right-skewed. Suppose we repeatedly take random samples of a certain size from this population and calculate the mean response time. We know that the population has mean $\mu=6.3$ and standard deviation $\sigma=2.8$ minutes.
  \end{ex*}
  \begin{extasks}[after-item-skip=\stretch{1}](1)
    \task Describe the sampling distribution if the sample size is $n=9$, and again when $n=81$.
  \end{extasks}
  \begin{center}
    \begin{tikzpicture}[scale=1.0, declare function={
      N=101;
      mu=6.3; sig=2.8; 
      xmin=mu-3.2*sig;
      xmax=mu+3.2*sig;
      ymin=-0.1*gauss(mu,mu,sig);
      h=0.08*gauss(mu,mu,sig);}]

      \begin{axis}[
        every axis plot post/.append style={
        mark=none, domain={xmin}:{xmax},samples=N,smooth},
        axis lines=center,
%        ymin=ymin,
        axis line style={black,->},
        every axis x label/.style={at={(current axis.right of origin)},anchor=north},
        width=0.75*\textwidth, height=0.325*\textwidth,
        ticklabel style={font=\footnotesize,inner sep=0.5pt,fill=white,opacity=1.0, text opacity=1},
        clip=false,
        ]

        % PLOTS
        \addplot[lander_blue,thick] {gauss(x,mu,sig)} 
          node[pos=0.675, above right, black, fill=white, opacity=0.0, text opacity=1.0, font=\normalsize, align=left] {$\sigma$};
        \addplot[red,thick] {gauss(x,mu,sig/3)} 
          node[pos=0.55, right, black, fill=white, opacity=0.0, text opacity=1.0, font=\normalsize, align=left] {$\frac{\sigma}{\sqrt{9}}$};
        \addplot[purple,thick] {gauss(x,mu,sig/9)} 
          node[pos=0.5, right, black, fill=white, opacity=0.0, text opacity=1.0, font=\normalsize, align=left] {$\frac{\sigma}{\sqrt{81}}$};
      \end{axis}
    \end{tikzpicture}
  \end{center}
  \vspace*{\stretch{1}}
  
  \begin{center}
    \fbox{\parbox{0.75\linewidth}{
      \textbf{Note:} Even if the population distribution has an unusual shape, the sampling distribution is fairly symmetric and unimodal.
    }}
  \end{center}
  \pagebreak
  
  \begin{ex*}
    According to one very large study done in the US, the mean resting pulse rate of adult women is about $\mu=74$ BPM, with standard deviation $\sigma=13$ BPM, where the distribution is known to be skewed right. Suppose we take a random sample of 36 women from this population.
  \end{ex*}
  \begin{extasks}[after-item-skip=\stretch{1}](1)
    \task What is the approximate probability that the average pulse rate of this sample will be below 71 or above 77?
    \task Can we find the probability that a single adult woman, randomly selected from this population, will have a resting pulse rate more than 3 BPM away from the mean value, $\mu=74$?
  \end{extasks}
  \vspace*{\stretch{1}}
  \pagebreak
  
  \begin{defn*}[The $t$-Distribution]
    The hypothesis tests and confidence intervals we will use for estimating and testing the mean are based on the \textbf{$t$-statistic}:
    \begin{align*}
      t&=\frac{\overline{x}-\mu}{SE_{est}}\\[5pt]
      SE_{est}&=\frac{s}{\sqrt{n}}
    \end{align*}
    The $t$-statistic follows the \textbf{$t$-distribution}. With the $t$-distribution, we do not need to check conditions for the CLT, but the distribution's shape is dependent on the \textbf{degrees of freedom (df)}.
  \end{defn*}
  \vspace*{1\baselineskip}
  
  \begin{center}
    \fbox{\parbox{0.65\linewidth}{
      If we know the population standard deviation $\sigma$, then we have the familiar $z$-statistic:
      \[z=\frac{\overline{x}-\mu}{\parens{\frac{\sigma}{\sqrt{n}}}}\]
    }}
  \end{center}

  \pagebreak
\end{document}
