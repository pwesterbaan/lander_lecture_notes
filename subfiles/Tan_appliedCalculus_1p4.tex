\documentclass[../mathNotesPreamble]{subfiles}

\begin{document}
%  \relscale{1.4} %TODO
  \section{1.4: Straight Lines}
    \begin{defn*}[Slope of a Nonvertical Line]
      If $(x_1,y_1)$ and $(x_2,y_2)$ are any two distinct points on a nonvertical line $L$, then the slope $m$ of $L$ is given by
        \[m=\frac{\Delta y}{\Delta x}=\frac{y_2-y_1}{x_2-x_1}\]
    \end{defn*}

    \begin{flushright}
      \smash{\raisebox{-\height}{
      \begin{tikzpicture}[scale=1.0, declare function={
        f(\x)=\x/3+2/3;
        a=2; b=5;
        fa=f(a); fb=f(b);}]
        \begin{axis}[
          axis lines=center,
          axis line style={black,->},
          xmin=-2, xmax=8,
          enlargelimits={abs=0.75},
          ticklabel style={font=\footnotesize,inner sep=0.5pt,fill=white,opacity=1.0, text opacity=1},
          xlabel=$x$, xlabel style={at={(ticklabel* cs:1)},anchor=north west},
          ylabel=$y$, ylabel style={at={(ticklabel* cs:1)},anchor=south west},
          every axis plot/.append style={line width=0.95pt, color=lander_blue, samples=255}
          ]
          \coordinate (MP) at ({(a,fa)}-|{(b,fb)});
          \addplot[<->,domain=-2.5:8] {f(x)} node[pos=0.9, above left, black] {$L$};
          \addplot[soldot] coordinates{(a,fa)} 
            node [black, above left, font=\normalsize, inner sep=1pt]
            {$(x_1,y_1)$};
          \addplot[soldot] coordinates{(b,fb)} 
            node [black, above left, font=\normalsize, inner sep=1pt]
            {$(x_2,y_2)$};
          \draw[dashed] (a,fa) -| (b,fb);
          \node[below, font=\normalsize] at ($(a,fa)!0.5!(MP)$) {$\Delta x=x_2-x_1$};
          \node[right, font=\normalsize] at ($(b,fb)!0.5!(MP)$) {$\Delta y=y_2-y_1$};
        \end{axis}
      \end{tikzpicture}
      }}
    \end{flushright}

    \begin{minipage}{0.5\linewidth}
      \begin{ex*}
        Compute the slope of the line passing through the points
      \end{ex*}
    \end{minipage}
    \begin{extasks}[after-item-skip=\stretch{1}](1)
      \task $(x_1,y_1)=(1,1)$ and $(x_2,y_2)=(4,2)$
      \task $(x_1,y_1)=(3,2)$ and $(x_2,y_2)=(-1,2)$
      \task $(x_1,y_1)=(4,1)$ and $(x_2,y_2)=(4,4)$
    \end{extasks}
    \vspace*{\stretch{1}}
    \pagebreak
    
    \begin{defn*}[Point-Slope Form of an Equation of a Line]
      An equation of the line that has slope $m$ and passes through the point $(x_1,y_1)$ is given by
        \[y-y_1=m(x-x_1)\]
    \end{defn*}
    
    \begin{ex*}
      Find the equation of the line going through the points
    \end{ex*}
    \begin{extasks}[after-item-skip=\stretch{1}](1)
      \task $(x_1,y_1)=(-2,1)$ and $(x_2,y_2)=(3,-2)$
      \task $(x_1,y_1)=(3,4)$ and $(x_2,y_2)=(-1,4)$
      \task $(x_1,y_1)=(2,0)$ and $(x_2,y_2)=(2,1)$
    \end{extasks}
    \vspace*{\stretch{1}}
    \pagebreak
    
    \begin{defn*}[Slope-Intercept Form of an Equation of a Line]
      An equation of the line that has slope $m$ and intersects the $y$-axis at the point $(0,b)$ is given by
        \[y=mx+b\]
    \end{defn*}
    \begin{ex*}
      Rewrite the equations in the previous example in slope-intercept form.
    \end{ex*}
    \pagebreak
    
    \begin{defn*}[Parallel and Perpendicular lines]
      Let $L_1$ and $L_2$ be lines with slopes $m_1$ and $m_2$ respectively. If $L_1$ and $L_2$ are \emph{parallel}, then
        \[m_1=m_2.\]
      If $L_1$ and $L_2$ are \emph{perpendicular}, then
        \[m_1=-\frac{1}{m_2}.\]
    \end{defn*}
    
    \begin{ex*}\mbox{}
      \begin{extasks}[after-item-skip=\stretch{1}](1)
        \task Find the line \emph{parallel} to $y=\frac{3}{2}x+1$ that passes through the point $(-4,10)$.
        \task Find the line \emph{perpendicular} to $y=\frac{3}{2}x+1$ that passes through the point $(-3,4)$.
      \end{extasks}
    \end{ex*}
    \vspace*{\stretch{1}}
    \pagebreak

    \textbf{Forms of Linear Equations}\\
      \begin{align*}
        \textnormal{General form: }& Ax+By=C\\
        \textnormal{Point-slope form: }& y-y_1=m(x-x_1)\\
        \textnormal{Slope-intercept form: }& y=mx+b\\
        \textnormal{Vertical line: }& x=a\\
        \textnormal{Horizontal line: }& y=b
      \end{align*}

  \pagebreak
\end{document}
