\documentclass[../mathNotesPreamble]{subfiles}

\providecommand{\relscalefact}{1.4}
\begin{document}
\relscale{\relscalefact}
  \section{3.2: Predicates and Quantified Statements II}

  \begin{defn*}
    \begin{itemize}
      \item
        The negation of a statement of the form
          \[\forall x \textnormal{ in } D,\ Q(x)\]
        is logically equivalent to a statement of the form
          \[\exists x\textnormal{ in } D \st \sim Q(x).\]
          \[\sim\parens{\forall x\in D,\ Q(x)} \equiv \exists x\in D\st \sim Q(x).\]
      \item
        The negation of a statement of the form
          \[\exists x\textnormal{ in } D \st Q(x)\]
        is logically equivalent to a statement of the form
          \[\forall x\textnormal{ in } D,\ \sim Q(x).\]
          \[\sim\parens{\exists x\in D\st Q(x)}\equiv \forall x\in D,\ \sim Q(x)\]
    \end{itemize}
  \end{defn*}

  \begin{ex*}
    Negate the following statements:
    \begin{extasks}[after-item-skip=\stretch{1}](1)
      \task $\forall$ primes $p$, $p$ is odd
      \task $\exists$ a triangle $T$ such that the sum of the angles of $T$ equals $200^\circ$
    \end{extasks}
    \vspace*{\stretch{1}}
  \end{ex*}
  \pagebreak

  \begin{ex*}
    Rewrite the following statements formally, then write the formal and informal negations.
    \begin{extasks}[after-item-skip=\stretch{1}](1)
      \task No politicians are honest
      \task The number $1,357$ is not divisible by any integer between $1$ and $37$.
    \end{extasks}
    \vspace{\stretch{1}}
  \end{ex*}
  \begin{ex*}
    Write informal negations for the following statements:
    \begin{extasks}[after-item-skip=\stretch{1}](1)
      \task All computer programs are finite.
      \task Some computer hackers are over $40$.
    \end{extasks}
    \vspace*{\stretch{1}}
  \end{ex*}
  \pagebreak

  \begin{thmBox*}[Negation of a Universal Conditional Statement]
    \[\sim\parens{\forall x, \textnormal{ if } P(x) \textnormal{ then } Q(x)} \equiv \exists x\st P(x) \textnormal{ and } \sim Q(x)\]
  \end{thmBox*}

  \begin{defn*}
    A statement of the form
      \[\forall x \textnormal{ in } D, \textnormal{ if } P(x) \textnormal{ then } Q(x)\]
    is called \textbf{vacuously true} or \textbf{true by default} if, and only if, $P(x)$ is false for every $x$ in $D$.
  \end{defn*}
  \begin{ex*}
    The following statement is vacuously true since it's negation is false:
    \begin{quote}
      All kangaroos enrolled in my class are passing.
    \end{quote}
    \vspace*{3\baselineskip}
  \end{ex*}
  \pagebreak

  \begin{defn*}
    Consider a statement of the form $\forall x\in D$, if $P(x)$ then $Q(x)$.
    \TabPositions{90mm}
    \begin{enumerate}
      \item Its \textbf{contrapositive} is the statement \tab $\forall x\in D$, if $\sim Q(x)$ then $\sim P(x)$.
      \item Its \textbf{converse} is the statement \tab $\forall x\in D$, if $Q(x)$ then $P(x)$.
      \item Its \textbf{inverse} is the statement \tab $\forall x\in D$, if $\sim P(x)$ then $\sim Q(x)$.
    \end{enumerate}
  \end{defn*}
  \begin{ex*}
    Write a formal and informal contrapositive, converse, and inverse for the following statement:
    \begin{quote}
      If a real number is greater than $2$, then its square is greater than $4$.
    \end{quote}
    \vspace*{\stretch{1}}
  \end{ex*}
  \pagebreak

  \begin{defn*}
    \TabPositions{105mm}
    \begin{itemize}
      \item ``$\forall x, r(x)$ is a \textbf{sufficient condition} for $s(x)$'' \tab $\rightarrow$ ``$\forall x$, if $r(x)$ then $s(x)$''
        \vspace*{0.5\baselineskip}
      \item ``$\forall x, r(x)$ is a \textbf{necessary condition} for $s(x)$'' \tab $\rightarrow$ ``$\forall x$, if $\sim r(x)$ then $\sim s(x)$''

        \tab $\rightarrow$ ``$\forall x$, if $s(x)$ then $r(x)$''
        \vspace*{0.5\baselineskip}
      \item ``$\forall x, r(x)$ \textbf{only if} $s(x)$'' \tab $\rightarrow$ ``$\forall x$, if $\sim s(x)$, then $\sim r(x)$''

        \tab $\rightarrow$ ``$\forall x$, if $r(x)$ then $s(x)$''
    \end{itemize}
  \end{defn*}
  \begin{ex*}
    Rewrite each of the following as a universal conditional statement, quantified either explicitly or implicitly. Do not use the word \emph{necessary} or \emph{sufficient}.
    \begin{extasks}[after-item-skip=\stretch{1}](1)
      \task Squareness is a sufficient condition for rectangularity.
      \task Being at least $35$ years old is a necessary condition for being president of the United States.
    \end{extasks}
    \vspace*{\stretch{1}}
  \end{ex*}
  \begin{ex*}
    Rewrite the following as a universal conditional statement:
    \begin{quote}
      A product of two numbers is $0$ only if one of the numbers is $0$.
    \end{quote}
    \vspace*{\stretch{1}}
  \end{ex*}
  \pagebreak

\end{document}
