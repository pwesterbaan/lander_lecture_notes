\documentclass[../mathNotesPreamble]{subfiles}

\providecommand{\relscalefact}{1.4}
\begin{document}
\relscale{\relscalefact}
  \section{5.3: Equations and Applications with Exponential and Logarithmic Functions}
    \begin{center}
      \fbox{\parbox{0.9875\linewidth}{
        \noindent\textbf{Solving exponential equations:}
        \begin{ex*}
          Solve $\displaystyle4\parens{25^{2x}}=312,500$
          \begin{align*}
            &\textnormal{\parbox{0.58\linewidth}{1. Isolate the exponential by rewriting the equation with a base raised to a power on one side:}}& \frac{4\parens{25^{2x}}}{4}&=\frac{312,500}{4}\\[5mm]
            &\textnormal{\parbox{0.58\linewidth}{2. Take the logarithm of both sides:}}& \ln\parens{25^{2x}}&=\ln\parens{78,125}\\[5mm]
            &\textnormal{\parbox{0.58\linewidth}{3. Use a property of logarithms to remove the\newline variable from the exponent:}}& 2x\ln\parens{25}&=\ln\parens{78,125}\\[5mm]
            &\textnormal{\parbox{0.58\linewidth}{4. Solve for the variable:}}& x&=\frac{\ln\parens{78,125}}{2\ln\parens{25}}\approx 1.75
          \end{align*}
        \end{ex*}
      }}
    \end{center}
    \begin{ex*}
      Unless the exponential function uses base $e$ or base $10$, \emph{it does not matter which logarithm we use}. Solve the following exponential equation first using base $10$, then using base $e$:
        \[6\parens{4^{3x-2}}=20\]
    \end{ex*}
    \pagebreak

    \begin{ex*}
      Suppose the demand function for $q$ thousand units of a certain commodity is given by
        \[p=30\parens{3^{-q/2}}\]
    \end{ex*}
    \begin{extasks}[after-item-skip=\stretch{0.5}](1)
      \task At what price per unit will the demand equal 4000 units?
      \task How many units, to the nearest thousand units, will be demanded if the price is $\$17.31$?
    \end{extasks}
    \vspace*{\stretch{1}}
    \pagebreak

    \begin{ex*}
      A company finds that its daily sales begin to fall after the end of an advertising campaign, and the decline is such that the number of sales is $S=2000\parens{2^{-0.1x}}$, where $x$ is the number of days after the campaign ends.
    \end{ex*}
    \begin{extasks}[after-item-skip=\stretch{0.5}](1)
      \task How many sales will be made after 10 days after the end of the campaign?
      \task If the company does not want sales to drop below 350 per day, when should it start a new campaign?
    \end{extasks}
    \vspace*{\stretch{1}}
    \pagebreak

    % \begin{ex*}  %TODO
    %   Recall the formulae for interest compounded $n$ times a year
    %   \[ A=P\parens{1+\frac{r}{n}}^{nt},\]
    %   and interest compounded continously
    %   \[ A=Pe^{rt}.\]
    %   Suppose we have an investment that compounds $2$ times a year
    %   at an interest rate of $3\%$. What interest rate will earn the
    %   same amount when compounded continuously?
    % \end{ex*}
    % \pagebreak

    \begin{ex*}
      The population of a certain city was 30,000 in 2000, and 40,500 in 2010. If the formula $P=P_0e^{ht}$ applies to the growth of the city's population, what population is predicted for the year 2030?
    \end{ex*}
    \pagebreak

    \begin{ex*}
      The Gompertz equation
        \[N=100\parens{0.03}^{0.2^t}\]
      predicts the size of a deer herd on a small island $t$ decades from now.
    \end{ex*}
    \begin{extasks}[after-item-skip=4\baselineskip](1)
      \task What is the size of the deer population now ($t=0$)?
      \task During what year will the deer population reach or exceed 70?
    \end{extasks}
    \vspace*{\stretch{1}}
    \pagebreak

    \begin{ex*}
      One company's revenue from the sales of computer tablets from 2015 to 2020 can be modeled by the logistic function
        \[y=\frac{9.46}{1+53.08e^{-1.28x}}\]
      where $x$ is the number of years past 2014 and $y$ is in millions of dollars.
    \end{ex*}
    \begin{extasks}[after-item-skip=\stretch{0.25}](1)
      \task Estimate the sales revenue for 2020
      \task During what year will the sales revenue exceed $\$4$ million?
    \end{extasks}
    \vspace*{\stretch{1}}
    \pagebreak

    \begin{ex*}[Bonus]
      Solve the following for $x$:
        \[6^{x-2}=2^{-3x}\]
    \end{ex*}
    \vspace*{\stretch{1}}

  \pagebreak
\end{document}
