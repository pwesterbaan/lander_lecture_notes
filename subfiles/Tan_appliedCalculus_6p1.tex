\documentclass[../mathNotesPreamble]{subfiles}

\providecommand{\relscalefact}{1.4}
\begin{document}
\relscale{\relscalefact}
  \section{6.1: Antiderivatives and the Rules of Integration}
  \begin{defn*}[Antiderivatives]
    A function $F$ is an \textbf{antiderivative} of $f$ on an interval $I$ if $F'(x)=f(x)$ for all $x$ in $I$.
  \end{defn*}
  \begin{ex*}
    Show that $F(x)=\dfrac{1}{3}x^3-2x^2+x-1$ is an antiderivative of $f(x)=x^2-4x+1$.
  \end{ex*}
  \vspace*{\stretch{1}}
  \begin{ex*}
    Let $F(x)=x$, $G(x)=x+2$, and $H(x)=x+C$, where $C$ is a constant. Show that $F$, $G$, and $H$ are all antiderivatives of $f(x)=1$.
  \end{ex*}
  \vspace*{\stretch{1}}
  \pagebreak

  \begin{thmBox*}[Theorem 1]
    Let $G$ be an antiderivative of a function $f$ on an interval $I$. Then, every antiderivative of $F$ of $f$ on $I$ must be of the form $F(x)=G(x)+C$, where $C$ is a constant.
  \end{thmBox*}

  \begin{ex*}
    If $f'(x)=x^2$, then $f(x)=\dfrac{x^3}{3}+C$ is the family of antiderivatives of $f'(x)$.
  \end{ex*}

  \begin{center}
    \begin{tikzpicture}
      \begin{axis}[
        cycle list={black, ClemsonOrange, lander_blue, red, green!75!blue!85, ClemsonPurple},
        axis lines=center,
        axis line style={-},
        xmin=-3, xmax=3,
        ymin=-5, ymax=5,
        xmajorticks=false,
        ymajorticks=false,
        ticklabel style={font=\footnotesize,inner sep=0.5pt,fill=white,opacity=1.0, text opacity=1},
        every axis plot/.append style={line width=0.95pt, samples=100},
        height=\axisdefaultheight, width=0.65\linewidth,
        ]
        \foreach \C in {3,...,-2}
          \addplot expression[domain=-3:3]{x^3/3+\C};
      \end{axis}
    \end{tikzpicture}
  \end{center}
  \vspace*{\stretch{1}}
  \begin{defn*}[Integration]
    The process of finding the antiderivative is called \textbf{integration}:
    \[\int f(x)\,dx= F(x)+C\]
    The \textbf{indefinite integral} of $f$ is the family of functions given by $F(x)+C$ where $F'(x)=f(x)$. The function to be integrated, $f$, is called the \textbf{integrand}. $C$ is the \textbf{constant of integration}.
  \end{defn*}
  \pagebreak

  \begin{thmBox*}[Rule 1: The Indefinite Integral of a Constant]
    \vspace*{-\baselineskip}
    \begin{align*}
      \int k\,dx &= kx+C \qquad (k,\textnormal{ a constant})
      \intertext{\textbf{Rule 2: The Power Rule}}
      \int x^n\,dx &=\dfrac{1}{\npo}\, x^\npo +C \qquad (n\neq -1)
    \end{align*}
  \end{thmBox*}
  \begin{ex*}
    Find each of the following indefinite integrals
  \end{ex*}
  \begin{extasks}[after-item-skip=\stretch{1}](2)
    \task $\displaystyle \int 2\,dx$
    \task $\displaystyle \int \pi^2\,dx$
    \task $\displaystyle \int x^3\,dx$
    \task $\displaystyle \int \dfrac{1}{x^{3/2}}\,dx$
  \end{extasks}
  \vspace*{\stretch{1}}
  \pagebreak

  \begin{thmBox*}[Rule 3: The Indefinite Integral of a Constant Multiple of a Function]
    \vspace*{-\baselineskip}
    \begin{align*}
      \int cf(x)\,dx &= c\int f(x)\,dx \qquad (c, \textnormal{ a constant})
      \intertext{\textbf{Rule 4: The Sum Rule}}
      \int \sbrkt{f(x)\pm g(x)}\,dx &= \int f(x)\,dx \pm \int g(x)\,dx
    \end{align*}
  \end{thmBox*}
  \begin{ex*}
    Find each of the following indefinite integrals
  \end{ex*}
  \begin{extasks}[after-item-skip=\stretch{1}](1)
    \task $\displaystyle \int \dfrac{1}{5}-\dfrac{2}{t^3}+2t\,dt$
    \task $\displaystyle \int 3x^5+4x^{3/2}-2x^{-1/2}\,dx$
  \end{extasks}
  \vspace*{\stretch{1}}
  \pagebreak

  \begin{thmBox*}[Rule 5: The Indefinite Integral of the Exponential Function]
    \vspace*{-\baselineskip}
    \begin{align*}
      \int e^x\,dx &= e^x+C
      \intertext{\textbf{Rule 6: The Indefinite Integral of the Function \ensuremath{f(x)=x^{-1}}}}
      \int x^{-1}\,dx&= \int \dfrac{1}{x}\,dx = \ln\abs{x}+c\qquad (x\neq 0)
    \end{align*}
  \end{thmBox*}
  \begin{ex*}
    Find each of the following indefinite integrals
  \end{ex*}
  \begin{extasks}[after-item-skip=\stretch{1}](2)
    \task $\displaystyle \int 2e^x-x^3+x^e-e^e\,dx$
    \task $\displaystyle \int 2x+\dfrac{3}{x}+\dfrac{4}{x^2}\,dx$
    \task $\displaystyle \int \dfrac{2}{\sqrt{x}}-\dfrac{2}{x}\,dx$
    \task $\displaystyle \int \dfrac{1}{4e^x}-\dfrac{4}{x}+e^x\,dx$
  \end{extasks}
  \vspace*{\stretch{1}}
  \pagebreak

  \begin{thmBox*}[Rule 1: The Indefinite Integral of a Constant]
    \vspace*{-\baselineskip}
    \begin{align*}
      \int k\,dx &= kx+C \qquad (k,\textnormal{ a constant})\\[15pt]
      \intertext{\textbf{Rule 2: The Power Rule}}
      \int x^n\,dx &=\dfrac{1}{\npo}\, x^\npo +C \qquad (n\neq -1)\\[15pt]
      \intertext{\textbf{Rule 3: The Indefinite Integral of a Constant Multiple of a Function}}
      \int cf(x)\,dx &= c\int f(x)\,dx \qquad (c, \textnormal{ a constant})\\[15pt]
      \intertext{\textbf{Rule 4: The Sum Rule}}
      \int \sbrkt{f(x)\pm g(x)}\,dx &= \int f(x)\,dx \pm \int g(x)\,dx\\[15pt]
      \intertext{\textbf{Rule 5: The Indefinite Integral of the Exponential Function}}
      \int e^x\,dx &= e^x+C\\[15pt]
      \intertext{\textbf{Rule 6: The Indefinite Integral of the Function \ensuremath{f(x)=x^{-1}}}}
      \int x^{-1}\,dx&= \int \dfrac{1}{x}\,dx = \ln\abs{x}+c\qquad (x\neq 0)
    \end{align*}
  \end{thmBox*}

  \pagebreak
\end{document}
