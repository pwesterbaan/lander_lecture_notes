\documentclass[../mathNotesPreamble]{subfiles}

\begin{document}
%  \relscale{1.4} %TODO
  \section{1.1: Solutions of Linear Equations and Inequalities in One Variable}
  \begin{defn*}
    A \textbf{function} $f$ is a special relation between $x$ and $y$ such that each input $x$ results in \emph{at most} one $y$. The symbol $f(x)$ is read ``$f$ of $x$'' and is called the \textbf{value of $f$ at $x$}
  \end{defn*}
  \vspace*{0\baselineskip}
  \begin{ex*}
    Let $f(x)=4x-1$. Evaluate the following:
    \begin{extasks}[after-item-skip=\stretch{1}](2)
      \task $f\parens{1}$
      \task $f\parens{\frac{1}{2}}$
      \task $f\parens{-2}$
      \task $f\parens{0}$
      \task $f(\textnormal{\faSmileO})$
      \task $f\parens{f(x)}$
    \end{extasks}
    \vspace*{\stretch{1}}
  \end{ex*}

  \noindent\textbf{Composite Functions:}

  \noindent\fbox{\parbox{0.9875\linewidth}{
  Let $f$ and $g$ be functions of $x$. Then, the \textbf{composite functions} $g$ of $f$ (denoted $g\circ f$) and $f$ of $g$ (denoted $f\circ g$) are defined as:
  \begin{align*}
    (g\circ f)(x)&=g\parens{f(x)}\\
    (f\circ g)(x)&=f\parens{g(x)}
  \end{align*}
  }}
  \begin{ex*}
    Let $g(x)=x-1$. Find:
    \begin{extasks}[after-item-skip=\stretch{1}](2)
      \task $\parens{g\circ f}(x)$
      \task $\parens{f\circ g}(x)$
    \end{extasks}
    \vspace*{\stretch{1}}
  \end{ex*}

  \pagebreak
  \noindent\textbf{Operations with Functions:}

  \noindent\fbox{\parbox{0.9875\linewidth}{
  Let $f$ and $g$ be functions of $x$ and define the following:\\[\baselineskip]
  \begin{tabularx}{0.7\linewidth}{*{2}{X}}
    Sum& $(f+g)(x)=f(x)+g(x)$\\
    Difference& $(f-g)(x)=f(x)-g(x)$\\
    Product& $(f\cdot g)(x)=f(x)\cdot g(x)$\\
    Quotient& $\parens{\frac{f}{g}}(x)=\frac{f(x)}{g(x)}$ if $g(x)\neq 0$
  \end{tabularx}
  }}

  \vspace*{\stretch{1}}
  \begin{defn*}
    \begin{align*}
      \intertext{An \textbf{expression} is a meaningful string of numbers, variables and operations:}
      3x-2&
    \intertext{An \textbf{equation} is a statement that two quantities or algebraic expressions are equal:}
      3x-2&=7
    \intertext{A \textbf{solution} is a value of the variable that makes the equation true:}
        3(3)-2&=7\\
        9-2&=7\\
        7&=7
    \end{align*}
    A \textbf{solution set} is the set of ALL possible solutions of an equation:
      \begin{enumerate}[label=]
        \item $3x-2=7$ only has the solution $x=3$,
        \item $2(x-1)=2x-2$ is true for all possible values of $x$.
      \end{enumerate} \vspace*{-0.5\baselineskip}
  \end{defn*}
  \vspace*{\stretch{1}}
  \pagebreak

  \noindent\textbf{Properties of Equality:}
  \begin{center}
    \fbox{\parbox{0.9875\linewidth}{
      \begin{description}
        \item[Substitution Property:] The equation formed by substituting one expression for an equal expression is equivalent to the original equation:
          \begin{align*}
            3(x-3)-\frac{1}{2}(4x-18)&=4\\
            3x-9-2x+9&=4\\
            x&=4
          \end{align*}
        \item[Addition Property:] The equation formed by adding the same quantity to both sides of an equation is equivalent to the original equation:
          \begin{align*}
            x-4&=6 & x+5&=12\\
            x-4\textcolor{red}{+4}&=6\textcolor{red}{+4}& x+5\textcolor{red}{+(-5)}&=12\textcolor{red}{+(-5)}\\
            x&=10& x&=7
          \end{align*}
        \item[Multiplication Property:] The equation formed by multiplying both sides of an equation by the same \emph{nonzero} quantity is equivalent to the original equation:
          \begin{align*}
            \frac{1}{3}x&=6 & 5x&=20\\
            \textcolor{red}{3}\parens{\frac{1}{3}x}&=\textcolor{red}{3}(6) & \frac{5x}{\textcolor{red}{5}}&=\frac{20}{\textcolor{red}{5}}\\
            x&=18& x&=4
          \end{align*}
      \end{description}
    }}
  \end{center}
  \pagebreak

  \noindent\textbf{Solving a linear equation:}
  \begin{center}
    \fbox{\parbox{0.9875\linewidth}{
      Using the properties of equality above, we can solve any linear equation in $1$ variable:
      \begin{ex*}
        Solve $\displaystyle\frac{3x}{4}+3=\frac{x-1}{3}$
        \begin{align*}
          &\textnormal{1. Eliminate fractions:}& 12\parens{\frac{3x}{4}+3}&=12\parens{\frac{x-1}{3}}\\
          &\textnormal{2. Remove/evaluate parenthesis:}& 9x+36&=4x-4\\
          &\textnormal{\parbox{0.45\linewidth}{3. Use addition property to isolate the variable to one side:}}& 9x+36\textcolor{red}{-36}\textcolor{lander_blue}{-4x}&=4x-4\textcolor{red}{-36}\textcolor{lander_blue}{-4x}\\
          &\textnormal{4. Use multiplication property to isolate variable:}& \frac{5x}{5}&=\frac{-40}{5}\\
          &\textnormal{5. Verify solution via substitution:}& \underbrace{\frac{3(\textcolor{lander_blue}{-8})}{4}+3}_{\textstyle -6+3=-3}&\overset{?}{=}\underbrace{\frac{(\textcolor{lander_blue}{-8})-1}{3}}_{\textstyle \frac{-9}{3}=-3}\\
        \end{align*}
      \end{ex*}
    }}
  \end{center}

  \begin{ex*}
    Solve the following:
    \begin{extasks}[after-item-skip=\stretch{1}](2)
      \task $\displaystyle\frac{3x+1}{2}=\frac{x}{3}-3$
      \task $\displaystyle\frac{2x-1}{x-3}=4+\frac{5}{x-3}$
    \end{extasks}
  \end{ex*}
  \pagebreak

  \begin{ex*}
    Solve $-2x+6y=4$ for $y$
    \begin{flushright}
      \vspace*{-1\baselineskip}
      \begin{tikzpicture}[scale=1.0, declare function={
        f(\x)=x/3+2/3;}]
        \begin{axis}[
          axis lines=center,
          axis line style={black,->},
          xmin=-2, xmax=6,
          enlargelimits={abs=0.75},
          ticklabel style={font=\footnotesize,inner sep=0.5pt,fill=white,opacity=1.0, text opacity=1},
          xlabel=$x$, xlabel style={at={(ticklabel* cs:1)},anchor=north west},
          ylabel=$y$, ylabel style={at={(ticklabel* cs:1)},anchor=south west},
          every axis plot/.append style={line width=0.95pt, color=lander_blue, samples=255}
          ]
          \addplot[<->,domain=-2.5:6] {f(x)};
        \end{axis}
      \end{tikzpicture}
    \end{flushright}
  \end{ex*}

  \begin{ex*}
    Suppose that the relationship between a firm's profit, $P$, and the number of items sold, $x$, can be described by the equation
      \[5x-4P=1200\]
    \begin{tasks}[after-item-skip=\stretch{1}](1)
      \task How many units must be produced and sold for the firm to make a profit of \$$150$?
      \task Solve this equation for $P$ in terms of $x$. Then, find the profit when $240$ units are sold.
    \end{tasks}
  \end{ex*}
  \vspace*{\stretch{1}}
  \pagebreak
  
  \begin{ex*}
    Jill Ball has $\$90,000$ to invest. She has chosen one relatively safe investment fund that has an annual yield of $10\%$ and another riskier one that has a $15\%$ annual yield. How much should she invest in each fund if she would like to earn $\$10,000$ in one year from her investments?
  \end{ex*}
  \vspace*{\stretch{1}}
  
  \begin{ex*}
    A woman making \$2,000 per month has her salary reduced by 10\% because of sluggish sales. One year later, after a dramatic improvement in sales, she is given a 20\% raise over her reduced salary. Find her salary after the raise. What percent change is this from the \$2,000 per month?
  \end{ex*}
  \vspace*{\stretch{1}}
  \pagebreak

  \begin{defn*}
    An \textbf{inequality} is a statement that one quantity is greater than (or less than) another quantity.
  \end{defn*}

  \subfile{HarMathAp12_prop_of_ineq}
  \pagebreak

  \begin{ex*}
    Solve
      \[-x+8\leq 2x-4\]
    first by gathering the $x$ variable on the left, then again on the right. See that the multiplication property holds in both cases. Plot the solution set on a numberline.

    \begin{flushright}
      \begin{tikzpicture}[scale=1.0, declare function={
        f(\x)=-x+8;
        g(\x)=2*x-4;}]
        \begin{axis}[
          axis lines=center,
          axis line style={black,->},
          xmin=-2, xmax=6,
          ymin=-2,
          enlargelimits={abs=0.75},
          ticklabel style={font=\footnotesize,inner sep=0.5pt,fill=white,opacity=1.0, text opacity=1},
          xlabel=$x$, xlabel style={at={(ticklabel* cs:1)},anchor=north west},
          ylabel=$y$, ylabel style={at={(ticklabel* cs:1)},anchor=south west},
          every axis plot/.append style={line width=0.95pt, color=lander_blue, samples=255}
          ]
          \addplot[<->,domain=-2:6] {f(x)};
          \addplot[<->, red, domain=1:6] {g(x)};
        \end{axis}
      \end{tikzpicture}
      \vspace*{-\baselineskip}

      \begin{tikzpicture}
        \begin{axis}[
          axis y line=none,
          axis x line=middle,
          axis line style={black,->},
          xmin=-5, xmax=5,
          xtick={-5,-4,...,5},
          height=75mm,
          width=0.75\linewidth,
          enlargelimits={abs=0.75},
          ticklabel style={font=\footnotesize,inner sep=0.5pt,fill=white,opacity=1.0, text opacity=1},
          xlabel=\phantom{$x$}, xlabel style={at={(ticklabel* cs:1)},anchor=north west},
          every axis plot/.append style={line width=0.95pt, color=lander_blue, samples=255}
          ]
          \addplot[draw=none] expression[domain=-5:5]{1};
        \end{axis}
      \end{tikzpicture}
    \end{flushright}
  \end{ex*}
   \vspace*{\stretch{1}}
   \begin{ex*}
     Plot the following inequalities:
   \end{ex*}
   \newcommand{\numbLine}{
     \begin{tikzpicture}
        \begin{axis}[
          axis y line=none,
          axis x line=middle,
          axis line style={black,->},
          xmin=-4, xmax=4,
          xtick={-5,-4,...,5},
          height=75mm,
          width=1.1\linewidth,
          enlargelimits={abs=0.75},
          ticklabel style={font=\footnotesize,inner sep=0.5pt,fill=white,opacity=1.0, text opacity=1},
          xlabel=\phantom{$x$}, xlabel style={at={(ticklabel* cs:1)},anchor=north west},
          every axis plot/.append style={line width=0.95pt, color=lander_blue, samples=255}
          ]
          \addplot[draw=none] expression[domain=-4:4]{1};
        \end{axis}
      \end{tikzpicture}}
   \noindent\begin{minipage}{0.5\linewidth}
     \begin{center}
       $x\leq 2$\\[2\baselineskip]
       \numbLine
     \end{center}
   \end{minipage}%
   \begin{minipage}{0.5\linewidth}
     \begin{center}
       $x>-3$\\[2\baselineskip]
       \numbLine
     \end{center}
   \end{minipage}%

  \pagebreak
\end{document}
