\documentclass[../mathNotesPreamble]{subfiles}

\providecommand{\relscalefact}{1.4}
\begin{document}
\relscale{\relscalefact}
  \section{2.5: Application: Number Systems and Circuits for Addition}

  Recall our how we write numbers in base 10:
  \begin{alignat*}{5}
%    r&l&r&l&r&l&r&l&r&l\\
    5,049&=&5\,\cdot\,&1000 &+0\,\cdot\,&100 &+4\,\cdot\,&10 &+9\,\cdot\,&1\\
      &=&5\,\cdot\,&10^3 &+0\,\cdot\,&10^2 &+4\,\cdot\,&10^1 &+9\,\cdot\,&10^0
  \end{alignat*}
  \vspace*{\stretch{1}}

  \begin{defn*}
    Any integer $b>1$ can be used as a base for a numbering system. A numbering system of base $b$ has the digits $0,1,\dots, b-1$.

    A \textbf{base $2$ notation} or \textbf{binary notation}, uses the digits $0,1$. In binary, every integer is represented as sum of products of the form
      \[d\cdot 2^n\]
    where $n\in \bbz$ and $d\in\set{0,1}$.
  \end{defn*}
  
  \begin{ex*}
    Below is the binary representation for the integers $1$ to $9$:
    \begin{center}
      \renewcommand{\arraystretch}{1.35}
      \begin{tabular}{@{}*{10}{C}R@{}}
1_{10} & = &  &  &  &  &  &  & 1\cdot 2^0 & = & 1_2 \\
2_{10} & = &  &  &  &  & 1\cdot 2^1 & + & 0\cdot 2^0 & = & 10_2 \\
3_{10} & = &  &  &  &  & 1\cdot 2^1 & + & 1\cdot 2^0 & = & 11_2 \\
4_{10} & = &  &  & 1\cdot 2^2 & + & 0\cdot 2^1 & + & 0\cdot 2^0 & = & 100_2 \\
5_{10} & = &  &  & 1\cdot 2^2 & + & 0\cdot 2^1 & + & 1\cdot 2^0 & = & 101_2 \\
6_{10} & = &  &  & 1\cdot 2^2 & + & 1\cdot 2^1 & + & 0\cdot 2^0 & = & 110_2 \\
7_{10} & = &  &  & 1\cdot 2^2 & + & 1\cdot 2^1 & + & 1\cdot 2^0 & = & 111_2 \\
8_{10} & = & 1\cdot 2^3 & + & 0\cdot 2^2 & + & 0\cdot 2^1 & + & 0\cdot 2^0 & = & 1000_2 \\
9_{10} & = & 1\cdot 2^3 & + & 0\cdot 2^2 & + & 0\cdot 2^1 & + & 1\cdot 2^0 & = & 1001_2 \\
      \end{tabular}
    \end{center}
  \end{ex*}
  \pagebreak

  \begin{thmBox*}[Converting binary $\rightarrow$ decimal:]
    To convert from binary to decimal, multiply each digit by its corresponding power of $2$ and sum the results.
  \end{thmBox*}

  \begin{ex*}
    Represent the following in decimal notation (base-$10$):
    \begin{extasks}[after-item-skip=\stretch{1}](2)
      \task $110_2$
      \task $1011_2$
      \task $11110_2$
      \task $101011_2$
    \end{extasks}
    \vspace*{\stretch{1}}
  \end{ex*}
  \pagebreak

  \begin{thmBox*}[Converting decimal $\rightarrow$ binary:]
    To convert from decimal to binary, we repeated divide by $2$, and record the remainders.
  \end{thmBox*}
  \begin{ex*}
    \begin{align*}
      27_{10}&=16+8+2+1\\
        &=1\cdot 2^4+1\cdot 2^3+0\cdot2^2+1\cdot2^1+1\cdot2^0\\
        &=11011_2
    \end{align*}
  \end{ex*}

  \begin{ex*}
    Represent the following in binary notation:
    \begin{extasks}[after-item-skip=\stretch{1}](2)
      \task $243_{10}$
      \task $587_{10}$
      \task $990_{10}$
      \task $531_{10}$
    \end{extasks}
    \vspace*{\stretch{1}}
  \end{ex*}
  \pagebreak

  \begin{thmBox*}[Binary arithmetic:]
    In binary arithmetic, $10_2$ behaves similarly to $10$ in decimal arithmetic.
  \end{thmBox*}

  \begin{ex*}
    Add $1101_2$ and $111_2$ using binary notation.
  \end{ex*}
  \vspace*{\stretch{1}}

  \begin{ex*}
    Subtract $1011_2$ from $11000_2$ using binary notation.
  \end{ex*}
  \vspace*{\stretch{1}}
  \pagebreak

  \begin{defn*}
    \textbf{The $8$-bit two's complement} for an integer $a$ between $-128$ and $127$ is the $8$-bit binary representation for 
      \[\begin{cases}
        a,& \textnormal{if } a\geq 0\\
        2^8-\abs{a},& \textnormal{if } a<0.
      \end{cases}\]
    Two's complement allows maximum representation for $2^8$ integers with $8$ binary digits.
  \end{defn*}

  \begin{ex*}
    Below are a few integers represented in binary using $8$-bit two's complement:
    \begin{align*}
      -128 \rightarrow 2^8-\abs{-128}=128_{10}&= 10000000_2&
      0 \rightarrow 0_{10}&= 00000000_2\\[10pt]
      -127 \rightarrow 2^8-\abs{-127}=129_{10}&= 10000001_2&
      1 \rightarrow 1_{10}&= 00000001_2\\[10pt]
      &\vdots&
      2 \rightarrow 2_{10}&= 00000010_2\\[10pt]
      -2 \rightarrow 2^8-\abs{-2}=254_{10}&= 10000000_2&
      & \vdots \\[10pt]
      -1 \rightarrow 2^8-\abs{-1}=255_{10}&= 11111111_2 \hspace*{50pt}&
      127 \rightarrow 127_{10}&= 01111111_2
    \end{align*}
  \end{ex*}
  \vspace*{\baselineskip}
  
  \begin{ex*}
    Find the $8$-bit two's complement for the following:
    \begin{extasks}[after-item-skip=\stretch{1}](2)
      \task $-46$
      \task $42$
      \task $120$
      \task $-82$
    \end{extasks}
    \vspace*{\stretch{1}}
  \end{ex*}
  \pagebreak

  \begin{thmBox*}[Two's complement of a negative integer:]
    To find the decimal representation of the negative integer with a given $8$-bit two's complement:
    \begin{itemize}
      \item Flip the bits
      \item Add $1$
      \item Convert to base-$10$ and swap the sign
    \end{itemize}
  \end{thmBox*}

  \begin{ex*}
    Find the decimal representation of the integers with the following $8$-bit two's complement:
    \begin{extasks}[after-item-skip=\stretch{1}](2)
      \task $11100101_2$
      \task $11000000_2$
    \end{extasks}
    \vspace*{\stretch{1.5}}
  \end{ex*}

  \begin{thmBox*}[Addition and Subtraction with Integers in Two's Complement Form:]
    When performing binary addition on integers written in Two's Complement form, we discard any ``carry'' bit.
  \end{thmBox*}
  \begin{ex*}
    Perform binary addition using the Two's Complement form of the following:
    \begin{extasks}[after-item-skip=\stretch{1}](2)
      \task $-87$ and $-46$
      \task $83$ and $-55$
    \end{extasks}
    \vspace*{\stretch{1}}
  \end{ex*}
  \pagebreak

  \begin{defn*}
    \textbf{Hexadecimal notation} uses a \textbf{base $16$ notation}. In hexadecimal, every integer is represented as sum of products of the form
      \[d\cdot 16^n\]
    where $n\in \bbz$ and $d\in\set{0,1,\dots,9,A,B,C,D,E,F}$.
  \end{defn*}
  \vspace*{\baselineskip}

  \begin{center}
    \setlength{\tabcolsep}{18pt}
    \begin{tabular}{@{}ccc@{}}
      \toprule
      Decimal & Hexadecimal & $4$-Bit Binary\\
      \midrule
      0 & 0 & 0000\\
      1 & 1 & 0001\\
      2 & 2 & 0010\\
      3 & 3 & 0011\\
      4 & 4 & 0100\\
      5 & 5 & 0101\\
      6 & 6 & 0110\\
      7 & 7 & 0111\\
      8 & 8 & 1000\\
      9 & 9 & 1001\\
      10 & A & 1010\\
      11 & B & 1011\\
      12 & C & 1100\\
      13 & D & 1101\\
      14 & E & 1110\\
      15 & F & 1111\\
      \bottomrule
    \end{tabular}
  \end{center}
  \pagebreak

  \begin{ex*}
    Convert $3CF_{16}$ to decimal notation.
    \vspace*{\stretch{1}}
  \end{ex*}
  \begin{ex*}
    Convert $B09F_{16}$ to binary notation.
    \vspace*{\stretch{1}}
  \end{ex*}
  \begin{ex*}
    Convert $100110110101001_2$ to hexadecimal notation.
    \vspace*{\stretch{1}}
  \end{ex*}

  \pagebreak
\end{document}
