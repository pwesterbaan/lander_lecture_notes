\documentclass[../mathNotesPreamble]{subfiles}

\begin{document}
% \relscale{1.4} %TODO
  \section{1.4: Organizing Categorical Data}
    \begin{defn*}
      In the context of statistics, \textbf{frequency} is the number of times a value of a variable is observed in a data set.\\
\textbf{Relative frequency} (proportion) is a ratio of the frequency of a variable to the total frequency of the group desired. This can be left as a fraction, decimal, or percentage.
    \end{defn*}

    \begin{ex*}
      The following \textbf{two-way table} contains the results of a national survey that asks American youths whether they wear a seat belt while driving or riding in a car:
    \end{ex*}
    \begin{center}
      \begin{tabular}{@{}l*{3}{c}@{}}\toprule
        & Male & Female & Total\\\midrule
        Not Always& 2& 3& \\
        Always& 3& 7& \\\midrule
        Total\\\bottomrule
      \end{tabular}
    \end{center}
    \begin{enumerate}[a)]
      \item Find the total number of males, females, and total participants in this survey.
      \item Identify the frequencies, and compute the percentages below:
        \begin{center}
          \hspace*{-30pt}
          \begin{tabular}{@{}l*{3}{c}@{}}\toprule
            & Male & Female & Total\\\midrule
            Not Always& & & \\
            Always& & & \\\midrule
            Total&&&100\%\\\bottomrule
          \end{tabular}
        \end{center}
      \item Are males or females more likely to take the risk of not wearing a seat belt?\\[\stretch{1}]
      \item Should we use the frequencies or the relative frequencies to make comparisons?
    \end{enumerate}

  \pagebreak
\end{document}
