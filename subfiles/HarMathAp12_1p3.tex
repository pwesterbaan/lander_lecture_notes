\documentclass[../mathNotesPreamble]{subfiles}

\providecommand{\relscalefact}{1.4}
\begin{document}
\relscale{\relscalefact}
  \section{1.3: Linear Functions}
    \begin{defn*}
      A \textbf{linear function} is a function of the form
        \[y=f(x)=mx+b\]
      where $m$ and $b$ are constants.
    \end{defn*}

    \noindent
%    \begin{minipage}[t]{0.5\linewidth}
      \begin{ex*}
        $y=-2x+8$
      \end{ex*}
      \vspace*{-\baselineskip}
%    \end{minipage}%
%    \begin{minipage}[t]{0.5\linewidth}
      \begin{flushright}
        \begin{tikzpicture}[scale=1.0, declare function={
          f(\x)=-2*x+8;}]
          \begin{axis}[
            axis lines=center,
            axis line style={black,->},
%            xmin=-2, xmax=6,
%            ymin=-2,
            enlargelimits={abs=0.75},
            ticklabel style={font=\footnotesize,inner sep=0.5pt,fill=white,opacity=1.0, text opacity=1},
            xlabel=$x$, xlabel style={at={(ticklabel* cs:1)},anchor=north west},
            ylabel=$y$, ylabel style={at={(ticklabel* cs:1)},anchor=south west},
            every axis plot/.append style={line width=0.95pt, color=lander_blue, samples=255}
            ]
            \addplot[<->,domain=-1:5] {f(x)};
            \addplot[soldot, lander_blue] coordinates{(0,8)} node[above right] {$(0,8)$};
            \addplot[soldot, lander_blue] coordinates{(4,0)} node[above right] {$(4,0)$};
          \end{axis}
        \end{tikzpicture}
      \end{flushright}
%    \end{minipage}

    \begin{center}
      A linear function can be uniquely determined using only \emph{two} distinct points.
    \end{center}
    \begin{defn*}
      The point(s) where a graph intersects the axes are called intercepts. The $x$-coordinate of the point where the function intersects the $x$-axis is called the \textbf{$x$-intercepts}. The $y$-coordinate of the point where the function intersects the $y$-axis is called the \textbf{$y$-intercepts}.
    \end{defn*}
    \begin{tasks}[label=\textbullet](2)
      \task To solve for the $y$-intercept:
        \begin{itemize}
          \item Set $x=0$,
          \item Solve for $y$.
        \end{itemize}
      \task To solve for the $x$-intercept:
        \begin{itemize}
          \item Set $y=0$,
          \item Solve for $x$.
        \end{itemize}
    \end{tasks}
    \pagebreak

    \begin{ex*}
      Find the intercepts and graph the following lines:
      \begin{extasks}[after-item-skip=\stretch{1}](2)
        \task $3x+2y=12$\\
        \task $x=4y$
        \task
          \begin{tikzpicture}[scale=1.0, declare function={
            f(\x)=-1.5*\x+6;}]
            \begin{axis}[
              axis lines=center,
              axis line style={black,->},
              enlargelimits={abs=0.75},
              ticklabel style={font=\footnotesize,inner sep=0.5pt,fill=white,opacity=1.0, text opacity=1},
              xlabel=$x$, xlabel style={at={(ticklabel* cs:1)},anchor=north west},
              ylabel=$y$, ylabel style={at={(ticklabel* cs:1)},anchor=south west},
              every axis plot/.append style={line width=0.95pt, color=lander_blue, samples=255}
              ]
              \addplot[draw=none,<->] expression[domain=-1:5]{f(x)};
            \end{axis}
          \end{tikzpicture}
        \task
          \begin{tikzpicture}[scale=1.0, declare function={
            f(\x)=0.25*\x;}]
            \begin{axis}[
              axis lines=center,
              axis line style={black,->},
              enlargelimits={abs=0.75},
              ticklabel style={font=\footnotesize,inner sep=0.5pt,fill=white,opacity=1.0, text opacity=1},
              xlabel=$x$, xlabel style={at={(ticklabel* cs:1)},anchor=north west},
              ylabel=$y$, ylabel style={at={(ticklabel* cs:1)},anchor=south west},
              every axis plot/.append style={line width=0.95pt, color=lander_blue, samples=255}
              ]
              \addplot[draw=none,<->] expression[domain=-4:4]{f(x)};
            \end{axis}
          \end{tikzpicture}
      \end{extasks}
    \end{ex*}
    \pagebreak

    \begin{defn*}
      If a nonvertical line passes through the points $P_1(x_1,y_1)$ and $P_2(x_2,y_2)$, its \textbf{slope}, denoted by $m$, is found using
        \[m=\frac{y_2-y_1}{x_2-x_1}=\frac{\Delta y}{\Delta x}\]
      $\Delta y$ is ``delta $y$'', and represents the change in $y$\newline
      $\Delta x$ is ``delta $x$'', and represents the change in $x$
    \end{defn*}
    \emph{Note:} The slope of a vertical line is undefined.
    \begin{ex*}
      Find the slope of the line passing through the points $(-2,1)$ and $(5,3)$.
    \end{ex*}
    \vspace*{\stretch{1}}
    \noindent
    \fbox{\parbox{0.9875\linewidth}{
      \emph{Note:}
      \begin{itemize}
        \item Two distinct nonvertical lines are \emph{parallel} if and only if their slopes are \emph{equal}.
        \item Two distinct nonvertical lines are \emph{perpendicular} if and only if their slopes are \emph{negative reciprocals}:\newline
          e.g. If $\ell_1$ has a nonzero slope $m$, then $\ell_2$ is perpendicular if its slope is $-\sfrac{1}{m}$.
      \end{itemize}
    }}
    \pagebreak

    \noindent\textbf{Point-slope form}
    \begin{defn*}
      The equation of the line passing through the point $(x_1,y_1)$ with slope $m$ can be written in the point-slope form:
        \[y-y_1=m\parens{x-x_1}\]
    \end{defn*}
    \begin{ex*}
      Find the equation of each line that passes through the point $(-3,4)$ and has
      \begin{extasks}[after-item-skip=\stretch{1}](2)
        \task a slope of $\displaystyle m=\frac{1}{4}$
        \task the point $(-2,1)$ on the line
        \task a slope of zero (horizontal)
        \task an undefined slope (vertical)
      \end{extasks}
      \vspace*{\stretch{1}}
    \end{ex*}
    \pagebreak

    \noindent\textbf{Slope-intercept form}
    \begin{defn*}
      The slope-intercept form of the equation of a line with slope $m$ and $y$-intercept $b$ is
        \[y=mx+b\]
    \end{defn*}
    \begin{ex*}[Example 7, p.82]
      The population of U.S. males, $y$ (in thousands), projected from $2015$ to $2060$ can be modeled by
        \[y=1125.9x+142,960\]
      where $x$ is the number of years after $2000$.
      \begin{itemize}
        \setlength{\itemsep}{\stretch{1}}
        \item Find the slope and $y$-intercept of the graph of this function.
        \item What does the $y$-intercept tell us about the population of U.S. males?
        \item Interpret the slope as a rate of change.
      \end{itemize}
      \vspace*{\stretch{1}}
    \end{ex*}
    \pagebreak

    \begin{ex*}
      Each day, a young person should sleep 8 hours plus $\frac{1}{4}$ hour for each year the person is under $18$ years of age. Assuming that the relation is linear, write the equation relating hours of sleep $y$ and age $x$
    \end{ex*}
    \vspace*{\stretch{1}}

    \textbf{Forms of Linear Equations}\\
      \begin{align*}
        \textnormal{General form: }& Ax+By=C\\
        \textnormal{Point-slope form: }& y-y_1=m(x-x_1)\\
        \textnormal{Slope-intercept form: }& y=mx+b\\
        \textnormal{Vertical line: }& x=a\\
        \textnormal{Horizontal line: }& y=b
      \end{align*}

    \pagebreak
\end{document}
